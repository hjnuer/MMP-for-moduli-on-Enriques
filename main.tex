\documentclass[leqno,11pt]{amsart}

%\documentclass{jams-l}

\usepackage[letterpaper,margin=1in]{geometry}

%\usepackage{times}

%\usepackage[T1]{fontenc}
%\usepackage[latin1]{inputenc}
%\usepackage{a4}
\usepackage[all]{xy} % diagrams
%\xyoption{all}
\usepackage{amsmath, amssymb, amsfonts, latexsym, mdwlist, amsthm, amscd}
\usepackage{subfig}
\usepackage{graphicx}
%\usepackage{floatflt}
\usepackage{wrapfig}
\usepackage{showkeys}


\usepackage[bookmarks, colorlinks, breaklinks, pdftitle={},
pdfauthor={}]{hyperref}
\hypersetup{linkcolor=blue,citecolor=blue,filecolor=black,urlcolor=blue}

%\usepackage{cite}

%Uncomment this for final versions:
%\usepackage{showkeys}

%\sloppy

%The following are tools for tikz:
\usepackage{tikz}
\usetikzlibrary{calc,trees,positioning,arrows,chains,shapes.geometric,%
    decorations.pathreplacing,decorations.pathmorphing,shapes,%
    matrix,shapes.symbols}

\tikzset{
>=stealth',
  punktchain/.style={
    rectangle,
    rounded corners,
    % fill=black!10,
    draw=black, thick,
    %text width=4em,
    minimum height=3em,
    text centered,
    on chain},
  line/.style={draw, thick, <-},
  element/.style={
    tape,
    top color=white,
    bottom color=blue!50!black!60!,
    minimum width=8em,
    draw=blue!40!black!90, very thick,
    text width=10em,
    minimum height=3.5em,
    text centered,
    on chain},
  every join/.style={->, thick,shorten >=1pt},
  decoration={brace},
  tuborg/.style={decorate},
  tubnode/.style={midway, right=2pt},
}


%alphabetical enumerate
\usepackage{paralist}
\usepackage{enumitem} 
\setdefaultenum{(a)}{(i)}{}{}
\setlist[enumerate,1]{label={\upshape(\alph*)}}
\setlist[enumerate,2]{label={\upshape(\roman*)}}
% for space-saving description environments}

%%%%%%%%%%%%%%%%%%%% Some abbreviations %%%%%%%%%%%%%%
\def\C{\ensuremath{\mathbb{C}}}
\def\D{\ensuremath{\mathbb{D}}}
\def\H{\ensuremath{\mathbb{H}}}
%\def\L{\ensuremath{\mathbb{L}}}
\def\N{\ensuremath{\mathbb{N}}}
\def\P{\ensuremath{\mathbb{P}}}
\def\Q{\ensuremath{\mathbb{Q}}}
\def\R{\ensuremath{\mathbb{R}}}
\def\V{\ensuremath{\mathbb{V}}}
\def\Z{\ensuremath{\mathbb{Z}}}

\newcommand{\dotcup}{\ensuremath{\mathaccent\cdot\cup}}

\def\alg{\mathrm{alg}}
\def\Amp{\mathrm{Amp}}
\def\Aut{\mathop{\mathrm{Aut}}\nolimits}
\def\lAut{\mathop{\mathcal Aut}\nolimits}
\def\ch{\mathop{\mathrm{ch}}\nolimits}
\def\Char{\mathop{\mathrm{char}}}
\def\CH{\mathop{\mathrm{CH}}}
\def\Coh{\mathop{\mathrm{Coh}}\nolimits}
\def\codim{\mathop{\mathrm{codim}}\nolimits}
\def\cone{\mathop{\mathrm{cone}}}
\def\cok{\mathop{\mathrm{cok}}}
\def\deg{\mathop{\mathrm{deg}}}
\def\diag{\mathop{\mathrm{diag}}\nolimits}
\def\dim{\mathop{\mathrm{dim}}\nolimits}
\def\ev{\mathop{\mathrm{ev}}\nolimits}
\def\inf{\mathop{\mathrm{inf}}\nolimits}
\def\End{\mathop{\mathrm{End}}}
\def\lEnd{\mathop{\mathcal End}}
\def\Ext{\mathop{\mathrm{Ext}}\nolimits}
\def\ext{\mathop{\mathrm{ext}}\nolimits}
\def\lExt{\mathop{\mathcal Ext}\nolimits} % means local Ext
\def\Fix{\mathop{\mathrm{Fix}(G)}}
\def\FixG{\mathop{\mathrm{Fix}(G')}}
\def\For{\mathop{\mathrm{Forg}_G}}
\def\Func{\mathop{{\mathrm{Func}}}\nolimits}
\def\GL{\mathop{\mathrm{GL}}}
\def\Hal{H^*_{\alg}}
\def\Hilb{\mathop{\mathrm{Hilb}}\nolimits}
\def\HN{\mathop{\mathrm{HN}}\nolimits}
\def\Hom{\mathop{\mathrm{Hom}}\nolimits}
%\def\lHom{\mathop{\underline{\mathrm{Hom}}}\nolimits} % means local Hom
\def\lHom{\mathop{\mathcal Hom}\nolimits}
\def\RlHom{\mathop{\mathbf{R}\mathcal Hom}\nolimits}
\def\RHom{\mathop{\mathbf{R}\mathrm{Hom}}\nolimits}
\def\id{\mathop{\mathrm{id}}\nolimits}
\def\Id{\mathop{\mathrm{Id}}\nolimits}
\def\im{\mathop{\mathrm{im}}\nolimits}
\def\coker{\mathop{\mathrm{coker}}\nolimits}
\def\Imm{\mathop{\mathrm{Im}}\nolimits}
\def\Inf{\mathop{\mathrm{Inf}_G}}
\def\Isom{\mathop{\mathrm{Isom}}\nolimits}
\def\Jac{\mathop{\mathrm{Jac}}\nolimits}
\def\JH{\mathop{\mathrm{JH}}\nolimits}
\def\Ker{\mathop{\mathrm{Ker}}\nolimits}
\def\Lag{\mathop{\mathrm{Lag}}\nolimits}
\def\lcm{\mathop{\mathrm{lcm}}\nolimits}
\def\Lie{\mathop{\mathrm{Lie}}\nolimits}
\def\Mor{\mathop{\mathrm{Mor}}\nolimits}
\def\mod{\mathop{\mathrm{mod}}\nolimits}
\def\min{\mathop{\mathrm{min}}\nolimits}
\def\mult{\mathop{\mathrm{mult}}\nolimits}
\def\Nef{\mathrm{Nef}}
\def\num{\mathop{\mathrm{num}}\nolimits}
\def\Num{\mathop{\mathrm{Num}}\nolimits}
\def\NS{\mathop{\mathrm{NS}}\nolimits}
\def\ord{\mathop{\mathrm{ord}}\nolimits}
\def\Ob{\mathop{\mathrm{Ob}}}
\def\perf{\mathop{\mathrm{perf}}}
\def\Pic{\mathop{\mathrm{Pic}}\nolimits}
\def\PGL{\mathop{\mathrm{PGL}}}
\def\Proj{\mathop{\mathrm{Proj}}}
\def\Quot{\mathop{\mathrm{Quot}}\nolimits}
\def\rk{\mathop{\mathrm{rk}}}
\def\Sch{\mathop{\mathrm{Sch}}\nolimits}
\def\Sing{\mathop{\mathrm{Sing}}}
\def\Spec{\mathop{\mathrm{Spec}}}
\def\lSpec{\mathop{\mathcal Spec}\nolimits}
\def\SL{\mathop{\mathrm{SL}}\nolimits}
\def\SSL{\mathop{\mathrm{SSL}}}
\def\ST{\mathop{\mathrm{ST}}\nolimits}
\def\stab{\mathop{\mathrm{stab}}}
\def\supp{\mathop{\mathrm{supp}}}
\def\Sym{\mathop{\mathrm{Sym}}\nolimits}
\def\Tor{\mathop{\mathrm{Tor}}\nolimits}
\def\Tr{\mathop{\mathrm{Tr}}\nolimits}
\def\td{\mathop{\mathrm{td}}\nolimits}
\def\Real{\mathop{\mathrm{Re}}\nolimits}
\def\Res{\mathop{\mathrm{Res}}\nolimits}
\def\top{\mathop{\mathrm{top}}\nolimits}
\def\Tot{\mathop{\mathrm{Tot}}\nolimits}
\def\virt{\mathrm{virt}}
\def\v{\mathop{\pi^*v}\nolimits}
\def\Cone{\mathop{\mathrm{Cone}}\nolimits}
\def\chr{\mathop{\mathrm{char}}\nolimits}

\def\o{\mathop{\ord(\omega_S)}\nolimits}
\def\DT{\mathop{\mathrm{DT}}}
\def\PT{\mathop{\mathrm{PT}}}

\newenvironment{Prf}{\textit{Proof.}\/}{\hfill$\Box$}


\def\MG13{\ensuremath{{\mathcal M}_{\Gamma_1(3)}}}
\def\tildeMG13{\ensuremath{\widetilde{\mathcal M}_{\Gamma_1(3)}}}
\def\Stab{\mathop{\mathrm{Stab}}}
\def\Stabd{\mathop{\Stab^{\dagger}}}
\def\into{\ensuremath{\hookrightarrow}}
\def\onto{\ensuremath{\twoheadrightarrow}}

\def\blank{\underline{\hphantom{A}}}

%%%%%Macro-added%%%%%%%%%%%

\def\Db{\mathrm{D}^{\mathrm{b}}}


%%%%%%%%%%%%%%%%%%%%%%%



\def\pt{[\mathrm{pt}]}
\def\Pz{\P^2}

\def\shiftto{\to^{[1]}}


%\newcommand\TFILTB[3]{%
%  #1  an object to filtrate
%  #2  quotients
%  #3  end of the filtration
%  Example \TFILTB E A n
%\xymatrix@=1pc{
%{0 = {#1}_0} \ar[rr]&&
%{{#1}_1} \ar[rr]\ar[ld] &&
%{{#1}_2} \ar[r]\ar[ld] &
%{\cdots} \ar[r] & { {#1}_{#3-1}} \ar[rr] &&
%{{#1}_{#3} = {#1}} \ar[ld]
%\\
%& *{{#2}_1} \ar@{.>}[ul] &&
%{{#2}_2} \ar@{.>}[ul] & &&&
%{{#2}_{{#3}}} \ar@{.>}[ul]
%}}


%\newcommand{\com}{{\scriptscriptstyle\bullet}}



\def\abs#1{\left\lvert#1\right\rvert}

\newcommand\stv[2]{\left\{#1\,\colon\,#2\right\}}

% Allows for repeating a theorem number:
\makeatletter
\newtheorem*{rep@theorem}{\rep@title}
\newcommand{\newreptheorem}[2]{%
\newenvironment{rep#1}[1]{%
 \def\rep@title{#2 \ref{##1}}%
 \begin{rep@theorem}}%
 {\end{rep@theorem}}}
\makeatother


%\swapnumbers
\newtheorem{Thm}{Theorem}[section]
\newreptheorem{Thm}{Theorem}
%\newtheorem{Thm-s}[Thm]{Theorem}
%\newtheorem{Prop-s}[Thm]{Proposition}
%\newtheorem{Lem-s}[Thm]{Lemma}
\newtheorem{Prop}[Thm]{Proposition}
\newtheorem{PropDef}[Thm]{Proposition and Definition}
\newtheorem{Lem}[Thm]{Lemma}
\newtheorem{PosLem}[Thm]{Positivity Lemma}
\newtheorem{Cor}[Thm]{Corollary}
\newreptheorem{Cor}{Corollary}
\newtheorem{Con}[Thm]{Conjecture}
\newreptheorem{Con}{Conjecture}
\newtheorem{Ques}[Thm]{Question}
\newtheorem{Obs}[Thm]{Observation}
\newtheorem{Ass}[Thm]{Assumption}
\newtheorem{Note}[Thm]{Note}

\newtheorem{thm-int}{Theorem}

\theoremstyle{definition}
\newtheorem{Def-s}[Thm]{Definition}
\newtheorem{Def}[Thm]{Definition}
\newtheorem{Rem}[Thm]{Remark}
\newtheorem{DefRem}[Thm]{Definition and Remark}
\newtheorem{Prob}[Thm]{Problem}
\newtheorem{Ex}[Thm]{Example}


\def\C{\ensuremath{\mathbb{C}}}
\def\D{\ensuremath{\mathbb{D}}}
\def\G{\ensuremath{\mathbb{G}}}
\def\H{\ensuremath{\mathbb{H}}}
\def\N{\ensuremath{\mathbb{N}}}
\def\P{\ensuremath{\mathbb{P}}}
\def\Q{\ensuremath{\mathbb{Q}}}
\def\R{\ensuremath{\mathbb{R}}}
\def\Z{\ensuremath{\mathbb{Z}}}

\def\AA{\ensuremath{\mathcal A}}
\def\BB{\ensuremath{\mathcal B}}
\def\CC{\ensuremath{\mathcal C}}
\def\DD{\ensuremath{\mathcal D}}
\def\EE{\ensuremath{\mathcal E}}
\def\FF{\ensuremath{\mathcal F}}
\def\GG{\ensuremath{\mathcal G}}
\def\HH{\ensuremath{\mathcal H}}
\def\II{\ensuremath{\mathcal I}}
\def\JJ{\ensuremath{\mathcal J}}
\def\KK{\ensuremath{\mathcal K}}
\def\LL{\ensuremath{\mathcal L}}
\def\MM{\ensuremath{\mathcal M}}
\def\NN{\ensuremath{\mathcal N}}
\def\OO{\ensuremath{\mathcal O}}
\def\PP{\ensuremath{\mathcal P}}
\def\RR{\ensuremath{\mathcal R}}
\def\SS{\ensuremath{\mathcal S}}
\def\QQ{\ensuremath{\mathcal Q}}
\def\TT{\ensuremath{\mathcal T}}
\def\VV{\ensuremath{\mathcal V}}
\def\WW{\ensuremath{\mathcal W}}
\def\XX{\ensuremath{\mathcal X}}
\def\YY{\ensuremath{\mathcal Y}}
\def\ZZ{\ensuremath{\mathcal Z}}

\newcommand{\mor}[1][]{\xrightarrow{#1}}
\newcommand{\isomor}{\mor[\sim]}

\def\Y{\ensuremath{\tilde{Y}}}


\def\OPP{\ensuremath{\widehat{\mathcal P}}}

\def\cht{\ensuremath{\widetilde{\ch}}}

\def\AAA{\mathfrak A}
\def\CCC{\mathfrak C}
\def\DDD{\mathfrak D}
\def\EEE{\mathfrak E}
\def\FFF{\mathfrak F}
\def\GGG{\mathfrak G}
\def\LLL{\mathfrak L}
\def\MMM{\mathfrak M}
\def\NNN{\mathfrak N}
\def\RRR{\mathfrak R}
\def\SSS{\mathfrak S}
\def\VVV{\mathfrak V}


\def\XG{\ensuremath{[X/G]}}
\def\MX{\ensuremath{\mathfrak M_{\sigma',X}(\v)}}
\def\MXs{\ensuremath{\mathfrak M^s_{\sigma',X}(\v)}}
\def\mX{\ensuremath{M_{\sigma',X}(\v)}}
\def\mXs{\ensuremath{M^s_{\sigma',X}(\v)}}

\def\MXG{\ensuremath{\mathfrak M_{\sigma,\XG}(v)}}
\def\MXGs{\ensuremath{\mathfrak M^s_{\sigma,\XG}(v)}}
\def\mXG{\ensuremath{M_{\sigma,\XG}(v)}}
\def\mXGs{\ensuremath{M^s_{\sigma,\XG}(v)}}
\def\mXGss{\ensuremath{M^{ss}_{\sigma,\XG}(v)}}

\def\MS{\ensuremath{\mathfrak M_{\sigma,S}(v)}}
\def\MSs{\ensuremath{\mathfrak M^s_{\sigma,S}(v)}}
\def\mS{\ensuremath{M_{\sigma,S}(v)}}
\def\mSs{\ensuremath{M^s_{\sigma,S}(v)}}
\def\mSss{\ensuremath{M^{ss}_{\sigma,S}(v)}}



\def\MY{\ensuremath{\mathfrak M_{\sigma,\XG}(v)}}
\def\MYs{\ensuremath{\mathfrak M^s_{\sigma,\XG}(v)}}
\def\mY{\ensuremath{M_{\sigma,\XG}(v)}}
\def\mYs{\ensuremath{M^s_{\sigma,\XG}(v)}}
\def\mYss{\ensuremath{M^{ss}_{\sigma,\XG}(v)}}


\def\cal{\mathcal}
\def\Bbb{\mathbb}
\def\frak{\mathfrak}

\def\simpos{\sim_{\R^+}}


%This command creates a box marked ``To Do'' around text.
%To use type \todo{  insert text here  }.

\newcommand{\info}[1]{\vspace{5 mm}\par \noindent
\marginpar{\textsc{Info}}
\framebox{\begin{minipage}[c]{0.95 \textwidth}
\tt #1 \end{minipage}}\vspace{5 mm}\par}

\newcommand{\todo}[1]{\vspace{5 mm}\par \noindent
\marginpar{\textsc{ToDo}}
\framebox{\begin{minipage}[c]{0.95 \textwidth}
\tt #1 \end{minipage}}\vspace{5 mm}\par}

%\renewcommand{\info}[1]{}
%\renewcommand{\todo}[1]{}

\newcommand{\ignore}[1]{}


\begin{document}
\author{Howard Nuer}
\author{K\={o}ta Yoshioka}
\title[MMP via wall-crossing for moduli sheaves on Enriques surfaces]{MMP via wall-crossing for moduli spaces of stables sheaves on an Enriques surface}
\maketitle
\setcounter{tocdepth}{1}
\tableofcontents
\section{Introduction}

\begin{Thm}\label{Thm:MainTheorem1}
Let $\sigma,\tau$ be generic stability conditions with respect to $v$.
\begin{enumerate}
\item The two moduli spaces $M_{\sigma}(v)$ and $M_{\tau}(v)$ of Bridgeland-stable objects are birational to each other.
\item More precisely, there is a birational map induced by a derived (anti-)autoequivalence $\Phi$ of $\Db(X)$ in the following sense: there exists a common open subset $U\subset M_{\sigma}(v),U\subset M_{\tau}(v)$, with complements of codimension at least two, such that for any $u\in U$, the corresponding objects $E_u\in M_{\sigma}(v)$ and $F_u\in M_{\tau}(v)$ satisfy $F_u=\Phi(E_u)$.
\end{enumerate}
\end{Thm}


\begin{Thm}\label{Thm:application1}
Let $v$ be a primitive Mukai vector such that $\rk v$ is odd.
Then for a general $\sigma$,
there is an (anti-)autoequivalence $\Phi$ of ${\bf D}(X)$
which induces an isomorphism $\Phi:U \to U'$
where $U \subset M_\sigma(v,L)$ and 
$U' \subset \Hilb_X^{\frac{v^2+1}{2}}$ are dense open subsets.  
In particular, $M_\sigma(v,L)$ is birationally equivalent to
$\Hilb_X^{\frac{v^2+1}{2}}$, 
$\pi_1(M_\sigma(v,L)) \cong \Z/2 \Z$,
$K_{M_\sigma(v,L)} \not \cong  \OO_{M_\sigma(v,L)}$ and 
$K_{M_\sigma(v,L)}^{\otimes 2} \cong  \OO_{M_\sigma(v,L)}$
for $v^2 \geq 1$. 
Moreover if $X$ is unnodal, then the complements of $U,U'$ are codimension 2.
\end{Thm}

By the proof of \cite{Yos03}, there is an (anti)-autoequivalence
$\Phi$ such that $\rk \Phi(v) =1$.
Applying Theorem \ref{Thm:MainTheorem1}, we get the first claim.
For the second claim, we use \cite[sect. 1]{OS}.







\begin{Thm}\label{Thm:application2}
Assume that $\Pic(X)  \cong \Pic(\widetilde{X})$. 
Let $v$ be a primitive Mukai vector such that $\rk v$ is even
 and $\ell(v)=1$.
Then there is an elliptic fibration $\pi:X \to \P^1$
and a Mukai vector $w=(0,C,\chi)$ 
such that 
(1) $C \to \P^1$ is a double cover of $\P^1$, $\chi \ne 0$ and 
(2) there is an (anti-)autoequivalence $\Phi$ of ${\bf D}(X)$
which induces an isomorphism $\Phi:U \to U'$
where $U \subset M_\sigma(v,L)$ and 
$U' \subset M_H(w,L')$ are open subsets 
whose complements are codimension 2.  
In particular
$M_\sigma(v,L)$ is 
birationally equivalent to
$M_H(w,L')$
for a general $\sigma$.
If $v^2 \geq 2$, then
$\pi_1(M_\sigma(v,L)) \cong \Z/2 \Z$,
$K_{M_\sigma(v,L)} \cong \OO_{M_\sigma(v,L)}$
and $h^{p,0}(M_\sigma(v,L))=0$ for $p \ne 0, v^2+1$.
\end{Thm}

\begin{proof}
By the proof of \cite{Yos16b} and Theorem \ref{Thm:MainTheorem1},
we get the first claim.
By \cite{Yos16b}, \cite[Assumption 2.16]{Sacca} holds
for $M_H(w,L')$. Hence by 
\cite[Thm. 3.1, Thm. 4.4]{Sacca}, we get the second claim.
\end{proof}



We set $K(X)_v:=\{x \in K(X) \mid \langle x,v \rangle=0\}$.
For a universal family $\EE$ on $M_{\sigma}(v,L) \times X$,
we define
\begin{equation}
\begin{matrix}
\theta_{v,\sigma}:& K(X)_v & \to & \Pic(M_{\sigma}(v,L))\\
& x &\mapsto & \det (p_{M_\sigma(v,L)!}(\EE \otimes p_X^*(x^{\vee})).
\end{matrix} 
\end{equation}
Then
\begin{Cor}\label{Cor:Picard}
Let $v$ be a primitive Mukai vector with $v^2 \geq 3$.
\begin{enumerate}
\item[(1)]
Assume that $v^2$ is odd. Then
$\theta_{v,\sigma}$ is an isomorphism for a general $\sigma$.
\item[(2)]
Assume that $\Pic(X)=\Pic(\widetilde{X})$, $\ell(v)=1$ and 
$v^2$ is even.
Then $\theta_{v, \sigma}$ is an isomorphism.
\end{enumerate}
\end{Cor}




\section{Review:stability conditions on Enriques surfaces and moduli spaces}
\begin{Thm}\label{Thm:generic moduli spaces} \todo{Insert Theorem on non-emptiness and dimension of generic moduli spaces}

\end{Thm}

{\color{red}
In this section, we give  a summary of stability for Enriques surfaces in
\cite{Nue14a}, \cite{Yos16b}. 

Let $X$ be an Enriques surface over an algebraically 
closed field $k$ of $\chr(k) \ne 2$.
For a coherent sheaf $E$ on $X$, we define the Mukai vector $v(E)$
of $E$ as
$v(E):=(\rk E,c_1(E),\chi(E)-\frac{\rk E}{2})
\in {\Bbb Z} \times \Num(X) \times {\Bbb Q}$.
For two Mukai vectors $v=(r,L,a)$ and $v'=(r',L',a')$,
$\langle v,v' \rangle:=(L,L')-ra'-r'a \in {\Bbb Z}$ is the Mukai pairing.  
For a stability condition $\sigma=(Z_\sigma,\AA_\sigma)$,
there is $\mho_\sigma \in H^*(X,\C)$
such that $Z_\sigma(E)=\langle \mho_\sigma,v(E) \rangle$.

There is a connected component $\Stab^\dagger(X)$
such that $\mathrm{Re} \mho_\sigma$ and 
$\mathrm{Im} \mho_\sigma$ span a positive definite 2-plane.
For a stability condition $\sigma \in \Stab^\dagger(X)$,
let $\MM_\sigma(v)$ be the moduli stack of $\sigma$-semi stable objects $E$
with $v(E)=v$ and $\MM_\sigma(v)^s$ 
the open substack consisting of $\sigma$-stable
objects. 
For a geneal $\sigma$,
let $M_\sigma(v)$ be the coarse moduli scheme of $S$-equivalence
classes of $\sigma$-semi stable objects $E$ with $v(E)=v$. 
$M_\sigma(v)$ is a projective scheme by \cite[sect. 9]{Nuer} and \cite{Yos16b}.

For $L \in \NS(X)$,
$\MM_\sigma(v,L)$ is the substack of $\MM_\sigma(v)$
consisting of $E$ with $c_1(E)=L$.
We define $\MM_\sigma(v,L)^s$ and $M_\sigma(v,L)$ similary.

%\begin{Thm}[{\cite{Nuer14b},\cite{Yos16b}}]
%For a general $\sigma \in \Stab(X)^*$, 
%there is a projective coarse moduli space
%$M_\sigma(v)$ of $\sigma$-semi-stable objects $E$ with $v(E)=v$.
%\end{Thm}

\begin{Def}
For a stability condition $\sigma$, we set
\begin{equation}
\xi_\sigma:=\mathrm{Im}\frac{ \mho_\sigma}{\langle \mho_\sigma, v \rangle}
\in v^\perp.
\end{equation}
\end{Def}

By the proof of the projectivity of $M_\sigma(v)$ in \cite{Yos16b} with the argument
in \cite{MYY14} or \cite{BM14a}, we get the following claim.
\begin{Thm}
For a general $\sigma$,
$\theta_{v,\sigma}(\xi_\sigma)$ is an ample divisor 
on $M_\sigma(v)$.
\end{Thm}



\subsection{Some properties of moduli spaces}

The following results follow from \cite{Nue14a}. 
Since $\MM_\sigma(v)$ is isomorphic to a moduli stack of Gieseker semi-stable
sheaves \cite{Yos16b}, they also follow from corresponding results for
Gieseker semi-stable sheaves \cite{Yos16a}.

\begin{Thm}[{cf. \cite[Thm. 6.12]{Nue14a}}]\label{Thm:exist:nodal}
Let $X$ be an Enriques surface over $k$.
We take $r,s \in {\Bbb Z}$ and $L \in \NS(X)$ such
that $r+s$ is even.
Assume that 
$\gcd(r,L,\frac{r+s}{2})=1$, i.e., the Mukai vector $v:=(r,L,\frac{s}{2})$
is primitive.
Then
$\MM_\sigma(v,L) \ne \emptyset$ for a general $\sigma$
if and only if
\begin{enumerate}
\item
$\gcd(r,L,s)=1$ and $(L^2)-rs \geq -1$ or 
\item 
$\gcd(r,L,s)=2$ and $(L^2)-rs \geq 2$ 
or 
\item
$\gcd(r,L,s)=2$,
$(L^2)-rs =0$ and $L \equiv \frac{r}{2}K_X \mod 2$ or
\item
$(L^2)-rs =-2$,
$L \equiv D+\frac{r}{2}K_X \mod 2$, where 
$D$ is a nodal cycle, that is, $(D^2)=-2$ and $H^1({\cal O}_D)=0$.
\end{enumerate}
\end{Thm}




\begin{Lem}\label{lem:pss}
Let $v$ be a Mukai vector with $\langle v^2 \rangle>0$.
%(we don't assume the primitivity of $v$).
We set
\begin{equation}
\MM_\sigma(v)^{pss}:=\{E \in \MM_\sigma(v) \mid
\text{$E$ is properly $\sigma$-semi-stable }\}.
\end{equation}
Assume that $\sigma$ is general with respect to $v$.
Then
\begin{enumerate}
\item[(1)]
$\dim \MM_\sigma(v)^{pss} \leq \langle v^2 \rangle-1$.
Moreover $\dim \MM_\sigma(v)^{pss} \leq \langle v^2 \rangle-2$ unless
$v=2v_0$ with $\langle v_0^2 \rangle=1$.
 \item[(2)]
$\MM_\sigma(v)^{s} \ne \emptyset$ and 
$\dim \MM_\sigma(v)=\langle v^2 \rangle$.
\end{enumerate}
\end{Lem}



\begin{Prop}\label{Prop:normal/reduced}
Let $v=(r,\xi,\frac{s}{2})$ 
be a Mukai vector with $\langle v^2 \rangle>0$. Then
for a general $\sigma$, we have the following.
\begin{enumerate}
\item[(1)]
$\MM_\sigma(v,L)$ is reduced and  
$\dim \MM_\sigma(v,L)=\langle v^2 \rangle$.
\item[(2)]
$\MM_\sigma(v,L)$ is normal, unless
\begin{enumerate}
\item[(i)] $v=2v_0$ with $\langle v_0^2 \rangle=1$ and 
$L \equiv \frac{r}{2}K_X \mod 2$ or
\item[(ii)] $\langle v^2 \rangle=2$.
\end{enumerate}
\end{enumerate}
\end{Prop}



\begin{Prop}\label{prop:isotropic}
Let $v$ be an isotropic and primitive Mukai vector.
\begin{enumerate}
\item[(1)]
If $\MM_\sigma(l v)^s \ne \emptyset$, then
$l=1,2$.
\item[(2)]
$\MM_\sigma(2 v,L)^s \ne \emptyset $
if and only if $\ell(v)=1$ and $L \equiv 0 \mod 2$.
Moreover
$$
\MM_\sigma(2 v)^s=\{\varpi_*(F) \mid F \in 
\MM_{\varpi^*(\sigma)}(w)^s,\;
\iota^*(F) \not \cong F \},
$$ 
where $\varpi:\widetilde{X} \to X$ is the covering K3 surface of $X$,
$\iota \in \Aut(\widetilde{X})$ the covering involution and
$w=\varpi^*(v)$.
In particular, $\MM_\sigma(2v)^s$ is smooth of
dimension 1.
\item[(3)]
$\dim \MM_\sigma(lv) \leq l$.
If $\ell(v)=1$, then
$\dim \MM_\sigma(lv) \leq [\frac{l}{2}]$.
%In particular,
%$\dim {\cal M}(0,l f/2,ld)^{ss} \leq [\frac{l}{2}]$, where
%$(0,f/2,d)$ is primitive.
\end{enumerate} 
\end{Prop}





\begin{Thm}\label{Thm:intro:irred}
Let $v=(r,\xi,\frac{s}{2})$ 
be a primitive Mukai vector on an Enriques surface $X$,
$L$ be a divisor on $X$ with $[L \mod K_X]=\xi$, and 
$\sigma$ a general stability condition with respect to $v$. Then
\begin{enumerate}
\item[(1)]
$\MM_\sigma(v,L)$ is connected.
\item[(2)]
If  $X$ is unnodal or $\langle v^2 \rangle \geq 4$, 
then $\MM_\sigma(v,L)$ is irreducible.
\end{enumerate}
\end{Thm}
}





\section{The hyperbolic lattice associated to a wall}
\begin{Prop}\label{Prop:lattice classification}Let $\HH$ be the hyperbolic lattice associated to a wall $\WW$ and $\sigma_0$ a generic stability condition on the wall.  Then $\HH$ and $\sigma_0$ satisfy one of the following mutually exclusive conditions:
\begin{enumerate}
\item\label{enum:nonegativeclasses} $\HH$ contains no spherical or exceptional classes.  
\item\label{enum:OneNegative} \begin{enumerate}
	\item\label{enum:OneSpherical} $\HH$ contains precisely one spherical class, up to sign, and there exists a unique $\sigma_0$-stable spherical object $S$ with $v(S)\in\HH$.
    \item\label{enum:OneExceptional} $\HH$ contains precisely one exceptional class, up to sign, and there exists exactly two $\sigma_0$-stable exceptional objects $E,E(K_X)$ with $v(E)=v(E(K_X))\in\HH$.
    \end{enumerate}
\item\label{enum:TwoNegative} There are infinitely many spherical or exceptional classes in $\HH$, and either 
\begin{enumerate}
\item\label{enum:TwoSpherical} there exist exactly two $\sigma_0$-stable spherical objects $S,T$ whose classes are in $\HH$; or
\item\label{enum:TwoExceptional} there exist exactly four $\sigma_0$-stable exceptional objects $E_1,E_1(K_X),E_2,E_2(K_X)$ with $v(E_1)=v(E_1(K_X)),v(E_2)=v(E_2(K_X))\in\HH$; or
\item\label{enum:OneExceptionalOneSpherical} there exists exactly one $\sigma_0$-stable spherical object $S$ and exactly two $\sigma_0$-stable exceptional objects $E,E(K_X)$ with $v(S),v(E)=v(E(K_X))\in\HH$.
\end{enumerate}
In case (c), $\HH$ is non-isotropic.
\end{enumerate}
\end{Prop}
\begin{proof}
Suppose that $\HH$ contains precisely one spherical (resp. exceptional) class $s$ (resp. $e$).  Then by Theorem \ref{Thm:generic moduli spaces}, there exists a unique $\sigma_+$-stable object $S$ with $v(S)=s$ (resp. precisely two $\sigma_+$-stable objects $E$ and $E(K_X)$ with $v(E)=v(E(K_X))=e$), which must then be spherical (resp. exceptional) by \cite[Lemma 4.3]{Yos16b}.  Suppose that $S$ (resp. $E$) is strictly $\sigma_0$-semistable.  Then by \cite[Lemma 4.6]{Yos16b} every $\sigma_0$-stable factor $F$ of $S$ (resp. $E\oplus E(K_X)$) must satisfy $\Ext^1(F,F)=0$.  But then $v(F)^2<0$, so by \cite[Lemma 4.3]{Yos16b} $v(F)^2=-1$ or $-2$, i.e. $v(F)$ is either spherical or exceptional.  But this is a contradiction to the assumption, so $S$ (resp. $E,E(K_X)$) is $\sigma_0$-stable, giving Case (b).

It remains to consider Case (c).  It will suffice for our purposes to show that, up to twisting by $K_X$, there cannot be any combination of three stable spherical or exceptional objects $S_1,S_2,S_3$ in $\PP_0(1)$.  Since each $S_i$ is $\sigma_0$-stable of the same phase and distinct up to twisting by $K_X$, we must have $\Hom(S_i,S_j)=\Hom(S_j,S_i(K_X))=0$ for each $i\neq j$.  Thus if $s_i=v(S_i)$, then $\langle s_i,s_j\rangle=\ext^1(S_i,S_j)\geq 0$.

Now any two of the $S_i$ must be linearly independent, and we may choose, say, $s_1$ and $s_2$ to represent either both spherical or both exceptional $\sigma_0$-stable objects.    Denote by $m:=\langle s_1,s_2\rangle\geq 0$.  Since $\HH$ has signature $(1,-1)$, $$\langle s_1,s_2\rangle^2> s_1^2s_2^2=\begin{cases}
1, & \text{ if }s_1^2=s_2^2=-1,\\

4, & \text{ if }s_1^2=s_2^2=-2.\\
\end{cases}$$
So $m\geq 2$ or $3$.  We write $s_3=xs_1+ys_2$, and from $\langle s_3,s_1\rangle,\langle s_3,s_2\rangle\geq 0$, we get that \begin{equation}\label{eq:positivity}\begin{cases}
\frac{1}{m}x\leq y\leq mx, & \text{ if }s_1^2=s_2^2=-1,\\

\frac{2}{m}x\leq y\leq \frac{m}{2}x, & \text{ if }s_1^2=s_2^2=-2.\\  
\end{cases}\end{equation}  But solving the quadratic equation $$0=(xs_1+ys_2)^2$$ gives \begin{equation}\label{eq:isotropic solutions}\frac{y}{x}=\begin{cases}
m\pm\sqrt{m^2-1}, & \text{ if }s_1^2=s_2^2=-1,\\

\frac{m\pm\sqrt{m^2-4}}{2}, & \text{ if }s_1^2=s_2^2=-2,
\end{cases}\end{equation} from which we see two things.  First, as $m\geq 2$ (resp. $m\geq 3$), the solution $\frac{y}{x}$ in \eqref{eq:isotropic solutions} is irrational so that in subcases (c)(i) and (c)(ii) there can be no isotropic classes, as these would give rational solutions in \eqref{eq:isotropic solutions}.  Second, as $$m-\sqrt{m^2-1}\leq\frac{1}{m}\leq m\leq m+\sqrt{m^2-1}$$ for $m\geq 2$ and $$\frac{m-\sqrt{m^2-4}}{2}\leq\frac{2}{m}\leq\frac{m}{2}\leq\frac{m+\sqrt{m^2-4}}{2}$$ for $m\geq 3$, we see that for $s_3$ satisfying \eqref{eq:positivity} we must have $s_3^2>0$, in contradiction to the fact that $s_3^2=-1$ or $-2$.  

Thus we see that there can only be at most two $\sigma_0$-stable spherical or exceptional objects with Mukai classes in $\HH$.  Notice further that if $\HH$ admits any combination of two linearly independent spherical or exceptional classes, then the group generated by the associated spherical and $(-1)$ reflections is infinite, so the orbit of a spherical or exceptional class gives infinitely many of the same kind. 

Finally, it only remains to show that in subcase (c)(iii) $\HH$ is non-isotropic.  Writing an integral isotropic class $xs+ye$ with $x,y\in\Q$, where $s=v(S)$ and $e=v(E)=v(E(K_X))$, the slope would satisfy $$\frac{y}{x}=\frac{m\pm\sqrt{m^2-2}}{2},$$ where $m=\langle s,e\rangle$.  But $S$ and $E$ are $\sigma_0$-stable objects of the same phase with classes in a lattice of signature $(1,-1)$, so we must have $m\geq 2$, as in the arguments for the preceeding cases.  But this gives a contradiction as $m^2-2$ cannot be a square for $m\geq 2$.
\end{proof}

\begin{Thm}\label{classification of walls}
Let $\HH\subset \Hal(X,\Z)$ be a primitive hperbolic rank two sublattice containing $v$, and let $\WW\subset\Stabd(X)$ be a potential wall associated to $\HH$.  

The set $\WW$ is a totally semistable wall if and only if (TSS1) there exists an effective spherical or exceptional class $w\in\HH$ such that $\langle v,w\rangle<0$, (TSS2) there exists an isotropic class $u\in\HH$ with $\ell(u)=2$ such that $\langle v,u\rangle=1$, (TSS3) there exists an isotropic class $u\in\HH$ such that $\langle v,u\rangle=\ell(u)$ and $\langle v,w\rangle=0$ for $w\in\HH$ with 
\begin{equation}
w^2=\begin{cases}
-1, &\mbox{ if }\ell(u)=2 \\
-2, &\mbox{ if }\ell(u)=1
\end{cases}.
\end{equation}
In addition,
\begin{enumerate}
\item\label{thm:Classification,Divisorial} The set $\WW$ is a wall inducing a divisorial contraction if one of the following conditions hold:
\begin{description*}
\item[(Brill-Noether)] there exists a spherical class $w\in\HH$ such that $\langle w,v\rangle=0$, or
\item[(Hilbert-Chow)] there exists an isotropic class $u$ with $\langle v,u\rangle=1$ and $\ell(u)=2$, or
\item[(Li-Gieseker-Uhlenbeck)] there exists a primitive isotropic class $u\in\HH$ with $\langle v,u\rangle=2=\ell(u)$, or
\item[(Name?)] there exists an isotropic class $u\in\HH$ with $\langle v,u\rangle=1=\ell(u)$ and $v^2\geq 3$.
\end{description*} 
\item\label{thm:Classification,Fibration} The set $\WW$ is a wall inducing a $\P^1$-fibration on $M_{\sigma_+}(v,L)$ if one of the following conditions hold:
\begin{description*}
\item[(Exceptional case)] there exists a primitive isotropic class $u$ with $\langle v,u\rangle=2=\ell(u)$, an exceptional class $w$ with $\langle v,w\rangle=0$, and $L\equiv K_X\pmod 2$, or
\item[(Spherical case)] there exists an isotropic class $u$ with $\langle v,u\rangle=1=\ell(u)$, a spherical class $w$ with $\langle v,w\rangle=0$, and $L\equiv D+\frac{\rk v}{2}K_X\pmod 2$, where $D$ is a nodal cycle.
\end{description*}
\item\label{thm:Classification,Flops}
Otherwise, if $v$ is primitive and either
\begin{enumerate}
\item \label{enum:sum2positive}
$v^2\geq 3$ and $v$ can be written as the sum 
$v = a_1 + a_2$ with $a_i\in P_\HH$, or 
\item\label{enum:exceptional} there exists an exceptional class $w$ and either
\begin{enumerate}
\item\label{enum:exceptionalflop1}
$0< \langle  w,v\rangle\leq\frac{v^2}{2}$, or
\item\label{enum:exceptionalflop2}
$\langle v,w\rangle=0$ and $v^2\geq 3$; or
\end{enumerate}
\item\label{enum:spherical} there exists a spherical class $w$ and either
\begin{enumerate}
\item\label{enum:sphericalflop1}
$0 < \langle w, v\rangle < \frac{v^2}2$, or
\item\label{enum:sphericalflop2}
$\langle w,v\rangle=\frac{v^2}{2}$ and $\HH_{\WW}$ falls into subcase \ref{enum:TwoSpherical} of case \ref{enum:TwoNegative} in Proposition \ref{Prop:lattice classification},
\end{enumerate}
\end{enumerate}
then $\WW$ induces a small contraction.
\item In all other cases, $\WW$ is either a fake wall (if it is totally semistable or induces a non-contracted divisor of strictly $\sigma_0$-semistable objects) or not a wall at all.
\end{enumerate}


\end{Thm}

\section{Dimension estimates substacks of Harder-Narasimhan filtrations}
 
Let $\MM_{\sigma}(v)$ be the moduli stack of 
$\sigma$-semi-stable objects with the Mukai vector $v$:
for a scheme $T$,
$\MM_{\sigma}(v)(T)$ is the category of $E \in {\bf D}(X \times T)$ such that $E$ is relatively perfect over $T$ (\cite[Defn. 2.1.1]{L}) and 
$E_t$ are $\sigma$-semi-stable objects
with $v(E_t)=v$ for all $t \in T$. 
By (the proof of) \cite[Thm. 4.12]{Tod08}, 
it is an Artin stack of finite type.


For Mukai vectors $v_1,v_2,...,v_s$ with the same phase $\phi$ with respect to $\sigma$, 
let $\FF(v_1,v_2,...,v_s)$ be the stack of filtrations:
for a scheme $T$, 
$\FF(v_1,v_2,...,v_s)(T)$ is the category of filtrations
\begin{equation}
0 \subset F_1 \subset F_2 \subset \cdots \subset F_s
\end{equation} 
such that $F_i/F_{i-1} \in {\bf D}(X \times T)$ are relatively perfect over $T$, 
$(F_i/F_{i-1})_t \in \MM_{\sigma}(v_i)$ for all $t \in t$ 
and a family of objects $F_s$ such that
$F_s \in {\bf D}(X \times T)$ is relatively perfect over $T$ and
$(F_s)_t \in \MM_{\sigma}(v)$.

\begin{Prop}
$\FF(v_1,v_2,...,v_s)$ is an Artin stack of finite type.
\end{Prop}
\begin{proof}
Assuming that ${\cal F}(v_1,v_2,...,v_{s-1})$ and
${\cal M}_{\sigma}(v_s)$ are Artin stack,
we shall prove that 
${\cal F}(v_1,v_2,...,v_{s-1},v_s)$ is also an Artin stack.
We set $v:=\sum_{i=1}^s v_i$.
It is sufficient to show that the natural morphism
\begin{equation}
\FF(v_1,v_2,...,v_{s-1},v_s) \to 
\FF(v_1,v_2,...,v_{s-1}) \times \MM_{\sigma}(v)
\end{equation}
is representable by a scheme of finite type.

Let $T$ be a scheme and 
$T \to \FF(v_1,v_2,...,v_{s-1}) \times \MM_{\sigma}(v)$
a morphism. Then
we have a family of filtrations
\begin{equation}
0 \subset F_1 \subset F_2 \subset \cdots \subset F_{s-1}
\end{equation} 
on $X \times T$
such that $F_i/F_{i-1}$ are relatively perfect over $T$, 
$(F_i/F_{i-1})_t \in \MM_{\sigma}(v_i)$ for all $t \in T$ 
and a family of objects $F_s$ such that
$F_s$ is relatively perfect over $T$ and
$(F_s)_t \in \MM_{\sigma}(v)$.
By \cite[Prop. 1.1]{Ina02},
there is a scheme $p:Q \to T$ 
which represents the functor $\QQ:(Sch/T) \to (Sets)$ such that 
\begin{equation}
\QQ(U \overset{\varphi}{\to} T)=
\{f \mid f:(\varphi \times 1_X)^*(F_{s-1}) \to (\varphi \times 1_X)^*(F_s)\}.
\end{equation}
Let 
$\xi:(p \times 1_X)^*(F_{s-1}) \to (p \times 1_X)^*(F_s)$
be the universal family of homomorphisms.
Let $Q^0$ be the open subscheme
of $Q$ such that $\xi_q$ is injective in $\AA_{\sigma}$
for $q \in Q^0$.
Thus $\mathrm{cone}(\xi_q) \in \AA_{\sigma}$ for all $q \in Q^0$. 
Then we have a family of filtrations
\begin{equation}
0 \subset F_1 \subset F_2 \subset \cdots \subset F_s.
\end{equation} 
Therefore
\begin{equation}
Q^0 \cong \FF(v_1,v_2,...,v_{s-1},v_s) 
\times_{\FF(v_1,v_2,...,v_{s-1}) \times \MM_{\sigma}(v)}
T
\end{equation}
In particular, $\FF(v_1,v_2,...,v_{s-1},v_s)$ is an Artin stack.


More precisely, if
$T \to \FF(v_1,v_2,...,v_{s-1}) \times \MM_{\sigma}(v)$ is smooth,
then $Q^0 \to \FF(v_1,v_2,...,v_{s-1},v_s)$ is smooth,
that is,
$Q^0 \times_{\FF(v_1,v_2,...,v_s)} U \to U$ is smooth for any
$U \to \FF(v_1,v_2,...,v_s)$. 

For a scheme $T$ and two families
\begin{equation}
\begin{split}
F: & 0 \subset F_1 \subset F_2 \subset \cdots \subset F_s\\
F': & 0 \subset F_1' \subset F_2' \subset \cdots \subset F_s'
\end{split}
\end{equation} 
of relatively perfect filtrations,
an isomorpjism $\phi:F \to F'$ is
an isomorphism $F_s \to F_s'$ which preserves the filtration. 
%\begin{equation}
%\Iso_T(F,F') \to \Iso_T(
%\end{equation}
Hence it is parametrized by a closed subscheme of
$\Isom_T(F_s,F_s')$, which shows that the diagonal morphism is representable
by a scheme.
\end{proof}



We have a morphism
$\FF(v_1,v_2,...,v_s) \to \FF(v_1,v_2,...,v_{s-1}) \times \MM_{\sigma}(v_s)$
and hence a morphism
$$
\FF(v_1,v_2,...,v_s) \to \prod_{i=1}^s \MM_{\sigma}(v_i).
$$
Let $\FF(v_1,v_2,...,v_s)^*$ be the open substack of
$\FF(v_1,v_2,...,v_s)$ which is the pull-back of
$\prod_{i=1}^s \MM_{\sigma_-}(v_i)$, where $\sigma_-$ is sufficiently close to
$\sigma$.



Assume that $v_1,v_2,...,v_s$ be the Mukai vectors of a Harder-Narasimhan
filtration of an object with respect to a stability condition $\sigma_-$.
$\FF(v_1,v_2,...,v_s)^* \to \MM_{\sigma}(v)$ is injective and the image
is the substack of $\MM_\sigma(v)$ parameterizing objects
with the Harder-Narasimhan filtrations associated to the Mukai vectors
$v_1,v_2,...,v_s$.
In the same way as in \cite[Prop. 6.2]{Bri12},
we have
\begin{equation}
\begin{split}
\dim \FF(v_1,v_2,...,v_s)^*=&
\dim \FF(v_1,v_2,...,v_{s-1})^*+\dim \MM_{\sigma_-}(v_s)+
\langle v-v_s,v_s \rangle\\
=& \sum_{i=1}^s \dim \MM_{\sigma_-}(v_i)+\sum_{i<j}\langle v_i,v_j \rangle.
\end{split}
\end{equation}


\begin{proof}
For an atlas $\varphi:T \to \FF(v_1,v_2,...,v_{s-1})^* \times \MM_{\sigma_-}(v_s)$,
we set 
$R:=T \times_{\FF(v_1,v_2,...,v_{s-1})^* \times \MM_{\sigma_-}(v_s)} T$.
Let $0 \subset F_1 \subset \cdots \subset F_{s-1}$ and
$E_s$ be objects on $X \times T$ 
corresponding to the morphism $\varphi$. 
We note that $\Hom((E_s)_t,(F_{s-1})_t [k])=0$ for
$k \ne 0,1$.
We shall stratify $T=\cup_i T_i$ by the dimension of  
$n(t):=\Hom((E_s)_t,(F_{s-1})_t)$, thus, 
$n(t)$ is constant on $T_i$
and $n_{|T_i} \ne n_{|T_j}$ for $i \ne j$. 
We set $R_{ij}:=R \times_{T \times T} T_i \times T_j$.
Then $R_{ij}=\emptyset $ if $i \ne j$ and we have a stratification
$R=\cup_i R_{ii}$.
Let $V_i^k \to T_i$ $(k=0,1)$ be vector bundles associated to 
$\Hom_{p_i}((E_s)_{|T_i},(F_{s-1})_{|T_i}[k])$, 
where $p_i:T_i \times X \to T_i$ is the projection.
Then 
$$
T_i \times_{ \FF(v_1,v_2,...,v_{s-1})^* \times \MM_{\sigma_-}(v_s)}
\FF(v_1,v_2,\cdots,v_s)^*=V_i^1
$$
and 
$$
V_i^1 \times_{\FF(v_1,v_2,...,v_s)^*} V_i^1 \cong 
V_i^0 \times_{R_{ii}} V_i^1
$$
with a commutative diagram
\begin{equation}
\begin{CD}
V_i^0 \times_{R_{ii}} V_i^1 @>>> V_i^1 \times V_i^1 \\
@VVV @VVV\\
R_{ii} @>>> T_i \times T_i.
\end{CD}
\end{equation}
%
Since
\begin{equation}
\begin{split}
& \dim V_i^1 \times V_i^1-\dim V_i^0 \times_{R_{ii}} V_i^1\\
=& \dim T_i \times T_i +2\rk V_i^1-(\dim R_{ii}+\rk V_i^1+\rk V_i^0)\\
=& (\dim T_i \times T_i-\dim R_{ii})+\langle v-v_s, v_s \rangle,
\end{split}
\end{equation}
we get 
\begin{equation}
\begin{split}
\dim \FF(v_1,v_2,...,v_s)^* =& 
\max_i \{\dim V_i^1 \times V_i^1-\dim V_i^0 \times_{R_{ii}} V_i^1\}\\
=& \max_i \{(\dim T_i \times T_i-\dim R_{ii})\}+\langle v-v_s, v_s \rangle\\
=&
\dim \FF(v_1,v_2,...,v_{s-1})^*+
\dim \MM(v_s)+\langle v-v_s, v_s \rangle.
\end{split}
\end{equation}
\end{proof}
  

\begin{Prop}\label{Prop:HN codim}
$$\codim\FF(v_1,...,v_n)^*
=\sum_{i=1}^n (v_i^2-\dim\MM_{\sigma_-}(v_i))+\sum_{i<j}\langle v_i,v_j\rangle.$$

\end{Prop}

Before we enter into a lattice specific analysis of the wall-crossing behavior, we present a general result on the codimension of the strictly $\sigma_0$-semistable locus corresponding to the simplest Harder-Narasimhan filtration as above:
\begin{Prop}\label{Prop:HN filtration all positive classes}
Let $\FF(a_1,...,a_n)^0$ be the substack of $\MM_{\sigma_+}(v)$ parametrizing objects with $\sigma_-$ Harder-Narasimhan filtration factors of classes $a_1,...,a_n$ (in order of descending phase with respect to $\phi_{\sigma_-}$), and suppose that $a_i^2>0$ for all $i$.  Then $\codim\FF(a_1,...,a_n)^0\geq 2$.  
\end{Prop}
\begin{proof}
By Theorem \ref{Thm:generic moduli spaces}, the assumption that $a_i^2>0$ implies that $\dim\MM_{\sigma_-}(a_i)=a_i^2$.  Thus by Proposition \ref{Prop:HN codim}, $$\codim\FF(a_1,...,a_n)^0=\sum_{i<j}\langle a_i,a_j\rangle.$$  But as $a_i^2\geq 1$ and $\HH$ has signature $(1,-1)$, we must have $$\langle a_i,a_j\rangle>\sqrt{a_i^2 a_j^2}\geq 1,$$ for $i<j$.  Thus $\langle a_i,a_j\rangle\geq 2$.  It follows that $$\codim\FF(a_1,...,a_n)^0\geq n(n-1)\geq 2,$$ as $n\geq 2$.
\end{proof}

\section{Totally semistable non-isotropic walls}
We begin this section with a sufficient condition for a potential wall $\WW$ to be totally semistable.  

\begin{Lem}\label{Lem: condition for totally semistable wall}
Let $\WW$ be a potential wall such that $\langle v,w\rangle<0$ for an effective spherical or exceptional class $w\in\HH_{\WW}$.  Then $\WW$ is totally semistable.
\end{Lem}

\begin{proof}
Suppose there were a $\sigma_0$-stable object $E$ of class $v$.  Let $\tilde{E_0}$ be a $\sigma_0$-semistable object $\tilde{E_0}$ with $v(\tilde{E_0})=w$.  As all stable factors of $\tilde{E_0}$ are spherical or exceptional \cite[Lemma 4.3, Lemma 4.6]{Yos16b}, we may find a $\sigma_0$-stable object $E_0$ such that $\langle v,v(E_0)\rangle<0$ and $v(E_0)^2=-1$ or $-2$.  As $E$ and $E_0$ (resp. $E$ and $E_0(K_X)$) are non-isomorphic $\sigma_0$-stable objects of the same phase, we must have $\Hom(E,E_0)=\Hom(E_0(K_X),E)=0$.  But then $0>\langle v,v(E_0)\rangle=\ext^1(E,E_0)\geq 0$, a contradiction.
\end{proof}

The condition in Lemma \ref{Lem: condition for totally semistable wall} will turn out to be necessary in the non-isotropic case, see Lemma \ref{Lem:non-isotropic no totally semistable wall}.  In the remainder of this section we use the theory of Pell's equation to reduce to the case of Mukai vectors pairing non-negatively with every effective spherical or exceptional class.  Such a Mukai vector is called \emph{minimal in its $G_{\HH}$-orbit}, or simply \emph{minimal} for short, because of the following definition.

\begin{PropDef}\label{PropDef: minimal vectors}
Let $G_{\HH}\subset\Aut(\HH)$ be the group generated by spherical and exceptional reflections associated to effective spherical and exceptional classes in $C_{\WW}$.  For a given positive class $v\in P_{\HH}\cap\HH$, the $G_{\HH}$-orbit of $v$ contains a unique class $v_0$ such that $\langle v,w\rangle\geq 0$ for all effective spherical and exceptional classes $w\in C_{\WW}$.  We call $v_0$ the minimal class of the orbit $G_{\HH}\cdot v$.
\end{PropDef}

The proof of the existence of $v_0$ is almost identical to that of \cite[Proposition and Definition 6.6]{BM14b}, so we omit it.  In the remainder of the section we consider explicitly only Case (cii) and (ciii) as Cases (a) and (b) are irrelevant or trivial, respectively, and Case (ci) is covered in detail in \cite[Section 6]{BM14b}.

\subsection{Case (cii): Exactly two $\sigma_0$-stable exceptional objects up to $-\otimes\OO(K_X)$}

Assume that $\HH$ contains at least two $(-1)$-vectors.
We set $\mho:=\pi(\sigma_0)$. Since
$\mathrm{Re}\mho$ and $\mathrm{Im}\mho$ span
a positive definite 2-plane,
$\mho^\perp \cap K(X)_\R$ is negative definite.
In particular $Z^{-1}(0) \cap \HH_\R$ is negative definite.
Let $L^+$ be the half plane of $\HH_\R$ such  that
$v \in L^+$ if and only if $Z(v) \in \R_{\leq 0}$.  
%Let $P^+$ be the positive cone, that is, $P^+$ is the connected component
%of $\{ v \in \R \mid v^2>0 \}$ contained in $L^+$.
We set
\begin{equation}
\Delta:=\{u \in \HH \mid u^2=-1 \}.
\end{equation}
Let $w_1$ be a $(-1)$-vector. Then $\HH=\Z w_1+\Z w_2$
where $\langle w_1,w_2 \rangle=0$ and $D:=w_2^2>0$.
Then $\Delta$ is described by 
the Pell equation
\begin{equation}\label{eq:Pell}
x^2-Dy^2=1.
\end{equation}
Assume that $D$ is irrational. 
Let $(p_1,q_1)$ be the fundamental solution of 
\eqref{eq:Pell} with $p_1<0$ and $q_1>0$.
We define $p_n, q_n \in\Z$ by
\begin{equation}
p_n+q_n \sqrt{D}=
\begin{cases}
-(-p_1-q_1 \sqrt{D})^n, & n > 0\\
(-p_1-q_1 \sqrt{D})^n, & n \leq 0
\end{cases}
\end{equation}
We set $u_n:=p_n w_1+q_n w_2$.
Then 
$$
\Delta=\{\pm u_n \mid n \in \Z\}.
$$ 
It is easy to see that
\begin{equation}
\begin{split}
u_{n+1}=&-R_{u_n}(u_{n-1}),\; (n \geq 2)\\
u_{n-1}=&-R_{u_n}(u_{n+1}),\; (n \leq -1)\\
u_2=&R_{u_1}(u_0),\;\; u_{-1}=R_{u_0}(u_1).
\end{split}
\end{equation}


Since $Z^{-1}(0) \cap \HH_\R$ is negative definite
and 
$\lim_{n \to \pm \infty}\frac{p_n}{q_n}=\mp \sqrt{D}$,
there are $w=xw_1+y w_2, w'=x' w_1+y' w_2 \in \Delta$ such that
$x,x'>0$, $Z^{-1}(0) \cap (\R_{>0}w+\R_{>0}w') \ne \emptyset$
and $\Delta \cap (\R_{>0}w+\R_{>0}w')= \emptyset$.
Then $w,-w' \in L^+$ or $-w,w' \in L^+$, so we may assume that
$w,-w' \in L^+$.
Then we have 
$$
\{ v \in \HH \mid v \in L^+, v^2 \geq -1\} \subset 
\Q_{\geq 0}w+\Q_{\geq 0}(-w').
$$
Replacing $w_1$ by $w$, we assume that $w_1=w$.
Replacing $w_2$ by $-w_2$ if necessary,
we also assume that 
$$
P_{\HH}=\{xw_1+yw_2 \in \HH \mid y^2 D-x^2>0, y \geq 0 \}.
$$
Then the effective cone is $\Q_{\geq 0}u_0+\Q_{\geq 0}u_1$.
Thus 
$p_n w_1+q_n w_2 \in L^+$ for $n \geq 1$ and
$-(p_n w_1+q_n w_2) \in L^+$ for $n \leq 0$.
Let $T_1$ and $T_0$ be $\sigma_0$-semi-stable objects
with 
$$
v(T_1)=p_1 w_1+q_1 w_2,\; 
v(T_0)=-(p_0 w_1+q_0 w_2)=w_1.
$$ 
Then they are $\sigma_0$-stable.

%
%
We set
\begin{equation}
\CC_n:=\left\{x w_1+y w_2 \left|
 \frac{q_{n+1}}{p_{n+1}}y<x<\frac{q_n}{p_n}y, y>0 \right. \right\}.
\end{equation}
Then $\{\CC_n \mid n \in \Z \}$ is the chamber decomposition
of $P_{\HH}$ by $(-1)$-vectors and 
$$
\CC_0=\{ v \in P_{\HH} \mid \langle v, u_n \rangle>0, n \in \Z \}.
$$
For $v_0 \in \CC_0$, 
we set 
\begin{equation}
v_n:= 
\begin{cases}
R_{u_n} \circ R_{u_{n-1}} \circ \cdots \circ R_{u_1}(v_0), & n>0\\
R_{u_{n+1}}^{-1} \circ R_{u_{n+2}}^{-1} 
\circ \cdots \circ R_{u_0}^{-1}(v_0), & n \leq 0.
\end{cases}
\end{equation}
Then for $v \in \CC_n$, there is $v_0 \in \CC_0$ such that
$v_n=v$.


\subsection{Abelian categories $\AA_i$}


We assume that $\phi^+(T_1)>\phi^+(T_0)$ (and hence $\phi^-(T_1)<\phi^-(T_0)$).
Let $T_i^\pm$ be $\sigma^\pm$-stable objects with
$v(T_i^\pm)=u_i$.
Then 
\begin{equation}
\phi^+(T_1^+) > \phi^+(T_2^+)>\cdots>\phi^+(E)>
\cdots >\phi^+(T_{-1})>\phi^+(T_0^+)
\end{equation}
for any $\sigma_+$-stable object $E$ with
$v(E)^2 \geq 0$.
We note that $T_i^+=T_i^-=T_i$ $(i=0,1)$ are $\sigma_0$-stable objects.


\begin{Def}
Let $R_{T_i^\pm}:{\bf D}(X) \to {\bf D}(X)$ be an equivalence such that
\begin{equation}
R_{T_i^\pm}(E):=\mathrm{cone}({\bf R}\Hom(T_i^\pm,E)\otimes T_i^\pm  \oplus 
{\bf R}\Hom(T_i^\pm (K_X),E) \otimes T_i^\pm (K_X) \to E).
\end{equation}
\end{Def}


\begin{Def}
Assume that $i  \geq 0$.
\begin{enumerate}
\item[(1)]
Let $(\TT_i,\FF_i)$ be a torsion pair of $\PP(1)$ such that
\begin{enumerate}
\item
$\TT_i=\langle T_1^+,T_1^+(K_X),T_2^+,T_2^+ (K_X),...,
T_i^+,T_i^+ (K_X) \rangle$ 
is the subcategory of $\PP(1)$ generated by $\sigma_+$-stable objects
$F$ with $\phi^+(F)>\phi^+(T_{i+1}^+)$ and
\item
$\FF_i$ is the subcategory of $\PP(1)$ generated by 
$\sigma_+$-stable objects $F$
with $\phi^+(F) \leq \phi^+(T_{i+1}^+)$.
\end{enumerate}
Let $\AA_i:=\langle \TT_i[-1],\FF_i \rangle$ be the tilting.
\item[(2)]
Let $(\TT_i^*,\FF_i^*)$ be a torsion pair of $\PP(1)$ such that
\begin{enumerate}
\item
$\TT_i^*$ is the subcategory of $\PP(1)$ generated by 
$\sigma_-$-stable objects $F$
with $\phi^-(F) \geq \phi^-(T_{i+1}^-)$.
\item
$\FF_i^*=\langle T_1^-,T_1^-(K_X),T_2^-,T_2^-(K_X),...,
T_i^-,T_i^-(K_X) \rangle$ 
is the subcategory of $\PP(1)$ generated by $\sigma_-$-stable objects
$F$ with $\phi^-(F)<\phi^-(T_{i+1}^-)$.
\end{enumerate}
Let $\AA_i^*:=\langle \TT_i^*,\FF_i^*[1] \rangle$ be the tilting.
\end{enumerate}
\end{Def}
Since
$\TT_0=0$ and $\FF_0^*=0$,
we have $\AA_0=\AA_0^*=\PP(1)$.


\begin{Rem}\label{rem:simple-objects}
$T_{i+1}^+,T_{i+1}^+(K_X), T_i^+[-1],T_i^+(K_X)[-1]$ are irreducible objects of $\AA_i$
and 
$T_{i+1}^-[1],T_{i+1}^-(K_X)[1], T_i^-,T_i^-(K_X)$ are irreducible objects of $\AA_i^*$.
\end{Rem}









\begin{Prop}\label{Prop:equiv1}
$R_{T_{i+1}^+}$ $(i \geq 0)$ induces an equivalence
$\AA_i \to \AA_{i+1}$.
\end{Prop}

\begin{proof}
We first prove that 
$R_{T_{i+1}^+}(\AA_i) \subset \AA_{i+1}$.
We set $\Phi:=R_{T_{i+1}^+}$ and $\Phi^p(E):=H^p(\Phi(E))$ $(E \in {\bf D}(X))$.
For $E \in \PP(1)$, 
$\Ext^p(T_{i+1}^+,E)=\Ext^p(T_{i+1}^+(K_X),E)=0$ for $p \ne 0,1,2$.
Hence we have an exact sequence
\begin{equation}
\begin{CD}
0 @>>> \Phi^{-1}(E) @>>> \Hom(T_{i+1}^+,E) \otimes T_{i+1}^+ \oplus  
\Hom(T_{i+1}^+(K_X),E) \otimes T_{i+1}^+ (K_X) @>{\varphi}>> E \\
 @>>> \Phi^0(E) @>>> \Ext^1(T_{i+1}^+,E) \otimes T_{i+1}^+ \oplus  
\Ext^1(T_{i+1}^+(K_X),E) \otimes T_{i+1}^+ (K_X) @>>> 0
\end{CD}
\end{equation}
and also an isomorphism
\begin{equation}
\Phi^1(E) \cong \Ext^2(T_{i+1}^+,E) \otimes T_{i+1}^+\oplus  
\Ext^2(T_{i+1}^+(K_X),E) \otimes T_{i+1}^+ (K_X) \in \TT_{i+1}.
\end{equation}

Assume that $E \in \FF_i$. Then
$\phi_{\max}^+(E) \leq \phi^+(T_{i+1}^+)$. Hence 
$\varphi$ is injective and $\coker \varphi \in \FF_i$. 
Then we have $\Phi^0(E) \in \FF_i$.
Since 
\begin{equation}
\begin{split}
\Hom(T_{i+1}^+,\Phi(E))& =\Hom(\Phi(T_{i+1}^+(K_X))[1],\Phi(E))
=\Hom(T_{i+1}^+(K_X),E[-1])=0,\\
\Hom(T_{i+1}^+(K_X),\Phi(E))&=\Hom(\Phi(T_{i+1}^+)[1],\Phi(E))
=\Hom(T_{i+1}^+,E[-1])=0,
\end{split}
\end{equation}
we get
$\Phi^0(E) \in \FF_{i+1}$.
Therefore $\Phi(E) \in \AA_i$.

For $E \in \TT_i$, 
$\coker \varphi \in \TT_i \subset \TT_{i+1}$.
Since $T_{i+1}^+, T_{i+1}^+ (K_X)\in \TT_{i+1}$,
$\Phi^0(E) \in \TT_{i+1}$.
By $T_{i+1}^+,T_{i+1}^+(K_X) \in \FF_i$, $\Phi^{-1}(E) \in \FF_i$.
Since 
\begin{equation}
\begin{split}
\Hom(T_{i+1}^+,\Phi(E)[-1])& =\Hom(\Phi(T_{i+1}^+(K_X)),\Phi(E)[-2])
=\Hom(T_{i+1}^+(K_X),E[-2])=0,\\
\Hom(T_{i+1}^+(K_X),\Phi(E)[-1])&=\Hom(\Phi(T_{i+1}^+),\Phi(E)[-2])
=\Hom(T_{i+1}^+,E[-2])=0,
\end{split}
\end{equation}
we get $\Phi^{-1}(E) \in \FF_{i+1}$.
Therefore $\Phi(E[-1]) \in \AA_{i+1}$.


We next prove that $R_{T_{i+1}^+}^{-1}(\AA_{i+1}) \subset \AA_i$.
Let $\Psi$ be the inverse of $\Phi$, and set
$\Psi^p(E):=H^p(\Psi(E))$ $(E \in {\bf D}(X))$.
Then
\begin{equation}\label{eq:Psi-1}
\Hom(T_{i+1}^+,E) \otimes T_{i+1}^+ \oplus
\Hom(T_{i+1}^+(K_X),E) \otimes T_{i+1}^+(K_X) \cong \Psi^{-1}(E)
\end{equation}
and we have an exact sequence
\begin{equation}
\begin{CD}
0 @>>> \Ext^1(T_{i+1}^+,E) \otimes T_{i+1}^+ \oplus
\Ext^1(T_{i+1}^+(K_X),E) \otimes T_{i+1}^+(K_X) @>>> \Psi^0(E) @>>>E \\
 @>{\psi}>> \Ext^2(T_{i+1}^+,E) \otimes T_{i+1}^+ \oplus
\Ext^2(T_{i+1}^+(K_X),E) \otimes T_{i+1}^+(K_X) @>>> \Psi^1(E) @>>>0.
\end{CD}
\end{equation}

Assume that 
$E \in \FF_{i+1}$.
Then $\Psi^{-1}(E)=0$ by \eqref{eq:Psi-1}.
$\Psi^0(E) \in \FF_i$ and $\Psi^1(E) \in \TT_{i+1}$.
Since
\begin{equation}\label{eq:T_{i+1}}
\begin{split}
\Hom(\Psi(E),T_{i+1}^+[p])& =\Hom(E,\Phi(T_{i+1}^+)[p])=
\Hom(E, T_{i+1}^+(K_X)[p-1])=0,\\
\Hom(\Psi(E),T_{i+1}^+(K_X)[p])& =\Hom(E,\Phi(T_{i+1}^+(K_X))[p])=
\Hom(E, T_{i+1}^+[p-1])=0
\end{split}
\end{equation}
for $p \leq 0$,
$\Psi^1(E) \in \TT_i$.
Therefore $\Psi(E) \in \AA_i$.



Assume that $E \in \TT_{i+1}$.
Then $\ker \psi \in \TT_{i+1}$ and
$\coker \psi \in \TT_{i+1}$.
Hence $\ker \psi \in \TT_{i+1}$ and
$\coker \psi$ is generated by $T_{i+1}^+,T_{i+1}^+(K_X)$. 
Then \eqref{eq:T_{i+1}} implies $\Psi^0(E) \in \TT_i$ and
$\Psi^1(E)=0$.
Therefore $\Psi(E)[-1] \in \AA_i$.
\end{proof}






\begin{Prop}\label{Prop:equiv2}
$R_{T_{i+1}^-}^{-1}$ induces an equivalence
$\AA_i^* \to \AA_{i+1}^*$.
\end{Prop}

\begin{proof}
We set $\Phi:=R_{T_{i+1}^-}$ and $\Psi:=R_{T_{i+1}^-}^{-1}$.
We only show that 
$\Phi(\AA_{i+1}^*) \subset \AA_i^*$.
We note that 
\begin{equation}\label{eq:Phi2}
\begin{split}
\Hom(T_{i+1}^-,\Phi(E)[p])& =\Hom(\Phi(T_{i+1}^-(K_X))[1],\Phi(E)[p])
=\Hom(T_{i+1}^-(K_X),E[p-1])=0,\\
\Hom(T_{i+1}^-(K_X),\Phi(E)[p])&=\Hom(\Phi(T_{i+1}^-)[1],\Phi(E)[p])
=\Hom(T_{i+1}^-,E[p-1])=0
\end{split}
\end{equation}
for $E \in \PP(1)$ and $p \leq 0$.
Assume that $E \in \FF_{i+1}^*$.
For the morphism
\begin{equation}
\varphi:\Hom(T_{i+1}^-,E) \otimes T_{i+1}^- \oplus  
\Hom(T_{i+1}^-(K_X),E) \otimes T_{i+1}^- (K_X)
\to E,
\end{equation}
$\ker\varphi$ and $\im \varphi$ are generated by
$T_{i+1}^-,T_{i+1}^- (K_X)$.
By \eqref{eq:Phi2}, $\Phi^{-1}(E)=0$ and $\Phi^0(E) \in \FF_i^*$.
Since $\Phi^1(E)$ is generated by
$T_{i+1}^-,T_{i+1}^- (K_X)$,
$\Phi^1(E) \in \TT_i^*$.
Therefore $\Phi(E[1]) \in \AA_i^*$.


Assume that $E \in \TT_{i+1}^*$.
Then $\Phi^1(E)=0$.
Since $\Phi^{-1}(E) \in \FF_{i+1}^*$, \eqref{eq:Phi2} implies
$\Phi^{-1}(E) \in \FF_i^*$.
Since $\coker \varphi, T_{i+1}^-, T_{i+1}^- (K_X) \in \TT_i^*$,
$\Phi^0(E) \in \TT_i^*$.
Therefore $\Phi(E) \in \AA_i^*$.
\end{proof}



\begin{Def}
Assume that $i \leq 0$.
\begin{enumerate}
\item[(1)]
Let $(\TT_i^*,\FF_i^*)$ be a torsion pair of $\PP(1)$ such that
\begin{enumerate}
\item
$\TT_i^*$ is generated by $\sigma_+$-stable objects $E$ with 
$\phi^+(E) \geq \phi^+(T_i^+)$.
\item
$\FF_i^*:=\langle T_0^+,T_0^+ (K_X),...,T_{i+1}^+, T_{i+1}^+ (K_X) \rangle$.
\end{enumerate}
Let $\AA_i^*=\langle \TT_i^*,\FF_i^*[1] \rangle$ be the tilting.
\item[(2)]
Let $(\TT_i,\FF_i)$ be a torsion pair of $\PP(1)$ such that
\begin{enumerate}
\item
$\TT_i:=\langle T_0^-,T_0^- (K_X),...,T_{i+1}^-, T_{i+1}^- (K_X) \rangle$.
\item
$\FF_i$ is generated by $\sigma_-$-stable objects $E$ with 
$\phi^-(E) \leq \phi^-(T_i^-)$.
\end{enumerate}
Let $\AA_i=\langle \TT_i[-1],\FF_i\rangle$ be the tilting.
\end{enumerate}
\end{Def}
Since $\FF_0^*=\TT_0=0$,
we have $\AA_0^*=\AA_0=\PP(1)$.
We also have a similar result to Remark \ref{rem:simple-objects}.

Then we also have the following claims whose proof are similar to those 
for Proposition \ref{Prop:equiv1} and
Proposition \ref{Prop:equiv2}.
\begin{Prop}\label{Prop:equiv3}
Assume that $i \leq 0$.
\begin{enumerate}
\item[(1)]
We have an equivalence
$R_{T_i^+}^{-1}:\AA_i^* \to \AA_{i-1}^*$.
\item[(2)]
We have an equivalence
$R_{T_i^-}:\AA_i \to \AA_{i-1}$.
\end{enumerate}
\end{Prop}







\subsection{Stability}
In order to study the behavior of the stability under $R_{T_i^\pm}$,
we prepare the following notions. 
Let $\BB$ be an abelian category with a  
stability function $Z:{\bf D}(\BB) \to \C$ such that
$Z(\BB \setminus \{0 \})= \H \cup \R_{<0}$. 
We set $\phi(E):=\arg Z(E)$.
\begin{Def}
$E \in \BB$ is $Z$-semi-stable if
$\phi(F) \leq \phi(E)$ for all subobject $F$ of $E$
in $\BB$.
\end{Def}

\begin{Def}\label{defn:B'}
For a real number $\theta \in (-1,1)$, 
we set $Z_\theta:=e^{\pi \sqrt{-1} \theta} Z$.
\begin{enumerate}
\item[(1)]
If $\theta \geq 0$, then
let $(\TT',\FF')$ be a torsion pair of $\BB$
such that
\begin{enumerate}
\item
$\TT'$ is generated by $Z$-stable objects $E \in \BB$ with
$\phi(E)+\theta>1$. 
\item
$\FF'$ is generated by $Z$-stable objects $E \in \BB$ with
$\phi(E)+\theta \leq 1$. 
\end{enumerate}
We set $\BB':=\langle \TT'[-1],\FF' \rangle$.
\item[(2)]
If $\theta \leq 0$, then
let $(\TT'',\FF'')$ be a torsion pair of $\BB$
such that
\begin{enumerate}
\item
$\TT''$ is generated by $Z$-stable objects $E \in \BB$ with
$\phi(E)+\theta>0$. 
\item
$\FF''$ is generated by $Z$-stable objects $E \in \BB$ with
$\phi(E)+\theta \leq 0$. 
\end{enumerate}
We set $\BB'':=\langle \TT'',\FF''[1] \rangle$.
\end{enumerate}
\end{Def}


\begin{Ex}\label{ex:BB}
\begin{enumerate}
\item[(1)]
We take an orientation preserving injective homomorphism 
$Z:\HH \to \C$ such that $Z(u_0),Z(u_1) \in \H \cup \R_{<0}$
and $Z(u_1)/Z(u_0) \in \H$.
Then $E \in \PP(1)$ is $\sigma_+$-semi-stable if and only if
$E$ is $Z$-semi-stable.
In this case, $\AA_i$ $(i \geq 0)$ is an example of $\BB'$
for some $\theta \geq 0$,
and $\AA_i^*$ $(i \leq 0)$ is an example of $\BB''$
for some $\theta \leq 0$. 
\item[(2)]
We take an orientation reversing homomorphism 
$Z:\HH \to \C$ such that $Z(u_0),Z(u_1) \in \H \cup \R_{<0}$
and $Z(u_0)/Z(u_1) \in \H$.
Then $E \in \PP(1)$ is $\sigma_-$-semi-stable if and only if
$E$ is $Z$-semi-stable.
In this case, $\AA_i$ $(i \leq 0)$ is an example of $\BB'$
and $\AA_i^*$ $(i \geq 0)$ is an example of $\BB''$. 
\end{enumerate}
\end{Ex}

 

\begin{Prop}\label{Prop:BB}
\begin{enumerate}
\item[(1)]
%Let $(\TT',\FF')$ be the torsion pair in Definition \ref{defn:B'}. 
Assume that $E\in \BB'$.
Then $E$ is $Z'$-semi-stable if and only if 
\begin{enumerate}
\item[(i)]
 $E \in \FF'$
and $E$ is $Z$-semi-stable or 
\item[(ii)]
 $E[1] \in \TT'$ and
$E[1]$ is $Z$-semi-stable.
\end{enumerate}
\item[(2)]
Assume that $E\in \BB''$.
Then $E$ is $Z''$-semi-stable if and only if 
\begin{enumerate}
\item[(i)]
 $E[-1] \in \FF''$ and
$E[-1]$ is $Z$-semi-stable or
\item[(ii)]
 $E \in \TT''$ and
$E$ is $Z$-semi-stable.
\end{enumerate}
\end{enumerate}
\end{Prop}

For a convenience sake, we give a proof,
although it may be obvious.

\begin{proof}
(1)
Let $E$ be a $Z'$-semi-stable object of $\BB'$.
For the exact sequence
\begin{equation}
0 \to H^0(E) \to E \to H^1(E)[-1] \to 0
\end{equation}
in $\BB'$,
if $H^0(E) \ne 0$ and $H^1(E) \ne 0$, then 
$\phi(H^1(E)[-1])<\phi(E)<\phi(H^0(E))$.
Hence $H^0(E)=0$ or $H^1(E) \ne 0$.

Assume that $H^1(E)=0$, that is, $E \in \FF'$.
Let $E_1$ be a subobject of $E$ in $\BB$.
Then there is a subobject $E_1' \subset E$ in $\BB$ such that
$E_1 \subset E_1'$, $E_1'/E_1 \in \TT'$ and $E/E_1' \in \FF'$.
Since $E/E_1'$ is a quotient object of $E$ in $\BB'$,
$Z'$-semi-stability of $E$ implies 
$\phi(E) \leq \phi(E/E_1')$.
Since $\phi(E_1'/E_1) \geq \phi(E/E_1) \geq \phi(E/E_1')$,
we get $\phi(E) \leq \phi(E/E_1)$.
Therefore $E$ is $Z$-semi-stable.

Assume that $H^0(E)=0$, that is, $E[1] \in \TT'$.
Let $E_1$ be a subobject of $E[1]$ in $\BB$.
Then there is a subobject $E_1' \subset E_1$ in $\BB$ such that
$E_1' \in \TT'$ and $E_1/E_1' \in \FF'$.
Since $E[1]/E_1' \in \TT'$,
$E_1'$ is a subobject of $E[1]$ in $\BB'[1]$.
By the $Z'$-semi-stability of $E$, 
$\phi(E_1') \leq \phi(E[1])$.
Since $\phi(E_1') \geq \phi(E_1) \geq \phi(E_1/E_1')$,
$\phi(E_1) \leq \phi(E[1])$.
Therefore $E$ is $Z$-semi-stable.

We shall prove the converse direction.
Let $E$ be a $Z$-semi-stable object of $\BB$.
We first assume that $E \in \FF'$.
For an exact sequence
\begin{equation}
0 \to E_1 \to E \overset{\varphi}{\to} E_2 \to 0
\end{equation}
in $\BB'$,
$\phi(\im \varphi) \leq \phi(E_2)$ by
$\coker \varphi \in \TT'$.
By the $Z$-semi-stability of $E$,
$\phi(E) \leq \phi(\im \varphi)$.
Hence $\phi(E) \leq \phi(E_2)$.
Therefore $E$ is $Z'$-semi-stable.

We next assume that $E \in \TT'$.
For an exact sequence
\begin{equation}
0 \to E_1 \to E[-1] \to E_2 \to 0
\end{equation}
in $\BB'$,
$H^i(E_1)=0$ for $i \ne 1$ and
we have an exact sequence
\begin{equation}
0 \to H^0(E_2) \to H^1(E_1) \overset{\varphi}{\to} E \to H^1(E_2) \to 0.
\end{equation}
Then 
$\phi(\im \varphi) \geq \phi(H^1(E_1))$ by
$\ker \varphi \in \TT'$.
By the $Z$-semi-stability of $E$,
$\phi(E) \geq \phi(\im \varphi)$.
Hence $\phi(E) \geq \phi(E_2)$.
Therefore $E$ is $Z'$-semi-stable.

(2)
For the abelian category $\BB[1]$ with 
the stability function 
$$
Z^*:\BB[1] \overset{[-1]}{\to} \BB \overset{Z}{\to}\C,
$$  
we see that
$(\BB[1])'=\BB''$, where $Z_{1+\theta}^*=Z_\theta$.
Hence the claim follows from (1).
\end{proof}


\subsubsection{Preservation of stability}

\begin{Prop}\label{Prop:isom-pm}
\begin{enumerate}
\item[(1)]
\begin{enumerate}
\item
$R_{T_1^+}:\AA_0 \to \AA_1$ induces an isomorphism
$M_{\sigma_-}(v) \to M_{\sigma_+}(v')$, where  $v'=R_{T_1^+}(v)$.
\item
$R_{T_1^-}^{-1}:\AA_0^* \to \AA_1^*$ induces an isomorphism
$M_{\sigma_+}(v) \to M_{\sigma_-}(v')$, where  $v'=R_{T_1^-}^{-1}(v)$.
\end{enumerate}
\item[(2)]
\begin{enumerate}
\item
$R_{T_0^-}:\AA_0 \to \AA_{-1}$ induces an isomorphism
$M_{\sigma_+}(v) \to M_{\sigma_-}(v')$, where  $v'=R_{T_0^-}(v)$.
\item
$R_{T_0^+}^{-1}:\AA_0^* \to \AA_{-1}^*$ induces an isomorphism
$M_{\sigma_-}(v) \to M_{\sigma_+}(v')$, where  $v'=R_{T_0^+}^{-1}(v)$.
\end{enumerate}
\end{enumerate}
\end{Prop}

\begin{proof}
By Example \ref{ex:BB}, we can apply Proposition \ref{Prop:BB}
to compare the stabilities on $\AA_i$ and $\AA_i^*$ to 
$\sigma_\pm$-stability on $\AA_0=\AA_0^*$.
Since the orientation of $\HH$ is reversed under the reflection,
the claims follow from Proposition \ref{Prop:BB} and 
Propositions \ref{Prop:equiv1},
\ref{Prop:equiv2}, \ref{Prop:equiv3}.
\end{proof}


\begin{Cor}
If $\langle v,u_1 \rangle=0$,
then $R_{T_1}$ induces an isomorphism
$M_{\sigma_-}(v) \to M_{\sigma_+}(v)$. 
If $\langle v,u_0 \rangle=0$,
then $R_{T_0}$ induces an isomorphism
$M_{\sigma_+}(v)\to M_{\sigma_-}(v)$. 
\end{Cor}




\begin{Prop}\label{Prop:isom-2}
\begin{enumerate}
\item[(1)]
Assume that $i>0$.
\begin{enumerate}
\item
$R_{T_{i+1}^+} \circ R_{T_i^+}:\AA_{i-1} \to \AA_{i+1}$
induces an isomorphism
$M_{\sigma_+}(v) \to M_{\sigma_+}(v')$, where 
$v'=R_{T_{i+1}^+} \circ R_{T_i^+}(v)$.
\item
$R_{T_{i+1}^-}^{-1} \circ R_{T_i^-}^{-1}:\AA_{i-1}^* \to \AA_{i+1}^*$
induces an isomorphism
$M_{\sigma_-}(v) \to M_{\sigma_-}(v')$, where
$v'=R_{T_{i+1}^-}^{-1} \circ R_{T_i^-}(v)$.
\end{enumerate}
\item[(2)]
Assume that $i < 0$.
\begin{enumerate}
\item
$R_{T_{i}^-} \circ R_{T_{i+1}^-}:\AA_{i+1} \to \AA_{i-1}$
induces an isomorphism
$M_{\sigma_-}(v) \to M_{\sigma_-}(v')$, where
$v'=R_{T_{i}^-} \circ R_{T_{i+1}^-}(v)$.
\item
$R_{T_{i}^+}^{-1} \circ R_{T_{i+1}^+}^{-1}:\AA_{i+1}^* \to \AA_{i-1}^*$
induces an isomorphism
$M_{\sigma_+}(v) \to M_{\sigma_+}(v')$, where
$v'=R_{T_{i}^+}^{-1} \circ R_{T_{i+1}^+}^{-1}(v)$.
\end{enumerate}
\end{enumerate}
\end{Prop}

\begin{proof}
(1)
Since $R_{T_{i+1}^+} \circ R_{T_i^+}$ and 
$R_{T_{i+1}^-}^{-1} \circ R_{T_i^-}^{-1}$
preserve the orientation,
we get the claims by a similar argument as in Proposition \ref{Prop:isom-pm}.  
The proof of (2) is similar.
\end{proof}



\begin{Lem}
\begin{enumerate}
\item[(1)]
\begin{equation}
R_{T_i^+}(T_{i-1}^+) =
\begin{cases}
T_{i+1}^+[1],T_{i+1}^+(K_X)[1], & i \ne 0,1,\\
T_{i+1}^+, T_{i+1}^+(K_X), & i =0,1.
\end{cases}
\end{equation}
\item[(2)]
\begin{equation}
R_{T_i^+} \circ R_{T_{i-1}^+}=R_{T_{i+1}^+} \circ R_{T_i^+}.
\end{equation}
In particular
$R_{T_{i+1}^+} \circ R_{T_i^+} =R_{T_1} \circ R_{T_0}$ for all $i$.
\end{enumerate}
\end{Lem}

\begin{proof}
(1)
Assume that $i \geq 2$.
By Proposition \ref{Prop:isom-2} (1),
$R_{T_{i}^+}\circ R_{T_{i-1}^+}:\AA_{i-2} \to \AA_{i}$
induces an isomorphism
$$
M_{\sigma_+}(u_{i-1}) \to M_{\sigma_+}(u_{i+1}).
$$
Hence $R_{T_{i}^+}\circ R_{T_{i-1}^+}(T_{i-1}^+)
=R_{T_{i}^+}(T_{i-1}^+[-1])$
is a $\sigma_+$-stable object with the Mukai vector $u_{i+1}$.
Then we get 
$R_{T_{i}^+}(T_{i-1}^+[-1])=T_{i+1}^+,T_{i+1}^+(K_X)$.
If $i=0,1$, then Proposition \ref{Prop:isom-pm}
implies $R_{T_i^+}(T_{i-1}^+)=T_{i+1}^+,T_{i+1}(K_X)$.
Assume that $i<0$.
Then Proposition \ref{Prop:isom-2} (2) implies
$R_{T_{i}^+}\circ R_{T_{i-1}^+}:\AA_{i-2}^* \to \AA_{i}^*$
induces an isomorphism
$$
M_{\sigma_+}(-u_{i-1}) \to M_{\sigma_+}(-u_{i+1}).
$$
Hence we get 
$R_{T_{i}^+}(T_{i-1}^+)=T_{i+1}^+[1],T_{i+1}^+(K_X)[1]$.




(2)
Since $R_{T_i^+} \circ R_{T_{i-1}^+} \circ R_{T_i^+}^{-1}=
R_{R_{T_i^+}(T_{i-1}^+)}$, the claim follows from (1).
\end{proof}


In the same way, we also see that
 $R_{T_i^-} \circ R_{T_{i+1}^-} =R_{T_0} \circ R_{T_1}$ for all $i$.

\begin{Def}
We set $R_+:=R_{T_1} \circ R_{T_0}$ and
$R_-:=R_{T_0} \circ R_{T_1}$.
\end{Def}




Since $M_{\sigma^\pm}(v_0) \setminus M_{\sigma^\mp}(v_0)$
is at least of codimension 2,
we get the following results from Proposition \ref{Prop:isom-pm} and
Proposition \ref{Prop:isom-2}.


\begin{Prop}\label{Prop:NonMinimalIsomorphism}
\begin{enumerate}
\item[(1)]
Assume that $n$ is even and $v_n \in \CC_n$.
%If $n>0$, then
Then $R_+^{\frac{n}{2}} \circ R_-^{\frac{n}{2}}$ induces a birational map
\begin{equation}
M_{\sigma_-}(v_n) \cong M_{\sigma_-}(v_0) \cdots \to M_{\sigma_+}(v_0)
\cong M_{\sigma_+}(v_n).
\end{equation}
\item[(2)]
Assume that $n$ is odd and $v_n \in \CC_n$.
Then $R_+^{\frac{n-1}{2}} \circ R_{T_1} \circ R_{T_1} \circ R_-^{\frac{n-1}{2}}$
induces a birational map
\begin{equation}
M_{\sigma_-}(v_n) \cong M_{\sigma_-}(v_1) \cong M_{\sigma_+}(v_0) 
\cdots \to M_{\sigma_-}(v_0) \cong M_{\sigma_+}(v_1)
\cong M_{\sigma_+}(v_n).
\end{equation}
\end{enumerate}
\end{Prop}




Assume that $\langle v, u_n \rangle=0$.
If $n$ is odd, then 
we  take $v_1 =R_-^{\frac{n-1}{2}}(v) \in \HH$.
Then $v_1 \in u_1^\perp$.
In this case, 
$R_+^{\frac{n-1}{2}} \circ R_{T_1^+}
\circ R_-^{\frac{n-1}{2}}$ induces
an isomorphism
\begin{equation}\label{eq:FM-birat3}
M_{\sigma_-}(v) \cong M_{\sigma_-}(v_1) \cong M_{\sigma_+}(v_1) 
\cong M_{\sigma_+}(v).
\end{equation}

If $n$ is even, then we take $v_0=R_-^{\frac{n}{2}}(v) \in \HH$.
Then $v_0 \in u_0^\perp$.
In this case, 
$R_+^{\frac{n}{2}} \circ R_{T_0^+}
\circ R_-^{\frac{n}{2}}$ induces
an isomorphism
\begin{equation}\label{eq:FM-birat4}
M_{\sigma_-}(v) \cong M_{\sigma_-}(v_0) \cong M_{\sigma_+}(v_0) 
\cong M_{\sigma_+}(v).
\end{equation}


\begin{Prop}\label{Prop:Pic-relation}
We identify $\Pic(M_{\sigma_+}(v,L))$ with 
$\Pic(M_{\sigma_-}(v,L))$
by the birational maps in Proposition \ref{Prop:NonMinimalIsomorphism} 
or \eqref{eq:FM-birat3}, \eqref{eq:FM-birat4}.
Then we have the following relations:
\begin{equation}
\begin{split}
\theta_{v,\sigma_+}=& \theta_{v,\sigma_-},\; (v \in \CC_n),\\
\theta_{v,\sigma_+}=& \theta_{v,\sigma_-} \circ R_{u_n} ,\; (v \in u_n^\perp).
\end{split}
\end{equation}
\end{Prop}


Since $\xi_{\sigma_0} \in u_n^\perp$,
$\theta_{v,\sigma_+}(\xi_{\sigma_0})=
\theta_{v,\sigma_-}(\xi_{\sigma_0})$
by Proposition \ref{Prop:Pic-relation}.
\textcolor{red}{Need to define $\xi_\sigma$}
Hence the linear systems
$|n \theta_{v,\sigma_\pm}(\xi_{\sigma_0})|$ $(n \gg 0)$
define morphisms $\pi_\pm:M_{\sigma_{\pm}}(v) \to \P$
such that $\im \pi_+=\im \pi_-$.
Let $Z \to \im \pi_+=\im \pi_-$ be the normalization.
 
Let $E$ be a $\sigma_0$-stable object. Then
it is an irreducible object of $\AA_0$.
Let $n$ be a non-negative integer.
Then $E_n^+:=R_{T_n^+} \circ R_{T_{n-1}^+} \circ  \cdots \circ R_{T_1^+}(E)$
is an irreducible object of $\AA_n$
and $E_n^-:=R_{T_n^-}^{-1} \circ 
R_{T_{n-1}^-}^{-1} \circ  \cdots \circ R_{T_1^-}^{-1}(E)$
is an irreducible object of $\AA_n^*$.
By Remark \ref{rem:simple-objects} and 
the definition of $R_{T_i^\pm}$,
$E_n^\pm$ are successive extension of $E$ by $T_i^\pm$.
In particular, $E_n^\pm$ is $S$-equivalent to 
$E \oplus T_0^{\oplus k_0} \oplus T_1^{\oplus k_1}$
with respect to $\sigma_0$, where 
$v(E_n^\pm)=v(E)+k_0 v(T_0)+k_1 v(T_1)$. 
Hence $M_{\sigma_\pm}(v) \to Z$ are 
birational. Indeed if $v=v_n \in \CC_n$ and $n$ is even, then
$M_{\sigma_\pm}(v_0) \to M_{\sigma_\pm}(v) \to \im \pi_\pm$
is injective on the open subscheme $M_{\sigma_0}(v_0)^s$
of $\sigma_0$-stable objects.  
Hence the birational maps in Proposition \ref{Prop:Pic-relation}
are the maps
$M_{\sigma_-}(v) \to Z \cdots \to M_{\sigma_+}(v)$.

\begin{Rem}
Let $M'_{\sigma_0}(v)$ be the set of 
$S$-equivalence classes
of $\sigma_0$-semi-stable objects $E$ with $v(E)=v$.
Then $\im \pi_\pm$ are regarded as a subset of $M'_{\sigma_0}(v)$.
However $\im \pi_\pm$ are proper subset of $M'_{\sigma_0}(v)$
in general. 
\end{Rem}




Since $R_{T_i}(E)=E$ $(i=0,1)$ for a $\sigma_0$-stable object $E$ with 
$\langle v(E),v(T_i) \rangle=0$,
similar claims also hold for the compositions of isomorphisms 
\eqref{eq:FM-birat3} and \eqref{eq:FM-birat4} with $M_{\sigma_+}(v) \to 
M_{\sigma_0}(v)$.






\subsection{}

Assume that $\sqrt{D}$ is rational.
In this case, there is an isotropic Mukai vector in $\HH$.
Let $T$ be a unique $\sigma_0$-stable object. Then 
$\phi^+(T) >\phi^+(E)$ for all $\sigma_+$-stable object $E \ne T$
or      
$\phi^+(T) <\phi^+(E)$ for all $\sigma_+$-stable object $E \ne T$.
Let $v$ be a Mukai vector with $v^2\geq 0$.
If $\langle v,v(T) \rangle >0$,
then we have a birational map 
$M_{\sigma_+}(v) \cdots \to M_{\sigma_-}(v)$.
Assume that $\langle v,v(T) \rangle <0$.
Then $v'=R_{T}(v)$ satisfies 
$\langle v',v(T) \rangle >0$.  
We define $\AA_{-1}^*$, $\AA_0$ and $\AA_1$
as in the previous subsection.
Then $R_T$ induces equivalences $\AA_{-1}^* \to \AA_0$ and
$\AA_0 \to \AA_1$.
We also have isomorphisms 
$M_{\sigma_-}(v) \to M_{\sigma_+}(v')$ and
$M_{\sigma_-}(v') \to M_{\sigma_+}(v)$. 
Hence $R_T \circ R_T$ induces a birational map
\begin{equation}
M_{\sigma_-}(v) \to M_{\sigma_+}(v') \cdots 
\to M_{\sigma_-}(v') \to M_{\sigma_+}(v).
\end{equation}

If $\langle v, v(T) \rangle=0$, then 
$R_T$ induces an isomorphism
$M_{\sigma_-}(v) \cong M_{\sigma_+}(v)$.


























%\begin{Cor}\label{Cor:NonMinimalIsomorphism}
%\begin{enumerate}
%\item[(1)]
%Assume that $n$ is even and $v_n \in \CC_n$.
%Then we have a birational map
%\begin{equation}
%M_{\sigma^+}(v_n) \cong M_{\sigma^+}(v_0) \cdots \to M_{\sigma^-}(v_0)
%\cong M_{\sigma^-}(v_n).
%\end{equation}
%\item[(2)]
%Assume that $n$ is odd and $v_n \in \CC_n$.
%Then we have a birational map
%\begin{equation}
%M_{\sigma^+}(v_n) \cong M_{\sigma^+}(v_1) \cong M_{\sigma^-}(v_0) 
%\cdots \to M_{\sigma^+}(v_0) \cong M_{\sigma^+}(v_1)
%\cong M_{\sigma^-}(v_n).
%\end{equation}
%\end{enumerate}
%\end{Cor}


\subsection{Case (ciii): Exactly one $\sigma_0$-stable spherical object and one exceptional object}


We shall briefly explain that the case where $\HH$ contains 
exceptional and spherical vectors.
Let $\HH=\Z w_1 +\Z w_2$ be a hyperbolic lattice
such that $w_1^2=-1$, $\langle w_1,w_2 \rangle=0$ and
$D:=w_2^2>0$.
Assume that there is $u \in \HH$ with $u^2=-2$.
Then 
$x^2-D y^2=2$ has an integral solution, which implies
$\sqrt{D}$ is irrational.
Let $(s,t)$ be a solution of $x^2-D y^2=2$.
We set $\alpha:=s+t \sqrt{D}$.
Then $-\alpha^2/2=(1-s^2)-s t \sqrt{D},$
and $(x,y)=(1-s^2,-s t)$ is a solution of
$x^2-y^2D=1$.
Let $(x_1,y_1)$ be a minimal solution of
$x^2-y^2D=1$ such that $x_1<0$ and $y_1>0$, and set
$\beta:=x_1+y_1 \sqrt{D}$.
Then $\alpha^2/2=\pm \beta^n$.
If $n$ is even, then
$u+v\sqrt{D}:=\alpha/\beta^{\frac{n}{2}}$
satisfies $(u+v\sqrt{D})^2=2$ and $N(\alpha/\beta^{\frac{n}{2}})=u^2-v^2 D=2$,
which is a contradiction, where 
$N(x)$ is the norm of $x \in \Z[\sqrt{D}]$.
Hence we may set $n=2k+1$.
Then $s_1+t_1 \sqrt{D}=\alpha/\beta^{k}$
satisfies $(s_1+t_1 \sqrt{D})^2=\pm 2\beta$ and
$s_1^2-t_1^2 D=2$.
So replacing $(s,t)$ by $(s_1,t_1)$,
we assume that $\alpha:=s_1 +t_1 \sqrt{D}$ satisfies
$N(\alpha)=2$ and $\beta:=-\alpha^2/2=(1-s_1^2)-s_1 t_1 \sqrt{D}$
is the generator of the unit group
$\{ \gamma \in \Z[\sqrt{D}] \mid N(\gamma)=1 \}$.
We may also assume that $s_1<0$ and $t_1>0$.
We define $(s_n,t_n) \in \Z \times \Z$ by 
\begin{equation}
\begin{split}
s_{2n}+t_{2n} \sqrt{D}:=& (-1)^{n+1}\beta^n=-\alpha^{2n}/2^n,\; (n \geq 1)\\
s_{2n-1}+t_{2n-1}\sqrt{D}:=& \alpha^{2n-1}/2^{n-1},\; (n \geq 1)\\
s_0+t_0\sqrt{D}:=& 1,\\
s_{-2n}+t_{-2n} \sqrt{D}:=&-s_{2n}+t_{2n}\sqrt{D}=(-\beta)^{-n},\; (n \geq 1)\\
s_{-2n+1}+t_{-2n+1} \sqrt{D}:=& -s_{2n-1}+t_{2n-1}\sqrt{D}
=2^{n} (-\alpha)^{-(2n-1)},\; (n \geq 1).\\
\end{split}
\end{equation}
%
For $n \in \Z$, we set $u_n:=s_n w_1+t_n w_2$.
Then 
$$
u_n=\begin{cases}
-1,& 2 \mid n\\
-2, & 2 \nmid n
\end{cases} 
$$
It is easy to see that
\begin{equation}
\begin{split}
u_{n+1}=& -R_{u_n}(u_{n-1}),\;(n \geq 2)\\
u_{n-1}=& -R_{u_n}(u_{n+1}),\;(n \leq -1)\\
u_2=& R_{u_1}(u_0),\;u_{-1}=R_{u_0}(u_1).
\end{split}
\end{equation}








We set 
\begin{equation}
\Delta:=\{ u \in \HH \mid u^2=-1,-2 \}.
\end{equation}
Then $\Delta=\{\pm u_n \mid n\in \Z \}$.
As in subsection,
we take $w, w' \in \Delta$
such that
$$
\{v \in \HH \mid v \in L^+, v^2 \geq -2 \} \subset
\Q_{\geq 0}w+\Q_{\geq 0}(-w').
$$ 
Replacing the basis, we may assume that $u_0=w_1=w$
and $u_1=-w'$. 
Then $\Delta \cap L^+=\{ u_n \mid n \in \Z \}$.
Assume that there is a $\sigma_0$-semi-stable object $E$ with 
$v(E)^2=-2$ and $v(E) \in \HH$.
Then there is an integer $j$ with
$v(E)=u_{2j+1}$. We note that $u_{2n+1} \equiv u_1 \mod 2$ for all $n \in \Z$.
Hence $c_1(u_{2n+1}) \equiv Z \mod 2$, where
$Z$ is a nodal cycle (Theorem \ref{Thm:generic moduli spaces}).   
Then there are $\sigma_\pm$-stable objects $T_i^\pm$ with
$v(T_i^\pm)=u_i$.
By our choice of $u_0$ and $u_1$,
$T_0^\pm$ and $T_1^\pm$ are $\sigma_0$-stable objects.
In particular
$T_0^+=T_0^-$ and $T_1^+=T_1^-$.


Let $\CC_n \subset L^+$ be the region between 
$u_n^\perp$ and $u_{n+1}^\perp$. 
Then $\CC_0$ is the fundamental domain of the Weyl group associated to
$\Delta$.
We may assume that  
\begin{equation}
\phi^+(T_1^+) > \phi^+(T_2^+)>\cdots>\phi^+(E)>
\cdots >\phi^+(T_{-1}^+)>\phi^+(T_0^+)
\end{equation}
for any $\sigma_+$-stable object $E$ with
$v(E)^2 \geq 0$
and
\begin{equation}
\phi^-(T_1^-) < \phi^-(T_2^-)<\cdots<\phi^-(E)<
\cdots <\phi^-(T_{-1}^-)<\phi^-(T_0^-)
\end{equation}
for any $\sigma_-$-stable object $E$ with
$v(E)^2 \geq 0$.

We note that $T_i^\pm (K_X)=T_i^\pm$ if $i$ is odd and
$T_i^\pm (K_X) \not \cong T_i^\pm$ if $i$ is even.

\begin{Def}
\begin{enumerate}
\item[(1)]
For an exceptional object $E_0$,
let $R_{E_0}:{\bf D}(X) \to {\bf D}(X)$ be an equivalence such that
\begin{equation}
R_{E_0}(E):=
\mathrm{cone}({\bf R}\Hom(E_0,E)\otimes E_0 \oplus 
 {\bf R}\Hom(E_0(K_X),E)\otimes E_0(K_X) \to E).
\end{equation}
\item[(2)]
For a spherical object $E_0$,
let $R_{E_0}:{\bf D}(X) \to {\bf D}(X)$ be an equivalence such that
\begin{equation}
R_{E_0}(E):=\mathrm{cone}({\bf R}\Hom(E_0,E)\otimes E_0  \to E).
\end{equation}
\end{enumerate}
\end{Def}



Then by using these equivalences, we see that
\begin{equation}
R_+:=R_{T_1} \circ R_{T_0} (=R_{T_{i+1}^+} \circ R_{T_i^+}) 
\end{equation}
is an equivalence which preserves $\sigma_+$-semi-stability
and
\begin{equation}
R_-:=R_{T_0} \circ R_{T_1} (=R_{T_{i-1}^-} \circ R_{T_i^-}) 
\end{equation}
is an equivalence which preserves $\sigma_-$-semi-stability.
Then we have the following results.



\begin{Prop}\label{Prop:CompositionSphericalExceptional}
\begin{enumerate}
\item[(1)]
\begin{enumerate}
\item
Assume that $n$ is even and $v_n \in \CC_n$.
Then $R_-^{\frac{n}{2}} \circ R_+^{\frac{n}{2}}$ induces a birational map
\begin{equation}
M_{\sigma_-}(v_n) \cong M_{\sigma_-}(v_0) \cdots \to M_{\sigma_+}(v_0)
\cong M_{\sigma_+}(v_n).
\end{equation}
\item
Assume that $n$ is odd and $v_n \in \CC_n$.
Then $R_-^{\frac{n-1}{2}} \circ R_{T_1}^2 \circ R_+^{\frac{n-1}{2}}$ induces a birational map
\begin{equation}
M_{\sigma_-}(v_n) \cong M_{\sigma_-}(v_1) \cong M_{\sigma_+}(v_0) 
\cdots \to M_{\sigma_-}(v_0) \cong M_{\sigma_+}(v_1)
\cong M_{\sigma_+}(v_n).
\end{equation}
\end{enumerate}
\item[(2)]
Assume that $\langle v, u_n \rangle=0$.
\begin{enumerate}
\item
If $n$ is odd, then 
$R_+^{\frac{n-1}{2}} \circ R_{T_1^+}
\circ R_-^{\frac{n-1}{2}}$ induces
an isomorphism
\begin{equation}
M_{\sigma_-}(v) \cong M_{\sigma_-}(v_1) \cong M_{\sigma_+}(v_1) 
\cong M_{\sigma_+}(v),
\end{equation}
where $v_1=R_-^{\frac{n-1}{2}}(v)$.
\item
If $n$ is even, 
$R_+^{\frac{n}{2}} \circ R_{T_0^+}
\circ R_-^{\frac{n}{2}}$ induces
an isomorphism
\begin{equation}
M_{\sigma_-}(v_{n}) \cong M_{\sigma_-}(v_0) \cong M_{\sigma_+}(v_0) 
\cong M_{\sigma_+}(v_{n}),
\end{equation}
where $v_0=R_-^{\frac{n}{2}}(v)$.
\end{enumerate}
\end{enumerate}
\end{Prop}



























\subsubsection{An outline of the arguments.}

\begin{Def}
Assume that $i  \geq 0$.
\begin{enumerate}
\item[(1)]
Let $(\TT_i,\FF_i)$ be a torsion pair of $\PP(1)$ such that
\begin{enumerate}
\item
$\TT_i=\langle T_1^+,T_1^+(K_X),T_2^+,T_2^+ (K_X),...,T_i^+,
T_i^+ (K_X) \rangle$ 
is the subcategory of $\PP(1)$ generated by $\sigma_+$-stable objects
$F$ with $\phi^+(F)>\phi^+(T_{i+1}^+)$ and
\item
$\FF_i$ is the subcategory of $\PP(1)$ generated by 
$\sigma_+$-stable objects $F$
with $\phi^+(F) \leq \phi^+(T_{i+1}^+)$.
\end{enumerate}
Let $\AA_i:=\langle \TT_i[-1],\FF_i \rangle$ be the tilting.
\item[(2)]
Let $(\TT_i^*,\FF_i^*)$ be a torsion pair of $\PP(1)$ such that
\begin{enumerate}
\item
$\TT_i^*$ is the subcategory of $\PP(1)$ generated by 
$\sigma_-$-stable objects $F$
with $\phi^-(F) \geq \phi^-(T_{i+1}^-)$.
\item
$\FF_i^*=\langle T_1^-,T_1^-(K_X),T_2^-,T_2^-(K_X),...,
T_i^-,T_i^-(K_X) \rangle$ 
is the subcategory of $\PP(1)$ generated by $\sigma_-$-stable objects
$F$ with $\phi^-(F)<\phi^-(T_{i+1}^-)$.
\end{enumerate}
Let $\AA_i^*:=\langle \TT_i^*,\FF_i^*[1] \rangle$ be the tilting.
\end{enumerate}
\end{Def}
We also define $\AA_i, \AA_i^*$ for $i <0$ in a similar way.


%For a $\sigma$-stable object $E_0$ with $v(E_0)^2>0$, we set 
%$E_i:=R_{T_i^\pm}^{-1}(E_{i-1})$.








\begin{Prop}\label{Prop:equiv1-2}
$R_{T_{i+1}^+}$ induces an equivalence
$\AA_i \to \AA_{i+1}$.
\end{Prop}

\begin{proof}
We set $\Phi:=R_{T_{i+1}^+}$.
Assume that $i$ is odd.
For $E \in \PP(1)$, 
$$
\Ext^p(T_{i+1}^+,E)=\Ext^p(T_{i+1}^+(K_X),E)=0
$$
 for $p \ne 0,1,2$.
Hence we have an exact sequence
\begin{equation}
\begin{CD}
0 @>>> H^{-1}(\Phi(E)) @>>> \Hom(T_{i+1}^+,E) \otimes T_{i+1}^+ \oplus  
\Hom(T_{i+1}^+(K_X),E) \otimes T_{i+1}^+ (K_X) @>{\varphi}>> E \\
 @>>> H^0(\Phi(E)) @>>> \Ext^1(T_{i+1}^+,E) \otimes T_{i+1}^+ \oplus  
\Ext^1(T_{i+1}^+(K_X),E) \otimes T_{i+1}^+ (K_X) @>>> 0
\end{CD}
\end{equation}
and also an isomorphism
\begin{equation}
H^1(\Phi(E)) \cong \Ext^2(T_{i+1}^+,E) \otimes T_{i+1}^+\oplus  
\Ext^2(T_{i+1}^+(K_X),E) \otimes T_{i+1}^+ (K_X) \in \TT_{i+1}.
\end{equation}
Since $\Phi(T_i^+ (nK_X))=T_i^+((n+1)K_X)$, we have
\begin{equation}\label{eq:T_{i+1}:Phi}
\begin{split}
\Hom(T_{i+1}^+,\Phi(E)[p])& =\Hom(\Phi(T_{i+1}^+(K_X)),\Phi(E)[p-1])
=\Hom(T_{i+1}^+(K_X),E[p-1])=0,\\
\Hom(T_{i+1}^+(K_X),\Phi(E)[p])&=\Hom(\Phi(T_{i+1}^+),\Phi(E)[p-1])
=\Hom(T_{i+1}^+,E[p-1])=0
\end{split}
\end{equation}
for $p \leq 0$.

Assume that $E \in \FF_i$. Then
$\phi_{\max}^+(E) \leq \phi^+(T_{i+1}^+)$. Hence 
$\varphi$ is injective and $\coker \varphi \in \FF_i$. 
Then we have $H^0(\Phi(E)) \in \FF_i$.
By \eqref{eq:T_{i+1}:Phi}, we get
$H^0(\Phi(E)) \in \FF_{i+1}$.
Therefore $\Phi(E) \in \AA_{i+1}$.

For $E \in \TT_i$, 
$\coker \varphi \in \TT_i \subset \TT_{i+1}$.
Since $T_{i+1}^+, T_{i+1}^+ (K_X)\in \TT_{i+1}$,
$H^0(\Phi(E)) \in \TT_{i+1}$.
By $T_{i+1}^+,T_{i+1}^+(K_X) \in \FF_i$, $\Phi^{-1}(E) \in \FF_i$.
By \eqref{eq:T_{i+1}:Phi}, we have
$H^{-1}(\Phi(E)) \in \FF_{i+1}$.
Therefore $\Phi(E[-1]) \in \AA_{i+1}$.



Let $\Psi$ be the inverse of $\Phi$.
Then
\begin{equation}\label{eq:Psi-1:2}
\Hom(T_{i+1}^+,E) \otimes T_{i+1}^+ \oplus
\Hom(T_{i+1}^+(K_X),E) \otimes T_{i+1}^+(K_X) \cong \Psi^{-1}(E)
\end{equation}
and we have an exact sequence
\begin{equation}
\begin{CD}
0 @>>> \Ext^1(T_{i+1}^+,E) \otimes T_{i+1}^+ \oplus
\Ext^1(T_{i+1}^+(K_X),E) \otimes T_{i+1}^+(K_X) @>>> \Psi^0(E) @>>>E \\
 @>{\psi}>> \Ext^2(T_{i+1}^+,E) \otimes T_{i+1}^+ \oplus
\Ext^2(T_{i+1}^+(K_X),E) \otimes T_{i+1}^+(K_X) @>>> \Psi^1(E) @>>>0.
\end{CD}
\end{equation}
We also have
\begin{equation}\label{eq:T_{i+1}:2}
\begin{split}
\Hom(\Psi(E),T_{i+1}^+[p])& =\Hom(E,\Phi(T_{i+1}^+)[p])=
\Hom(E, T_{i+1}^+(K_X)[p-1])=0,\\
\Hom(\Psi(E),T_{i+1}^+(K_X)[p])& =\Hom(E,\Phi(T_{i+1}^+(K_X))[p])=
\Hom(E, T_{i+1}^+[p-1])=0
\end{split}
\end{equation}
for $p \leq 0$.



Assume that 
$E \in \FF_{i+1}$.
Then $H^{-1}(\Psi(E))=0$ by \eqref{eq:Psi-1:2}.
$H^0(\Psi(E)) \in \FF_{i}$ and $H^1(\Psi(E)) \in \TT_{i+1}$.
By \eqref{eq:T_{i+1}:2}, we get
$H^1(\Psi(E)) \in \TT_i$.
Therefore $\Psi(E) \in \AA_i$.

Assume that $E \in \TT_{i+1}$.
Then $\ker \psi \in \TT_{i+1}$ and
$\coker \psi \in \TT_{i+1}$.
Hence $H^0(\Psi(E)) \in \TT_{i+1}$ and
$\coker \psi$ is generated by $T_{i+1}^+,T_{i+1}^+(K_X)$.
Then \eqref{eq:T_{i+1}} implies 
$H^0(\Psi(E)) \in \TT_i$ and
$H^1(\Psi(E))=0$.
Therefore $\Psi(E)[-1] \in \AA_i$.

If $i$ is even, then
replacing 
$$
{\bf R}\Hom(T_{i+1}^+,E) \otimes T_{i+1}^+
\oplus {\bf R}\Hom(T_{i+1}^+ (K_X),E) \otimes T_{i+1}^+ (K_X)
\to E$$
 by
$$
{\bf R}\Hom(T_{i+1}^+,E) \otimes T_{i+1}^+
\to E,
$$ 
we get the same claim.
\end{proof}






\begin{Prop}\label{Prop:equiv2-2}
$R_{T_{i+1}^-}^{-1}$ induces an equivalence
$\AA_i^* \to \AA_{i+1}^*$.
\end{Prop}

\begin{proof}
We set $\Phi:=R_{T_{i+1}^-}$ and $\Psi:=R_{T_{i+1}^-}^{-1}$.
We only show that 
$\Phi(\AA_{i+1}^*) \subset \AA_i^*$.
We note that 
\begin{equation}\label{eq:Phi2:2}
\begin{split}
\Hom(T_{i+1}^-,\Phi(E)[p])& =\Hom(\Phi(T_{i+1}^-(K_X))[1],\Phi(E)[p])
=\Hom(T_{i+1}^-(K_X),E[p-1])=0,\\
\Hom(T_{i+1}^-(K_X),\Phi(E)[p])&=\Hom(\Phi(T_{i+1}^-)[1],\Phi(E)[p])
=\Hom(T_{i+1}^-,E[p-1])=0
\end{split}
\end{equation}
for $E \in \PP(1)$ and $p \leq 0$.

Assume that $E \in \FF_{i+1}^*$.
For the morphism
\begin{equation}
\varphi:\Hom(T_{i+1}^-,E) \otimes T_{i+1}^- \oplus  
\Hom(T_{i+1}^-(K_X),E) \otimes T_{i+1}^- (K_X)
\to E,
\end{equation}
we have $\ker\varphi$ and $\im \varphi$ are generated by
$T_{i+1}^-,T_{i+1}^- (K_X)$.
Hence $H^0(\Phi(E)) \in \FF_{i+1}^*$.
By \eqref{eq:Phi2:2}, $H^{-1}(\Phi(E))=0$ and 
$H^0(\Phi(E)) \in \FF_i^*$.
Since $H^1(\Phi(E))$ is generated by
$T_{i+1}^-,T_{i+1}^- (K_X)$,
$H^1(\Phi(E)) \in \TT_i^*$.
Therefore $\Phi(E[1]) \in \AA_i^*$.
Assume that $E \in \TT_{i+1}^*$.
Then $H^1(\Phi(E))=0$.
Since $H^{-1}(\Phi(E)) \in \FF_{i+1}^*$, \eqref{eq:Phi2:2} implies
$H^{-1}(\Phi(E)) \in \FF_i^*$.
Since $\coker \varphi, T_{i+1}^-, T_{i+1}^- (K_X) \in \TT_i^*$,
$H^0(\Phi(E)) \in \TT_i^*$.
Therefore $\Phi(E) \in \AA_i^*$.
\end{proof}







\section{Divisorial contractions in the non-isotropic case}
In this section we aim to prove the following result.

\begin{Prop}
Assume that the potential wall $\WW$ is non-isotropic.  Then $\WW$ induces a divisorial contraction if and only if there exists an effective spherical class $w\in\HH_{\WW}$ such that $\langle v,w\rangle=0$.  If $E_0$ is a $\sigma_{\pm}$-stable spherical object of class $w$, then the contracted divisor can be described as a Brill-Noether divisor of $E_0$: it is given either by the condition $\Hom(E_0,\blank)\neq 0$ or by $\Hom(\blank,E_0)\neq 0$.
\end{Prop}

\begin{Lem}\label{Lem:dimension negative}
If $u=lb$ with $b$ a primitive spherical or exceptional class, then for a generic stability condition $\sigma$, 
$$\dim\MM_{\sigma}(u)= \begin{cases}
 -l^2, & \text{ if }b^2=-2,\\

-\frac{l^2}{2}, & \text{ if }b^2=-1,l\equiv 0\pmod 2,\\

-\frac{l^2+1}{2}, & \text{ if }b^2=-1,l\equiv 1\pmod 2.\\
 \end{cases}$$
\end{Lem}
\begin{proof}
By \cite[Proposition 9.9]{Nue14b}, in case $b^2=-2$, the coarse moduli space $M_{\sigma}(u)$ consists of a single point, $S^{\oplus l}$, where $S$ is the unique $\sigma$-stable spherical object of class $b$.  As $\Aut(S^{\oplus l})=\GL_l(\C)$, we get $$\dim \MM_{\sigma}(u)=\dim M_{\sigma}(u)-\dim\Aut=-l^2.$$  If $b^2=-1$, then by \cite[Lemma 9.2]{Nue14b} the coarse moduli space $M_{\sigma}(u)$ consists of the $l+1$ points $\{E^{\oplus i}\oplus E(K_X)^{\oplus l-i}\}_{i=0}^l$, where $E$ and $E(K_X)$ are the two $\sigma_0$-stable exceptional objects of class $b$.  As $E,E(K_X)$ are both exceptional, $\Aut(E^{\oplus i}\oplus E(K_X)^{\oplus l-i})=\GL_i(\C)\times \GL_{l-i}(\C)$ of dimension $i^2+(l-i)^2$.  But then 
\begin{align}
\begin{split}\dim\MM_{\sigma}(u)&=\max_{0\leq i\leq l}\dim_{E^{\oplus i}\oplus E(K_X)^{\oplus l-i}} M_{\sigma}(u)-\Aut(E^{\oplus i}\oplus E(K_X)^{\oplus l-i})\\
&=-\min_{0\leq i\leq l} i^2+(l-i)^2,\\
\end{split}
\end{align}
which gives the dimension as claimed.
\end{proof}


\begin{Lem}\label{Lem:non-isotropic no totally semistable wall}Suppose that $\HH$ is non-isotropic and $v$ is minimal in its $G_{\HH}$-orbit.  Then $\WW$ cannot be a totally semistable wall, and if $\codim(M_{\sigma_+}(v)\backslash M^s_{\sigma_0}(v))=1$ then $\langle v,w\rangle=0$ for some effective spherical or exceptional class $w$.  Moreover, the generic member has of $M_{\sigma_+}(v)\backslash M^s_{\sigma_0}(v)$ has $\HN$-filtration factors of classes $w$ and $v-w$ with respect to $\sigma_-$-stability.
\end{Lem}
\begin{proof}
Consider the stack $\FF(a_1,\ldots,a_n)$ of Harder-Narasimhan filtrations with to respect to $\sigma_-$-stability $$0\subset F_1\subset F_2\subset \cdots\subset F_n$$ with semistable factors $F_i/F_{i-1}$ of Mukai vector $a_i$ such that $v=\sum_{i=1}^n a_i$.  We consider further the open substack $\FF^0(a_1,\ldots,a_n)$ of filtrations such that $F_n$ is $\sigma_+$-stable, which we assume to be non-empty.  We wish to estimate the codimension of $\FF^0(a_1,\ldots,a_n)$.    

Let $I=\{i|a_i^2>0\}$ and $a:=\sum_{i\in I}a_i$.  Write $b:=v-a$.  If $b^2>0$, then we automatically have $\langle a,b\rangle>\sqrt{a^2 b^2}\geq 1$, so $v^2=a^2+2\langle a,b\rangle+b^2\geq a^2+5$.  If instead $b^2<0$, then by assumption $a^2=v^2-2\langle v,b\rangle+b^2\leq v^2-1$.  So in any case $a^2<v^2$, and we write $a^2=v^2-k$ with $k\in\Z_{\geq 0}$ and $k>0$ if $v\neq a$.  Expanding the squares on each side gives $$0=\sum_{i\in I^c}a_i^2+2\sum_{i<j,(i,j)\in (I^2)^c}\langle a_i,a_j\rangle-k,$$ and rearranging gives \begin{equation}\label{non-isotropic estimate}
\sum_{i<j,(i,j)\in (I^2)^c}\langle a_i,a_j\rangle =\frac{k}{2} -\frac{1}{2}\sum_{i\in I^c}a_i^2.\end{equation}

As $\dim M_+(v)=v^2$ and 
$$\dim \FF(a_1,...,a_n)=\sum_{i=1}^n\dim\MM_{\sigma_-}(a_i)+
\sum_{i<j}\langle a_i,a_j\rangle$$ by \cite[Lemma 5.3]{KY08}, we get that \begin{align}\label{codim estimate}
\begin{split}\codim\FF^0(a_1,\ldots,a_n)&=\sum_i (a_i^2-\dim\MM_{\sigma_-}(a_i))+\sum_{i<j}\langle a_i,a_j\rangle\\
&=\sum_{i\in I^c}(a_i^2-\dim\MM_{\sigma_-}(a_i))+\sum_{i<j,(i,j)\in I^2}\langle a_i,a_j\rangle+\sum_{i<j,(i,j)\in (I^2)^c}\langle a_i,a_j\rangle, 
\end{split}
\end{align}
since $\dim\MM_{\sigma_-}(a_i)=a_i^2$ for $i\in I$.  Using the estimate from \eqref{non-isotropic estimate} and writing $a_i=l_i v_i$ with $v_i$ primitive for $i\in I^c$, we have
\begin{align}\label{eqn: non-isotropic codimension}
\begin{split}
\codim\FF^0(a_1,\ldots,a_n)&=\frac{k}{2}+\sum_{i\in I^c}(\frac{a_i^2}{2}-\dim\MM_{\sigma_-}(a_i))+\sum_{i<j,(i,j)\in I^2}\langle a_i,a_j\rangle\\
&=\frac{k}{2}+\sum_{i\in I^c:v_i^2\equiv l_i\equiv 1\pmod 2}\frac{1}{2}+\sum_{i<j,(i,j)\in I^2}\langle a_i,a_j\rangle,
\end{split}
\end{align}
where the final equality follows from Lemma \ref{Lem:dimension negative}.  Moreover, for $(i,j)\in I^2$ with $i\neq j$, the signature of $\HH$ forces $\langle a_i,a_j\rangle\geq 2$.  Thus $$\codim\FF^0(a_1,\ldots,a_n)\geq \frac{k}{2}+|I|(|I|-1)>0.$$  As this holds true for all possible HN-filtrations of objects in $M_{\sigma_+}(v)$ with respect to $\sigma_-$-stability, $\WW$ cannot be totally semistable.  

For the second and third claim, we note that if $\codim\FF^0(a_1,\ldots,a_n)=1$, then from \eqref{eqn: non-isotropic codimension} we see that $|I|=1$ and either $k=2$ and for any $i\in I^c$ such that $v_i^2=-1$, $l_i$ must be even, or $k=1$ and there is exactly one $i\in I^c$ such that $v^2=-1$ with $l_i$ odd.  Thus $b=v-a$ must satisfy $b^2<0$ and solving for $\langle v,b\rangle$ in $v^2-k=(v-b)^2$ gives $$0\leq 2\langle v,b\rangle=b^2+k\leq b^2+2<2,$$ so $\langle v,b\rangle=0$ and $b^2=-k$.  Thus $b$ is an effective spherical or exceptional class orthogonal to $v$.  As $v$ is minimal, $b$ must be $w_1$ or $w_2$, so the claim about the HN-filtration follows.   
\end{proof}

\begin{Lem}
Suppose that $\HH$ is non-isotropic and $\WW$ is a potential wall associated to $\HH$.  Suppose further that $v^2\geq 2$.  If there exists an effective spherical class $w$ with $\langle v,w\rangle=0$, then $\WW$ induces a divisorial contraction.  

The HN filtration of a generic element $E\in M_{\sigma_+}(v)\backslash M^s_{\sigma_0}(v)$ with respect to $\sigma_-$ is of the form $$0\to S\to E\to F\to 0\mbox{ or }0\to F\to E\to S\to 0,$$ where $S\in M^s_{\sigma_0}(w)$ and $F\in M^s_{\sigma_0}(v-w)$.
\end{Lem}
\begin{proof}
Using a Fourier-Mukai transform, we can always reduce to the case that $v$ is minimal in its orbit as in \cite[Corollary 7.3, Lemma 7.5]{BM14b}.  We assume this to be the case, and, as before, we only treat case (c).  Then the spherical class must be $s$ or $t$ (in case c(i)), and we assume it is $s$ with the other case being  dealt with similarly.  As in \cite[Lemma 7.4]{BM14b}, we first prove that $v-s$ is also minimal.

If $v^2=2$, then $(v-s)^2=0$, contrary to the assumption that $\HH$ is non-isotropic.  Thus in fact $v^2\geq 3$.  Write $v=xs+yu$, with $u$ either $t$ or $e$ in case (ci) and (cii), respectively.  Then $0=(s,v)$ gives $y=\frac{2}{m}x$.  As $\langle s,v-s\rangle=2$, to show that $v-s$ is minimal it suffices to check that $$0\leq \langle u,v-s\rangle=(mx+yu^2)-m=mx(1+\frac{2u^2}{m^2})-m=m(x(1+\frac{2u^2}{m^2})-1).$$  We now consider the cases (ci) and (cii) separately. 

First suppose we are in case (ci).  Then $u^2=-2$, $\frac{3}{2}\leq x^2(1-\frac{4}{m^2})$, and $m\geq 3$.  Suppose that $m\geq 4$.  This translates into $(1-\frac{4}{m^2})\geq\frac{3}{4}$, from which it follows that $$x^2(1-\frac{4}{m^2})^2\geq\frac{9}{8}>1,$$ so indeed $\langle u,v-s\rangle>0$.  Otherwise, $m=3$.  It is easy to show that in this case $v^2=3$ and $\langle v,s\rangle=0$ have no rational solutions, so we may assume that $v^2\geq 4$.  But then $2\leq x^2(1-\frac{4}{m^2})$, so $$x^2(1-\frac{4}{m^2})^2=x^2(1-\frac{4}{m^2})\frac{5}{9}\geq\frac{10}{9}>1,$$ and again we have $\langle u,v-s\rangle>0$.

Now suppose we are in case (cii).  Then $u^2=-1$, $\frac{3}{2}\leq x^2(1-\frac{2}{m^2})$, and $m\geq 2$.  Suppose first that $m\geq 3$.  Then $(1-\frac{2}{m^2})\geq\frac{7}{9}$ and thus $$x^2(1-\frac{2}{m^2})^2\geq x^2(1-\frac{2}{m^2})(\frac{7}{9})\geq\frac{7}{6}>1,$$ so we get $\langle u,v-s\rangle>0$.  Otherwise $m=2$ and again one can easily check that $v^2=3$ and $\langle v,s\rangle =0$ have no rational solutions, so $v^2\geq 4$, i.e. $x^2(1-\frac{2}{m^2})\geq 2$.  Thus $$x^2(1-\frac{2}{m^2})^2=x^2(1-\frac{2}{m^2})(\frac{1}{2})\geq 1,$$ so $\langle u,v-s\rangle\geq 0$.

As we have shown that $v-s$ is minimal, Lemma \ref{Lem:non-isotropic no totally semistable wall} guarantees that the generic element $F\in M_{\sigma_+}(v-s)$ is also $\sigma_0$-stable.  But then for the unique $\sigma_0$-stable spherical object $S$ with $v(S)=s$ we have $\ext^2(F,S)=\hom(S(K_X),F)=\hom(S,F)=0=\hom(F,S)$ by stability.  Thus $\ext^1(F,S)=\langle v-s,s\rangle=2$, so there is a family of extension $$0\to S\to E_p\to F\to 0,$$ parametrized by $p\in\P^1=\P(\Ext^1(F,S))$, which are all $S$-equivalent with respect to $\sigma_0$.  By \cite[Lemma 6.9]{BM14b}, they are $\sigma_+$-stable.  Thus $\pi^+$ contracts this rational curve.  Varying $F\in M_{\sigma_0}^s(v-s)$ sweeps out a family of $\sigma_+$-stable objects in $M_{\sigma_+}(v)$ of dimension $1+(v-s)^2+1=v^2=\dim M_{\sigma_+}(v)-1$.  Thus we get a divisor contracted by $\pi^+$, which must then have relative Picard rank one, so this is the only component contracted by $\pi^+$.  
\end{proof}

\begin{Lem}
Suppose that $\HH$ is non-isotropic and $\WW$ is a potential wall associated to $\HH$.  If there exists an effective exceptional class $w$ with $\langle v,w\rangle=0$, then $M_{\sigma_+}(v)\backslash M^s_{\sigma_0}(v)$ has codimension one but is not contracted by crossing $\WW$.

The HN filtration of a generic element $E\in M_{\sigma_+}(v)\backslash M^s_{\sigma_0}(v)$ with respect to $\sigma_-$ is of the form $$0\to S\to E\to F\to 0\mbox{ or }0\to F\to E\to S\to 0,$$ where $S\in M^s_{\sigma_0}(w)$ and $F\in M^s_{\sigma_0}(v-w)$.
\end{Lem}
\begin{proof}
As before, we may assume that $v$ is minimal.  Then in terms of Proposition \ref{Prop:lattice classification}, $\HH_\WW$ must fall into subcase \ref{enum:OneExceptional} of case \ref{enum:OneNegative} or either subcase \ref{enum:TwoExceptional} or \ref{enum:OneExceptionalOneSpherical} of case \ref{enum:TwoNegative}.  By minimality of $v$, there must be a $\sigma_0$-stable object of class $w$, and we denote it by $S$.  Without loss of generality, assume that $\phi^+(w)<\phi^+(v)$.  Furthermore, observe that there cannot exist any $n\in\Z$ such that $v^2=n^2$ as then $(v-nw)^2=0$, contrary to the hypothesis that $\WW$ is non-isotropic.  So, in particular, $v^2\geq 2$ and $(v-w)^2>0$.

Suppose, for now, that $v-w$ is minimal.  Then by Lemma \ref{Lem:non-isotropic no totally semistable wall} there exists a $\sigma_0$-stable object $F$ of class $v-w$.  By stability $\hom(F,S)=\ext^2(F,S)=\hom(F,S(K_X))=0$, so $\ext^1(F,S)=\langle v-w,w\rangle=1$, and there exists a unique non-trivial extension $$0\to S\to E\to F\to 0,$$ which is $\sigma_+$-stable by \cite[Lemma 6.9]{BM14b}.  By a dimension count, upon varying $F\in M_{\sigma_0}^s(v-w)$ these extensions sweep out a divisor of strictly $\sigma_0$-semistable objects which does not get contracted by $\pi^+$.  

It remains to show that $v-w$ is minimal.  As $\langle v-w,w\rangle=1>0$, this is clear in subcase \ref{enum:OneExceptional} of case \ref{enum:OneNegative}, so we can restrict ourselves to subcases \ref{enum:TwoExceptional} and \ref{enum:OneExceptionalOneSpherical} of case \ref{enum:TwoNegative}, where we can assume $w=w_1$ with $w_2$ the class of the $\sigma_0$-stable exceptional/spherical object.  Thus it remains to show that $0\leq\langle v-w_1,w_2\rangle$.  Writing $v=xw_1+yw_2$, the conditions $\langle v,w_1\rangle=0, v^2\geq 2$, and $\langle v-w_1,w_2\rangle\geq 0$ become $$y=\frac{x}{m},x^2\left(1+\frac{w_2^2}{m^2}\right)\geq 2,\mbox{ and }m\left[x\left(1+\frac{w_2^2}{m^2}\right)-1\right]\geq 0,$$ respectively.    As $m\geq 2$ in either case, we get $$\left(1+\frac{w_2^2}{m^2}\right)=\begin{cases}
1-\frac{1}{m^2}, &\text{if }w_2^2=-1,\\
1-\frac{2}{m^2},&\text{if }w_2^2=-2
\end{cases}\geq\begin{cases}
\frac{3}{4}, &\text{if }w_2^2=-1,\\
\frac{1}{2}, &\text{if }w_2^2=-2,
\end{cases}\geq\frac{1}{2},$$ in either case.
Thus $$x^2\left(1+\frac{w_2^2}{m^2}\right)^2\geq x^2\left(1+\frac{w_2^2}{m^2}\right)\left(\frac{1}{2}\right)\geq 2\left(\frac{1}{2}\right)\geq 1.$$  Taking square-roots gives that indeed $$x\left(1+\frac{w_2^2}{m^2}\right)-1\geq 0,$$ as required.
\end{proof}





\section{Isotropic walls}
\subsection{Preliminaries}
\begin{Lem}\label{Lem:isotropic lattice} Assume that there exists an isotropic class in $\HH$.  Then there are two effective, primitive, isotropic classes $w_1$ and $w_2$ in $\HH$ such that $\phi_-(w_1)>\phi_-(w_2)$.  Moreover, $P_{\HH}=\R_{\geq 0}w_1+\R_{\geq 0}w_2$ and $\langle v',w_i\rangle\geq 0$ for $i=1,2$ and any $v'\in P_{\HH}$.
\end{Lem}
\begin{proof}
If $w\in\HH$ is a primitive isotropic class, then up to replacing $w$ by $-w$, we may assume that $w_1=w$ is effective as well.  Completing $w_1$ to a basis $\HH=\Z w_1+\Z v'$, we see that $$0=(x w_1+y v')^2=2xy\langle w_1,v'\rangle+y^2 (v')^2$$ has a second integral solution since we can assume $y\neq 0$ and the signature of $\HH$ forces $\langle w_1,v'\rangle\neq 0$.  Taking the unique effective primitive class on the corresponding line, we get $w_2$.  Up to reordering, we can guarantee that $\phi_-(w_1)>\phi_-(w_2)$.  Clearly $P_{\HH}$ is as claimed, and the inequality $\langle v',w_i\rangle\geq 0$ follows accordingly.
\end{proof}

\begin{Lem}
Assume that $\rk v$ is odd.
Assume that there is an isotropic Mukai vector in $\HH$.
Then there are two isotropic and primitive Mukai vectors
$u_1,u_2 \in \HH$ and $\ell(u_1)=\ell(u_2)$.
\end{Lem}

\begin{proof}
Note that $\rk u_1$ and $\rk u_2$ are even.
There is a Mukai vector $w$ with $\HH=\Z w+\Z u_1$.
Since $\rk v$ is odd, $\rk w$ is also odd. 
Then for $u_2=x w+y u_1$ with $x,y \in \Z$ and $\gcd(x,y)=1$,
$x \rk w+y \rk u_1$ is even, which implies $x$ is even and $y$ is odd.
Then $c_1(u_2) \equiv  c_1(u_1) \mod 2$.
Hence $\ell(u_2)=\ell(u_1)$.
\end{proof}


\textcolor{red}
{From now on, let $u_1$ and $u_2$ be primitive, effective, isotropic classes in $\HH$
such that $\HH_{\Q}=\Q u_1+\Q u_2$.
Then there is $w \in \HH$ with $\HH=\Z w+\Z u_1$. }
\begin{Lem}\label{Lem:negative stable classes}
\begin{enumerate}
\item[(1)]
Assume that there is a Mukai vector $u=x w+yu_1 \in \HH$ with $u^2=-1$. 
Then
$\HH=\Z u+\Z u_1$ and
$u_2=\pm(u_1+2 \langle u_1,u \rangle u)$.
\item[(2)]
Assume that there is a Mukai vector $u=x w+yu_1 \in \HH$ with $u^2=-2$.
Then $\HH=\Z u+\Z u_1$ and
$u_2=\pm(u_1+ \langle u_1,u \rangle u)$.
\end{enumerate}
\end{Lem}

\begin{proof}
(1)
Assume that there is a Mukai vector $u=x w+yu_1 \in \HH$ with $u^2=-1$.
Then
$-1=x(x w^2+2y \langle w,u_1 \rangle)$ implies
$x=\pm 1$.  Replacing $w$ by $u$, we assume that $w^2=-1$.
Then $u_2=\pm(u_1+2 \langle u_1,w \rangle w)$.


(2) 
Assume that there is a Mukai vector $u=x w+yu_1 \in \HH$ with $u^2=-2$.
Then 
$-2=x(x w^2+2y \langle w,u_1 \rangle)$ implies
$x=\pm 1,\pm 2$.  
If $x=\pm 2$, then $2 \mid (x w^2+2y \langle w,u_1 \rangle)$,
which implies $4 \mid u^2$.
Hence $x=\pm 1$.
Replacing $w$ by $u$, we assume that $w^2=-2$.
Then we see that the claim holds.
\end{proof}

\begin{Rem}\label{Rem:negative stable classes}
\begin{enumerate}
\item
For case (1), $\rk w$ is odd. For case (2), $\rk w$ is even.
Hence (1) and (2) does not occur simultaneously.
\item
For both cases, $\ell(u_1)=\ell(u_2)$.

Indeed if $u^2=-2$ and 
if $\ell(u_i)=2$, then $2 \mid c_1(u_i)$ implies $2 \mid \langle u_i,u \rangle$
by
$\rk u \equiv \rk u_i \equiv 0 \mod 2$.
Hence $u_1 \equiv u_2 \mod 2$.
%For case (1), $\ell(u_1)=\ell(u_2)$.
\end{enumerate}
\end{Rem}

\begin{Lem}
Assume that $w^2=-1$, then
$\langle u_1,u_2 \rangle=2\langle u_2,w \rangle^2$.  If $w^2=-2$, then $\langle u_1,u_2\rangle=\langle u_2,w\rangle^2$.
\end{Lem}




\subsection{Minimal Mukai vectors}
In this section we assume that $v$ is \emph{minimal}, i.e. that $\langle v,w\rangle\geq 0$ for the (unique) spherical or exceptional effective class, if it exists.  

\textcolor{red}{
It seems that the following type of claim is proved in the minimal case: 
\begin{Prop}\label{Prop:isotropic-classification}
Asssume that $\HH_{\WW}$ is isotropic.
Assume that $v$ is minimal and $v^2>0$.
We set $r:=\rk v$.
\begin{enumerate}
\item[(1)]
If $\WW$ is totally semi-stable, that is, $\MM_{\sigma_0}^s(v,L)=\emptyset$, then
\begin{enumerate}
\item[(i)] $\langle v,u \rangle=1$ and $\ell(u)=2$, or 
\item[(ii)] $\langle v,u \rangle=\ell(u)=2$ and 
$\langle v,w \rangle=0$ for a spherical class $w$ with 
$L\equiv D+\frac{r}{2}K_X \mod 2$,
where $D$ is a nodal cycle, or 
\item[(iii)]
$\langle v,u \rangle=\ell(u)=2$ and 
$\langle v,w \rangle=0$ for an exceptional class $w$ with 
$L \equiv K_X \mod 2$, 
or
\item[(iv)] $v^2=2$, $\langle v,u \rangle=1$, $\ell(u)=1$, 
and $L \equiv D+\frac{r}{2}K_X \mod 2$, 
where $u \in \{u_1,u_2 \}$ and $D$ is a nodal cycle.
\end{enumerate}
\item[(2)]
If $\codim(M_{\sigma_+}(v,L) \setminus M_{\sigma_0}^s(v,L))=1$, then 
\begin{enumerate}
\item[(i)] $\langle v,u \rangle=\ell(u)=2$, 
\item[(ii)] $\langle v,w \rangle=0$
for a spherical class $w$ 
and $v$ does not satisfy (1) (ii)
\item[(iii)] 
$\langle v,w \rangle=0$
for an exceptional class $w$
and $v$ does not satisfy (1) (iii),
\item[(iv)]
$\langle v,u \rangle=\ell(u)=1$ and $v$ does not satisfy (1) (iv), 
where $u \in \{u_1,u_2 \}$.  
\end{enumerate}
\end{enumerate}
\end{Prop}
\begin{Rem}
\begin{enumerate}
\item[(1)]
For case (1) (ii), then
$v=w+u$ and $v^2=2$.
If $L \equiv D+\frac{r}{2}K_X+K_X \mod 2$, then $\WW=\emptyset$.
For case (1) (iii),
$v=2(w+u)$. If $L \equiv K_X \mod 2$, then
a connected component of $\MM_{\sigma_+}(v)$ has two irreducible
components (Proposition \ref{prop:connected}) and $\WW$ is one of them.
If $L \equiv 0 \mod 2$, then 
$\WW$ is a divisor.
For case (1) (iv),
we also see that
$v=w+2u$. By Proposition \ref{prop:irred-comp:v^2=2},
$M_\sigma (v)$ has two irreducible components and
each components becomes totally semi-stable at walls
$\WW$ (of type (1) (iv)) and $\WW'$ (of type (1) (ii)).
\item[(2)]
For case (2) (iv),
assume that $u=u_2$. Then
$v=u_1+\frac{v^2}{2} u_2$ with $\langle u_1,u_2 \rangle=1$ or
$v=w+\frac{v^2+1}{2}u_2$ with an effective exceptional class $w$.
In particular $\ell(u_1)=\ell(u_2)=1$.
\end{enumerate}
\end{Rem}
}

\subsubsection{$\ell(u_i)=2$ for some $i$}
By the above work, we see that we may assume that $i=2$, i.e. $M_{\sigma_0}^s(u_2)=M_{\sigma_0}(u_2)$, so $\ell(u_2)=2$ implies that $M_{\sigma_0}(u_2)\cong X$.  Using a Fourier-Mukai transform $$\Phi:\Db(X)\cong\Db(X),$$ we get $\Phi(u_2)=(0,0,1)$.  By this construction, skyscraper sheaves of points on $X$ are $\Phi(\sigma_0)$-stable{\color{red} Corrected 10/13}. By Bridgeland's Theorem 2.9, there exist divisor classes $\omega,\beta\in\NS(X)_{\Q}$, with $\omega$ ample, such that up to the $\GL_2(\R)$-action, $\Phi(\sigma_0)=\sigma_{\omega,\beta}$. In particular, the category $\PP_{\omega,\beta}(1)$ is the extension-closure of skyscraper sheaves of points, and the shifts $F[1]$ of $\mu_{\omega}$-stable torsion-free sheaves $F$ with slope $\mu_{\omega}(F) =\omega\cdot\beta$. Since $\sigma_0$ by assumption does not lie on any other wall with respect to $v$, the divisor $\omega$ is generic with respect to  $\Phi(v)$. Under these identifications, we have the following result whose proof is identical to that of \cite[Theorem 1.2]{MYY14} and \cite[Proposition 8.2]{BM14b}.

\begin{Prop}\label{Prop:Uhlenbeck morphism}
An object $E$ of class $v$ is $\sigma_+$-stable if and only if $\Phi(E)$ is the shift $F[1]$ of a $(\beta,\omega)$-Gieseker stable sheaf $F$ on $X$; the shift $[1]$ induces the following identification of moduli spaces: $$M_{\sigma_+}(v) = M_{\omega}^{\beta}(-\Phi(v)).$$
Moreover, the contraction morphism $\pi^+$ induced by the wall $\WW$ is the Li-Gieseker-Uhlenbeck (LGU) morphism to the Uhlenbeck compactification.
\end{Prop}

\begin{Prop}\label{Prop:LGU walls of low codimension}
Assume that $\HH_{\WW}$ is isotropic.  Suppose that $v^2>0$ and $\langle v,w\rangle\geq 0$ for any effective spherical or exceptional class $w$.  If $\WW$ is totally semistable (for some component), then $\langle v,u_2\rangle=1$ or $\langle v,u_2\rangle=2$ and $\langle v,w\rangle=0$ for \textcolor{red}{a spherical or} an exceptional class $w$.  Similarly, if $\codim(M_{\sigma_+}(v)\backslash M^s_{\sigma_0}(v))=1$, then either $\langle v,u_2\rangle=2$, $\langle v,u_1\rangle=2=\ell(u_1)$, $\langle v,w\rangle=0$ for a spherical {\color{red} in this case $\langle v,u_2 \rangle \ne 2$?} or exceptional class, or $\langle v,u_2\rangle=2$ and $\langle v,w\rangle=1$ for an exceptional class $w$.  In all other cases, the locus of strictly $\sigma_0$-semistable objects has codimension at least two.

%\todo{ How about the case where 
%$v=-x(\langle u_2,w \rangle w+u_2), x \in \Z_{>0}, w^2=-1$? 
%It seems the locus is of codimension 1.}
\textcolor{red}{The conditions are slightly confusing:
Is it true that
"then either (i) $\langle v,u_2\rangle=2$, (ii) $\langle v,u_1\rangle=2=\ell(u_1)$, (iii) $\langle v,w\rangle=0$ for a spherical or exceptional class, or (iv) $\langle v,u_2\rangle=2$ and $\langle v,w\rangle=1$ for an exceptional class $w$."?
}
\end{Prop}
\begin{proof}
By assumption, $\WW$ is not a wall for $u_2$, which satisfies $\ell(u_2)=2$.  Using the Fourier-Mukai transform $\Phi$, we may assume by Proposition \ref{Prop:Uhlenbeck morphism} that $M_{\sigma_+}(v)=M_{\omega}^{\beta}(-\Phi(v))$ and that $\WW$ induces the LGU morphism, so in particular $M^s_{\sigma_0}(v)$ corresponds to the open locus of $\mu$-stable locally free sheaves.  Then the result follows from the proofs of \cite[Proposition 2.6, Lemma 2.8, Proposition 2.11]{Yos16a}, where three cases are considered based on whether $\HH_{\WW}$ contains an effective exceptional class (Case B), a spherical class (Case C), or no such classes (Case A).  

For the sake of completeness, we indicate how the result follows from the dimension estimates in \cite{Yos16a}.  As $\rk(-\Phi(v))=\langle v,u_2\rangle$, it is clear that if $\langle v,u_2\rangle=1$, then $M_{\sigma_+}(v)$ is isomorphic to the Hilbert scheme of points and $\WW$ is a totally semistable wall inducing the Hilbert-Chow morphism, so we may assume that $\rk(-\Phi(v))=\langle v,u_2\rangle\geq 2$.  But then from \cite[Proposition 2.6]{Yos16a} (resp. \cite[Proposition 2.11]{Yos16a}) it follows that $\WW$ is not totally semistable in  Case A (resp. Case C \textcolor{red}{unless $\langle v,w \rangle=0$ and $\langle v,u_2 \rangle = 2$}) as there exist $\mu$-stable locally free sheaves in every component of $M_{\omega}^{\beta}(-\Phi(v))$.  \textcolor{red}{In Case C, if $\langle v,w \rangle=0$ and $\langle v,u_2 \rangle=2$, then
$\WW$ is totally semi-stable for an irreducible component. We shall prove there are two irreducible components in Proposition \ref{prop:irred-comp:v^2=2}.}
In Case A, the estimates \cite[(2.23-25)]{Yos16a} show that the strictly $\mu$-semistable locus can only have codimension one if $\langle v,u_1\rangle=2$ and $\ell(u_1)=2$, while the estimates \cite[(2.26,2.27)]{Yos16a} show that the non-locally free locus can only have codimension one if $\rk(-\Phi(v))=2$, i.e. $\langle v,u_2\rangle=2$.  Similarly, in Case C the estimate \cite[(2.39)]{Yos16a} shows that the strictly $\mu$-semistable locus can only have codimension one if 
$\langle v,w\rangle=0$ \textcolor{red}{and $\langle v,u_2 \rangle>2$} the unique effective spherical class $w\in\HH_{\WW}$, and the estimate \cite[(2.26) with $\delta=0$]{Yos16a} shows that the non-locally free locus can only have codimension one if again $2=\rk(-\Phi(v))=\langle v,u_2\rangle$.

It remains to consider Case B when there exists an effective exceptional class $w$.  In this case \cite[Lemma 2.8 (2)]{Yos16a} shows that $\WW$ is not totally semistable unless $\langle v,w\rangle=0$ and $\langle v,u_2\rangle=\rk(-\Phi(v))=2$, as the result there guarantees the existence of a $\mu$-stable locally free sheaf in every component.  Moreover, if $\langle v,u_2\rangle=\rk(-\Phi(v))>2$, then the locus of strictly $\mu$-semistable sheaves can only have codimension one if $\langle v,w\rangle=0$ by \cite[Lemma 2.8 (1)]{Yos16a} while the locus of non-locally free sheaves has codimension greater than one by \cite[(2.26) with $\delta=0$]{Yos16a}.  If instead $\langle v,u_2\rangle=\rk(-\Phi(v))=2$, then $\codim(M_{\sigma_+}(v)\backslash M^s_{\sigma_0}(v))=1$ only if $\langle v,w\rangle=0,1$.  
\end{proof}

\begin{Rem}
Proposition \ref{Prop:LGU walls of low codimension} can be proven similarly to the other results in this paper, but the above proof seemed more efficient and more instructive.
\end{Rem}

\begin{Rem}
In Case B, the reader may notice that when $\langle v,w\rangle=0$ and $\langle v,u_2\rangle=2$, we simultaneously claim that $\codim(M_{\sigma_+}(v)\backslash M^s_{\sigma_0}(v))=0$ and $1$ .  Indeed, in that case $-\Phi(v)=(2,0,-1)e^{\xi},-\Phi(w)=(1,0,\frac{1}{2})e^{\xi}$, so $v$ is divisible by 2, and $M_{\sigma_+}(v)$ is reducible.  On at least one component, all objects are strictly $\sigma_0$-semistable, while on a different component the strictly $\sigma_0$-semistable locus forms a divisor (see \cite[Lemma 2.8 (3)]{Yos16a}). 
\end{Rem}
 Now we demonstrate the converse of \ref{Prop:LGU walls of low codimension} in the following sequence of lemmas.
\begin{Lem}\label{Lem: Hilbert-Chow}
Let $\HH_{\WW}$ be isotropic.  Suppose that $\langle v,u_2\rangle=1$.  Then $v^2$ is odd and $\HH_{\WW}$ contains an exceptional class.  Moreover, $\WW$ is totally semistable and, if $v^2>1$, induces a divisorial contraction.  
\end{Lem}
\begin{proof}
If $\langle v,u_2\rangle=1$, then $1=\langle v,u_2\rangle=\rk(-\Phi(v))\equiv v^2\pmod 2$, so $v^2$ is odd and $M_{\omega}^{\beta}(-\Phi(v))$ is isomorphic to the Hilbert scheme of points.  Thus $v-\frac{v^2+1}{2}u_2\in\HH_\WW$ is an exceptional class.  Finally, $\WW$ is the Hilbert-Chow wall inducing the Hilbert-Chow morphism $\Hilb^n(X)\to\Sym^n(X)$, which is a divisorial contraction for $1<n=\frac{v^2+1}{2}$, and every ideal sheaf is strictly semistable as in \cite[Proposition 13.1]{Nue14b}.
\end{proof}





\begin{Lem}\label{Lem:P1FibrationExceptional}
Let $\HH_{\WW}$ be isotropic, and suppose that $\langle v,u_2\rangle=2$ and $\langle v,w\rangle=0$ for an exceptional class $w\in\HH_{\WW}$.  Then $\WW$ is totally semistable for, and induces a $\P^1$-fibration on, precisely one component of $M_{\sigma_+}(v,2L'+K_X)$.
\end{Lem}
\begin{proof}
Since $\langle v,u_2\rangle=2$, after applying $[-1]\circ\Phi$ and possibly tensoring by a line bundle, we may assume $v=(2,0,-1)$ and $w=(1,0,\frac{1}{2})$ and $M_{\sigma_+}(v)$ is isomorphic to $M_H(v)$, the moduli space of Gieseker semistable sheaves of Mukai vector $v$ with respect to a generic polarization $H$, and $\pi^+$ is the LGU-contraction morphism as in Proposition \ref{Prop:Uhlenbeck morphism}.  Then as mentioned in \cite[Remark 2.3]{Yos16a} and proven in Section \ref{App: exceptional case}, there is precisely one component of $M_{\sigma_+}(v,K_X)$ consisting of stable non-locally free sheaves $E$ fitting into the short exact sequence $$0\to E\to F\mor[(\phi_1,\phi_2)]\C_p\oplus\C_q\to 0,$$ where $F:=\OO_X\oplus\OO_X(K_X)$ and  $\phi_1:\OO_X\to\C_p\oplus\C_q$ and $\phi_2:\OO_X(K_X)\to\C_p\oplus\C_q$ are both surjective.  The polystable object in the same S-equivalence class as $E$ with respect to $\sigma_0$ is $(\C_p\oplus\C_q)[-1]\oplus F$, and the set of distinct $\sigma_+$ stable objects in the same S-equivalence class is parametrized by $\P\Hom(F,\C_p)\times\P\Hom(F,\C_q)/(\Aut(F)/\C^*)$, which is a curve birational to $\P^1$.  Thus $\WW$ is totally semistable for this component and induces a $\P^1$-fibration.  
\end{proof}

\textcolor{red}{
Lemma{Lem:P1FibrationSpherical}: 
Let $\HH_{\WW}$ be isotropic, and suppose that $\langle v,u_2\rangle=2$ and $\langle v,w\rangle=0$ for a spherical class $w\in\HH_{\WW}$.  
Then $\WW$ is totally semistable for, and induces a $\P^1$-fibration on, precisely one component of $M_{\sigma_+}(v,L)$ with $L \equiv D+\frac{\rk v}{2}K_X \mod 2$,
where $D$ is a nodal cycle. Moreover
$M_{\sigma_+}(v,L+K_X)=M_{\sigma_0}^s(v,L+K_X)$.
Proof:
As in the proof of Lemma \ref{Lem:P1FibrationExceptional},
we may assume that $v=(2,D+K_X,a)$ and $M_{\sigma_+}(v)$ is isomorphic to
$M_H(v)$. Let $S$ be a stable vector bundle with $v(S)=(2,D+K_X,a;1)$.
Then we have a family of non-locally free sheaves parameterized by $\P^1$-bundle over $X$: 
$$
0 \to E \to S \to \C_p \to 0.
$$
Thus this $\P^1$-bundle is $\WW$.
For $v=(2,D,a)$, all stable sheaves are $\mu$-stable locally free sheaves.
Hence $M_{\sigma_+}(v,L+K_X)=M_{\sigma_0}^s(v,L+K_X)$.
}

\begin{Lem}\label{Lem: isotropic divisorial l=2 1}
Suppose that $\HH_{\WW}$ is isotropic and $\langle v,u_2\rangle=2$.  Assume further that $v^2\geq 4$ if $\HH_{\WW}$ contains no effective spherical or exceptional classes.  Then $\WW$ is not a totally semistable wall and induces a divisorial contraction on $M_{\sigma_+}(v,L)$ except if $v^2=4$ and $\HH_{\WW}$ contains an exceptional class \textcolor{red}{ or if $v^2=2$ and $\HH_W$
contains a spherical class}, in which case for at least one component of each $M_{\sigma_+}(v,L)$, $\WW$ is not totally semistable and the divisorial component of $M_{\sigma_+}(v,L)\backslash M^s_{\sigma_0}(v,L)$ is not contracted in its entirety.
\end{Lem}
\begin{proof}
Using Proposition \ref{Prop:Uhlenbeck morphism}, we see that $\Phi(u_2)=(0,0,1)$ and that $\WW$ induces the LGU morphism from $M_{\sigma_+}(v)=M_{\omega}^{\beta}(-\Phi(v))$ to its Uhlenbeck compactification.  But then $\langle v,u_2\rangle=2$ implies that $\Phi(v)=-(2,D,s)$, where $s\in\Z$ and after possibly tensoring with a line bundle we may assume that either $D=0$ or $D$ is not divisible by 2.  

First suppose that $D$ is a non-zero divisor not divisibile by 2.  But then we are in Cases A or C of \cite[Theorem 2.1]{Yos16a}, with $\gcd(2,D)=1$, so every component of $M_{\omega}^{\beta}(-\Phi(v))$ contains a $\mu$-stable locally free sheaf and \textcolor{red}{$v^2> 2$} in Case C.  Thus $\WW$ is not totally semistable, and because $\gcd(2,D)=1$, there cannot be any properly $\mu$-semistable sheaves contracted by the LGU morphism. 

If $D=0$, then $(2,0,1)=-\Phi(v)-(s-1)\Phi(u_2)\in\Phi(\HH_{\WW})$, and by the primitivity of $\HH$, we must have $-\Phi(w)=(1,0,\frac{1}{2})\in\Phi(\HH)$, where $w^2=-1$, so we are in Case B of \cite[Theorem 2.1]{Yos16a}.  If $v^2>4$, then $\WW$ is not totally semistable by \cite[Lemma 2.8 (2)]{Yos16a} since every component contains a $\mu$-stable locally free sheaf, and moreover, $v^2=-4s$, so $v^2>4$ implies $v^2\geq 8$, while $v^2=4$ implies $v=2w+2u_2$.

Writing $v'=v-u_2$, by assumption and the preceeding paragraphs, we get that $v'^2\geq 0$ in Case A, \textcolor{red}{$v'^2> -2$} in Case C, and $v'^2\geq 4$ in Case B.  Then $\Phi(v')=-(2,D,s+1)$, so invoking \cite[Theorem 2.1]{Yos16a} we get that generically $M_{\omega}^{\beta}(-\Phi(v'))$ consists of $\mu$-stable locally free sheaves (in Case B and $v'^2=4$, at least one component of each $M_{\omega}^{\beta}(-\Phi(v'),L)$ contains $\mu$-stable locally free sheaves).  Taking such a locally free $\mu$-stable sheaf $F\in M_{\omega}^{\beta}(-\Phi(v'))$ and a point $x\in X$, $\Hom(F,\C(x))=2$, so the surjections $F\onto\C(x)$ give a $\P^1$ of extensions contracted by the LGU morphism.  A quick dimension count shows that these sweep out a divisor.

\textcolor{red}{When $v^2=2$, from Section \ref{App: exceptional case}, there is a component of $M_{\omega}^{\beta}(-\Phi(v),L)$ containing $\mu$-stable locally free sheaves, where $L \equiv C+K_X \mod 2$ for a nodal cycle.  We prove further in Section \ref{App: exceptional case} that the complement of the open locus of $\mu$-stable locally free sheaves is a divisor.
For this component, the morphism to the Uhlenbeck compactification is not a divisorial contraction, as claimed.
For $M_{\omega}^{\beta}(-\Phi(v),L+K_X)$, $\WW=\emptyset$.
}

When $v^2=4$, from Section \ref{App: exceptional case} the proof of \cite[Lemma 2.8 (3)]{Yos16a}, there is a component of each $M_{\omega}^{\beta}(-\Phi(v),L)$ containing $\mu$-stable locally free sheaves, where $L=\OO_X,\OO_X(K_X)$.  It is not difficult to see that if a sheaf $E\in M_{\omega}^{\beta}(-\Phi(v),L)$ is $\mu$-stable, then it must be locally free (this follows directly from \cite[Lemma 2.7]{Yos16a}).  We prove further in Section \ref{App: exceptional case} that the complement of the open locus of $\mu$-stable locally free sheaves is a divisor (plus an additional component if $L=\OO_X(K_X)$ as described in Lemma \ref{Lem:P1FibrationExceptional}) which does not contracted by the morphism to the Uhlenbeck compactification, as claimed.
\end{proof}

\begin{Lem}\label{Lem: isotropic divisorial l=2 2}
Suppose that $\HH_{\WW}$ contains a primitive isotropic vector with $\ell=2$.  Assume that $\langle v,w\rangle=0$ for an effective spherical class $w$. 
Then $\WW$ induces a divisorial contraction unless $\langle v,u_2 \rangle=2$.
\end{Lem}
\begin{proof}
Let $S$ be the unique $\sigma_0$-stable spherical object of class $w$.  Consider $a:=v-w$.  Then $a^2=v^2-2$ and $\langle a,w\rangle=2$.  By Lemma \ref{Lem:negative stable classes}, $\HH_\WW=\Z u_2+\Z w$ which forces $v^2$ to be even.  
\textcolor{red}{Since $\ell(u_2)=2$, $\langle w,u_2 \rangle$ is even. We note that 
$\langle v,u_2 \rangle=2$ if and only if $v^2=2$.
Indeed if we set $v=xw+yu_2$, then $\langle v,w \rangle=0$ implies $x=y \langle w,u_2 \rangle/2$
and $v^2=2x^2$. Hence $x=1$ implies $y=1$ and $\langle w,u_2 \rangle=2$, and hence $\langle v,u_2 \rangle=2$. Conversely $\langle v,u_2 \rangle=2$ implies $x=1$ and $\langle w,u_2 \rangle=2$.}
Thus $v^2>0$ implies \textcolor{red}{$v^2> 2$, so $a^2> 0$.}
By Proposition \ref{Prop:LGU walls of low codimension}, $M_{\sigma_0}^s(a)\neq\varnothing$ if $a^2>0$ and $\langle a,u_2\rangle\geq 2$, so letting $A$ vary in $M^s_{\sigma_0}(a)$, we see that the $\P^1$'s of $S$-equivalent extensions $$0\to S\to E\to A\to 0$$ sweep out a contracted divisor in $M_{\sigma_+}(v)$.
\textcolor{red}{
I think the following is not needed:
If instead $a^2>0$ but $\langle a,u_2\rangle=1$, then it is easy to see that $w=a+ku_2$ for some $k\in \Z$.  Thus $\langle w,u_2\rangle=\langle a,u_2\rangle=1$, so $\langle v,u_2\rangle=2$, so by Lemma \ref{Lem: isotropic divisorial l=2 1} $\WW$ again induces a divisorial contraction.  
Finally, assume that $a^2=0$.  Then $\langle a,v\rangle=\langle a,w\rangle=-w^2$.  Since $a\in C_{\WW}$ and $\langle a,w\rangle>0$, we must have $a=ku_2$, so $v=w+ku_2$.  But then $k\langle v,u_2\rangle=\langle v,a\rangle=-w^2$.  It follows that $\langle v,u_2\rangle=1,2$.  
If $\langle v,u_2\rangle=1$, then $\WW$ induces a (totally semistable) divisorial contraction by Lemma \ref{Lem: Hilbert-Chow}, and if $\langle v,u_2\rangle=2$, then again $\WW$ induces a divisorial contraction by Lemma \ref{Lem: isotropic divisorial l=2 1}.}
\end{proof}
\begin{Rem}
If $w$ in Lemma \ref{Lem: isotropic divisorial l=2 2} is instead an effective exceptional class, then the same argument produces a divisor of strictly $\sigma_0$-semistable objects in $M_{\sigma_+}(v)$ which is not contracted. 
\end{Rem}
\begin{Lem}\label{Lem: isotropic divisorial l=2 3}
Let $\WW$ be a potential wall and $v$ minimal in its orbit.  If there exists a primitive isotropic class $u$ with $\ell(u)=2$ such that $\langle v,u\rangle\in\{1,2\}$, $v^2\geq 4$ in case $\HH_{\WW}$ does not contain effective spherical or exceptional classes, and $v^2 > 4$ \textcolor{red}{or $v^2=3$?} in case $\HH_{\WW}$ contains an exceptional class, then $\WW$ induces a divisorial contraction.
\end{Lem}
\begin{proof}
The class $u$ is automatically effective.  
By {\color{red} Lemma \ref{Lem: Hilbert-Chow} and?} Lemma \ref{Lem: isotropic divisorial l=2 1}, the only remaining case is $u=u_1$.  

First, suppose that $\HH_{\WW}$ admits an effective spherical or exceptional class $w$.  Then $\langle v,w\rangle\geq 0$.  As $u_1=u_2+\langle u_2,w\rangle w$ if $w$ is spherical and $u_2+2\langle u_2,w\rangle w$ if $w$ is exceptional, we see that $0<\langle v,u_2\rangle\leq\langle v,u_1\rangle$.  Thus $\langle u_1,v\rangle\in\{1,2\}$ forces $\langle v,u_2\rangle\in\{1,2\}$, so the result follows from \textcolor{red}{Lemma \ref{Lem: Hilbert-Chow} and } Lemma \ref{Lem: isotropic divisorial l=2 1} again.

Now suppose that $\HH_{\WW}$ admits no effective spherical or exceptional classes, so $$M_{\sigma_0}^s(u_1)=M_{\sigma_0}(u_1)\cong X.$$  Furthermore, note that $\langle v,u_1\rangle=1$ with $\ell(u_1)=2$ cannot occur in this case.  Indeed, if $v^2$ is odd, then $v-\frac{v^2+1}{2}u_1$ is an exceptional class, contrary to our assumption on $\HH_\WW$, while if $v^2$ is even, then so is $\langle v,u_1\rangle$ since $\ell(u_1)=2$.  Thus $\langle v,u_1\rangle=2$.  Applying Lemma \ref{Prop:Uhlenbeck morphism} for the FM transform $\Psi$ identifying $M_{\sigma_0}(u_1)$ and $X$, 
we see that 
Lemma \ref{Lem: isotropic divisorial l=2 1} applies with $u_1$ instead, giving the result.
\end{proof}

\subsubsection{$\ell(u_1)=\ell(u_2)=1$} In this case we must use a different technique than above as we do not have at our disposal the Fourier-Mukai transforms coming from isotropic vectors with $\ell=2$.  So we take a direct dimension counting approach.  

Let the Harder-Narasimhan filtration of $E$ with respect to $\sigma_-$ correspond to a decomposition $v=\sum_i a_i$.  We shall estimate the codimension of the sublocus $\FF(a_0,\ldots,a_n)^0$ of destabilized objects which is equal to
 \begin{equation}
\sum_i (a_i^2-\dim \MM_{\sigma_-}(a_i))+\sum_{i<j}\langle a_i,a_j \rangle.
\end{equation}


(I) We first assume that one of the $a_i$ satisfies $a_i^2<0$, say $a_0=b_0 w$ for an effective spherical or exceptional class $w$.  

Assume that $a_1$ and $a_2$ are isotropic.
We may set $a_1=b_1 u_1$ and $a_2=b_2 u_2$.
Then 
 \begin{equation}\label{eq: 1,1 case I}
\begin{split}
& \sum_i (a_i^2-\dim \MM_{\sigma_-}(a_i))+\sum_{i<j}\langle a_i,a_j \rangle\\
\geq & (a_0^2-\dim \MM_{\sigma_-}(a_0))+\sum_{i \geq 1}b_0 \langle w,a_i \rangle
-\left\lfloor\frac{b_1}{2}\right\rfloor-\left\lfloor\frac{b_2}{2}\right\rfloor+b_1 b_2 \langle u_1,u_2 \rangle\\
\geq & -\dim\MM_{\sigma_-}(a_0)+b_0 \langle w,v \rangle-\left\lfloor\frac{b_1}{2}\right\rfloor-\left\lfloor\frac{b_2}{2}\right\rfloor+b_1 b_2 \langle u_1,u_2 \rangle\\
\geq & -\dim\MM_{\sigma_-}(a_0)-\left\lfloor\frac{b_1}{2}\right\rfloor-\left\lfloor\frac{b_2}{2}\right\rfloor+b_1 b_2 \langle u_1,u_2 \rangle.
\end{split}
\end{equation} 

First suppose that $w^2=-2$, so $\dim\MM_{\sigma_-}(a_0)=-b_0^2$ and thus 
\begin{equation}\label{eq: 1,1 case I spherical}
\begin{split}
\codim\FF(a_0,\ldots,a_n)^0 &\geq b_0^2+b_1 b_2-\frac{b_1}{2}-\frac{b_2}{2}\geq 1+b_1 b_2-\frac{b_1}{2}-\frac{b_2}{2}\\
&=1+\frac{b_2(b_1-1)+b_1(b_2-1)}{2}\geq 1
\end{split}
\end{equation}
and equality holds in the last inequality only if $\langle v, w\rangle=0,\langle u_1,u_2\rangle=1$, $v=w+u_1+u_2$.  But then $\langle v,w\rangle=-2<0$, contrary to assumption.  So we must have $\codim\FF(a_0,...,a_n)^0\geq 2$ in this case.

If instead $w^2=-1$, then $\dim\MM_{\sigma_-}(a_0)=\left\lfloor-\frac{b_0^2}{2}\right\rfloor$ and $2\mid\langle u_1,u_2\rangle$, so
\begin{equation}\label{eq: 1,1 case I exceptional}
\begin{split}
\codim\FF(a_0,\ldots,a_n)^0 &\geq \frac{b_0^2}{2}+2b_1b_2-\frac{b_1}{2}-\frac{b_2}{2}\geq\frac{1}{2}+2b_1b_2-\frac{b_1}{2}-\frac{b_2}{2}\\
&=\frac{1}{2}+\frac{b_1(2b_2-1)+b_2(2b_1-1)}{2}\geq\frac{3}{2},
\end{split}
\end{equation}
so $\codim\FF(a_0,\ldots,a_n)^0\geq 2$ in this case.

Now assume that $a_1=b_1 u_j$ and $a_i^2>0$ for $i>1$.  Then 
\begin{equation}
\begin{split}
 \sum_i (a_i^2-\dim \MM_{\sigma_-}(a_i))+\sum_{i<j}\langle a_i,a_j \rangle
&\geq  a_0^2-\dim\MM_{\sigma_-}(a_0)+\sum_{i \geq 1}b_0 \langle w,a_i \rangle
-\left\lfloor\frac{b_1}{2}\right\rfloor+\sum_{i \geq 2} b_1 \langle u_j,a_i \rangle\\
&\geq -\dim\MM_{\sigma_-}(a_0)+b_0 \langle w,v \rangle
-\left\lfloor\frac{b_1}{2}\right\rfloor+b_1 \langle u_j,a_2 \rangle\\ 
&\geq -\dim\MM_{\sigma_-}(a_0)+\frac{b_1}{2}.
\end{split}
\end{equation}

If $w^2=-2$ then $$-\dim\MM_{\sigma_-}(a_0)+\frac{b_1}{2}=b_0^2+\frac{b_1}{2}\geq\frac{3}{2}.$$  If $w^2=-1$, then $$\codim\FF(a_0,\ldots,a_n)^0\geq-\dim\MM_{\sigma_-}(a_0)+\frac{b_1}{2}\geq \frac{b_0^2}{2}+\frac{b_1}{2}\geq 1,$$ with equality in the last inequality only if $b_1=b_0=1$, in which case the first inequality is strict.  So we always have $\codim\FF(a_0,a_1,...,a_n)^0\geq 2$ in this case.

We can now assume that there are no positive classes in the Harder-Narasimhan factors, i.e. $v=b_0 w+b_1 u_j$.  But $v^2>0$ forces $j=2$, so we may assume this outright.  Then $0 \leq \langle v,w \rangle=b_0 w^2+b_1 \langle u_2,w \rangle$, so our estimate becomes 

 \begin{equation}\label{eq: 1,1 case I no positive}
\begin{split}
\codim\FF(a_0,a_1)^0=&\sum_i (a_i^2-\dim \MM_{\sigma_-}(a_i))+\sum_{i<j}\langle a_i,a_j \rangle\\
= & b_0^2 w^2-\dim\MM_{\sigma_-}(a_0)-\left\lfloor\frac{b_1}{2}\right\rfloor+b_0b_1\langle w,u_2\rangle\\
\geq & b_0^2 w^2-\dim\MM_{\sigma_-}(a_0)+\frac{b_0b_1\langle w,u_2\rangle}{2}+\frac{b_1}{2}\left(b_0\langle w,u_2\rangle-1\right).
\end{split}
\end{equation}

If $w^2=-2$, then $b_0^2w^2-\dim\MM_{\sigma_-}(a_0)=-b_0^2$, so the last line of \eqref{eq: 1,1 case I no positive} becomes $$-b_0^2+\frac{b_0 b_1\langle w,u_2\rangle}{2}+\frac{b_1}{2}(b_0\langle w,u_2\rangle-1)=\frac{b_0}{2}\langle v,w\rangle+\frac{b_1}{2}(\langle v,u_2\rangle-1)>0$$
unless $\langle v,w\rangle=0$ and $\langle v,u_2\rangle=1$.  But then $v=w+2u_2$ and $\langle w,u_2\rangle=1$, in which case indeed $\codim\FF(a_0,a_1)^0=0$.  Moreover, $\codim \FF(a_0,a_1)^0\geq 2$ unless $\langle w,u_2\rangle=2$ and $v=w+u_2$ or $\langle w,u_2\rangle=1$ and $v=w+3u_2$ or $w+4u_2$, where in each case $\codim\FF(a_0,a_1)^0=1$.

If $w^2=-1$, then $b_0^2w^2-\dim\MM_{\sigma_-}(a_0)=\left\lceil-\frac{b_0^2}{2}\right\rceil$, so the last line of \eqref{eq: 1,1 case I no positive} gives $$\codim\FF(a_0,a_1)^0\geq-\frac{b_0^2}{2}+\frac{b_0 b_1\langle w,u_2\rangle}{2}+\frac{b_1}{2}(b_0\langle w,u_2\rangle-1)=\frac{b_0}{2}\langle v,w\rangle+\frac{b_1}{2}(\langle v,u_2\rangle-1)>0$$
unless again $\langle v,w\rangle=0$ and $\langle v,u_2\rangle=1$, in which case $v=w+u_2$ and $\langle w,u_2\rangle=1$.  In this case however $\codim \FF(a_0,a_1)^0=1$.  Moreover, $\codim \FF(a_0,a_1)^0\geq 2$ unless $\langle w,u_2\rangle=1$ and $v=w+2u_2,2(w+u_2)$, in which case $\codim\FF(a_0,a_1)^0=1$.  

Finally, assume that other than $a_0=b_0 w$, $a_i^2>0$ for all $i>0$.  Then the estimate becomes 

\begin{equation}\label{eq: 1,1 case I no isotropic}
\begin{split}
\codim\FF(a_0,\ldots,a_n)^0=&\sum_i (a_i^2-\dim \MM_{\sigma_-}(a_i))+\sum_{i<j}\langle a_i,a_j \rangle\\
= & -\dim\MM_{\sigma_-}(a_0)+b_0\langle w,v\rangle+\sum_{0<i<j}\langle a_i,a_j\rangle\geq \frac{b_0^2}{2}>0.
\end{split}
\end{equation}
Moreover, $\codim\FF(a_0,\ldots,a_n)^0\geq 2$ unless $\langle w,v\rangle=0$ and $v=w+a_1$, in which case $\codim\FF(a_0,a_1)^0=1$.  Note that in this case we require $0<a_1^2=(v-w)^2=v^2+w^2$, so $v^2>2$ or $v^2>1$ if $w^2=-2$ or $w^2=-1$, respectively.

(II) We next assume that $a_i^2 \geq 0$ for all $i$.
 

We assume $a_1=b_1 u_1$ and $a_2=b_2 u_2$.
Then 
 \begin{equation}\label{eq: spherical 1,1 case II,a}
\begin{split}
\codim\FF(a_1,...,a_n)^0=& \sum_i (a_i^2-\dim \MM_{\sigma_-}(a_i))+\sum_{i<j}\langle a_i,a_j \rangle\\
\geq &
-\left\lfloor\frac{b_1}{2}\right\rfloor-\left\lfloor\frac{b_2}{2}\right\rfloor+b_1 b_2 \langle u_1,u_2 \rangle\\
\geq &\frac{b_1(b_2-1)+b_2(b_1-1)}{2}>0,
\end{split}
\end{equation}
unless $v=u_1+u_2$ and $\langle u_1,u_2\rangle=1$, in which case $\codim\FF(a_1,a_2)^0=1$.  If, say, $b_1=1$ and $b_2\geq 2$, then we have $$\codim\FF(a_1,a_2)^0\geq-\left\lfloor\frac{1}{2}\right\rfloor-\left\lfloor\frac{b_2}{2}\right\rfloor+b_2\langle u_1,u_2\rangle\geq \frac{b_2}{2}\geq 1,$$ with equality only if $b_2=2$ and $\langle u_1,u_2\rangle=1$.  Thus $\codim\FF(a_1,...,a_n)^0$ is always positive in this case, is equal to 1 only if $v=u_1+u_2,2u_1+u_2,u_1+2u_2$ with $\langle u_1,u_2\rangle=1$, and otherwise $\codim\FF(a_1,...,a_n)^0\geq 2$.


Now we assume that $a_1=b_1 u_j$ and $a_i^2>0$ for $i \geq 2$.
In this case, we also see that
\begin{equation}
\begin{split}
\codim\FF(a_1,...,a_n)^0=&\sum_i (a_i^2-\dim \MM_{\sigma_-}(a_i))+\sum_{i<j}\langle a_i,a_j \rangle\\
=&-\left\lfloor\frac{b_1}{2}\right\rfloor+\sum_{i>1}b_1\langle u_j,a_i\rangle+\sum_{1<i<k}\langle a_i,a_k\rangle\\
\geq&b_1(\langle v,u_j\rangle-\frac{1}{2})+\sum_{1<i<k}\langle a_i,a_k\rangle\\
\geq&b_1(\langle v,u_j\rangle-\frac{1}{2})\geq 2,
\end{split}
\end{equation}
unless $\langle v,u_j\rangle=1$, and $v=b_1u_j+a_2$ with $b_1=1,2$, in which case $\codim\FF(a_1,a_2)^0=1$.  

Finally, if $a_i^2>0$ for all $i$, then $\codim\FF(a_1,...,a_n)^0\geq 2$ by Proposition \ref{Prop:HN filtration all positive classes}.

We have thus proven the following proposition.

\begin{Prop}\label{Prop: 1-1 case totally semistable and codim 1}
Let $\WW$ be a potential wall for positive and minimal $v$ such that $\HH_{\WW}$ is isotropic with $\ell(u_1)=\ell(u_2)=1$.  If $\WW$ is totally semistable, then $v^2=2$, $\langle v,u_2\rangle=1$, and there exists an effective spherical class $w\in\HH_{\WW}$.  If $\codim(\MM_{\sigma_0}(v)\backslash\MM^s_{\sigma_0}(v))=1$, then $\langle v,w\rangle=0$ for a spherical or exceptional class $w$ or $\langle v,u\rangle=1$ for a primitive isotropic $u\in\HH_{\WW}$.
\end{Prop}

We prove the converse to Proposition \ref{Prop: 1-1 case totally semistable and codim 1} in the following lemmas.
\begin{Lem}\label{Lem:isotropic totally semistable divisorial contraction l=1}
Suppose that $v^2=2$, $\langle v,u_2\rangle=1$, and that there exists an effective spherical class $w\in\HH_{\WW}$.  Then $\WW$ is totally semistable and induces a $\P^1$-fibration on $M_{\sigma_+}(v,L)$ for $L\equiv D+\frac{\rk v}{2}K_X\pmod 2$, where $D$ is a nodal cycle.  For the other determinant, $M_{\sigma_+}(v,L+K_X)\backslash M_{\sigma_0}^s(v,L+K_X)$ is a divisor which does not get contracted.  
\end{Lem}
\begin{proof}
Write $v=xw+yu_2$ for $x,y\in\Z$.  Then $1=\langle v,u_2\rangle=x\langle w,u_2\rangle$ which implies that $x=1=\langle w,u_2\rangle$.  But then $v^2=2$ forces $y=2$.  Since $\ell(u_2)=1$, $M^s_{\sigma_0}(2u_2,2L')$ is non-empty and two-dimensional \cite[Proposition 1.9]{Yos16a}.  Moreover, for the unique $\sigma_0$-stable spherical object $S$ of class $w$ and any $A\in M^s_{\sigma_0}(2u_2,2L')$, stability ensures that $\ext^1(A,S)=\langle 2u_2,w\rangle=2$.  Then the $\P^1$'s worth of extensions $$0\to S\to E\to A\to 0$$ get contracted by crossing $\WW$ and generically sweep out the entire irreducible component $M_{\sigma_+}(v,L)$, where $$L=2L'+\det(S)\equiv \det(S)\equiv D+\frac{\rk v}{2}K_X\pmod 2.$$

For the other determinant, observe that the only decompositions of $v$ into effective classes are $v=w+2u_2=u_1+u_2$, and from the proof of Proposition \ref{Prop: 1-1 case totally semistable and codim 1} only the former decomposition corresponds to a totally semistable wall.  Moreover, in the case of the decomposition $v=w+2u_2$ for the determinant $L+K_X$ the strictly $\sigma_0$-semistable locus has codimension 1, so $\WW$ is not totally semistable for $M_{\sigma_+}(v,L+K_X)$.  Indeed, if $E\in M_{\sigma_+}(v,L+K_X)$ has this decomposition for its Harder-Narasimhan filtration with respect to $\sigma_-$, then the quotient of $E$ by $S$ would be in $M_{\sigma_-}(2u_2,2L'+K_X)$.  But $M_{\sigma_-}^s(2u_2,2L'+K_X)=\varnothing$ by \cite[Proposition 1.9]{Yos16a}, so $A=E/S$ would have to be strictly $\sigma_-$-semistable, and from determinant considerations its Jordan-H\"{o}lder factors would have to be $A_1\in M_{\sigma_-}(u_2,L')$ and $A_2\in M_{\sigma_-}(u_2,L'+K_X)$.  But necessarily $A_1\ncong A_2,A_2(K_X)$, as $\det(A_1)\neq\det(A_2)=\det(A_2(K_X))$, so $$\ext^1(A_1,A_2)=\langle u_2,u_2\rangle+\hom(A_1,A_2)+\ext^2(A_1,A_2)=\hom(A_2,A_1(K_X))=0,$$ from which it follows that $A=A_1\oplus A_2$.  Using \cite[Lemmas 6.1-6.3]{CH15}, we thus get a unique $\sigma_+$-stable extension $$0\to S\to E\to A_1\oplus A_2\to 0,$$ unique in its S-equivalence class with respect to $\sigma_0$, and varying the $A_i$ spans a divisor of strictly $\sigma_0$-semistable objects with the prescribed Harder-Narasimhan filtration for $\sigma_-$.  

Now consider the other decomposition, $v=u_1+u_2$, and take $A_1\in M_{\sigma_+}(u_1,L_1)$ and $A_2\in M_{\sigma_+}(u_2)$.  Then any nontrivial extension $$0\to A_1\to E\to A_2\to 0$$ is $\sigma_+$-stable by \cite[Lemma 9.3]{BM14b}, as the parallelogram spanned by $u_1$ and $u_2$ has no lattices points other than its vertices.  In order for $\det(E)=L+K_X$ we must have $\det(A_2)+L_1=L+K_X$.  But as $M_{\sigma_0}^s(u_1)=\varnothing$, $A_1$ must be strictly $\sigma_0$-semistable with stable factors $S$ and $A_2'\in M_{\sigma_+}(u_2)$ so that $$K_X\equiv\det(A_2)+\det(A_2)'\pmod 2,$$ that is, $\det(A_2)\neq\det(A_2)'$.  Hence $\Hom(S,A_2)=\Hom(A_2'(K_X),A_2)=0$, and thus more importantly $\Hom(A_1(K_X),A_2)=0$, so we see that $\ext(A_2,A_1)=1$, and there exists a unique $\sigma_+$-stable extension $E$, unique also in its S-equivalence class with respect to $\sigma_0$.  Letting the $A_i$ vary again sweeps out a divisor that is not contracted by $\WW$, as claimed.
\end{proof}

\begin{Lem}\label{Lem:isotropic divisorial l=1 1}
Assume that $\langle v,u_2\rangle=1$.  Then $\WW$ induces a divisorial contraction if $v^2\geq 3$.  If $v^2=1$ or 2, then $M_{\sigma_+}(v,L)\backslash M^s_{\sigma_0}(v,L)$ is a divisor which is not contracted by $\WW$ \textcolor{red}{unless $L \equiv D+\frac{\rk v}{2}K_X (\mod 2)$,
where $D$ is a nodal cycle.}
\end{Lem}
\begin{proof}
Suppose first that $\HH_{\WW}$ contains an effective spherical or exceptional class $w$ and write $v=xw+yu_2$ with $x,y\in\Z$.  Then $1=\langle v,u_2\rangle=x\langle w,u_2\rangle$, so $x=1=\langle w,u_2\rangle$.  It follows that $v^2=w^2+2y\equiv w^2\pmod 2$.  

Set $v':=v-2u_2$, and suppose that $v^2\geq 3$ if $w$ is exceptional and $v^2\geq 8$ if $w$ is spherical.  Then $v'^2=v^2-4\geq-1$ ($v'^2\geq 4$, resp.).  Then $\langle v',u_2\rangle=1$, so $v'\in C_{\WW}$ and $M^s_{\sigma_0}(v')\neq\varnothing$ by Proposition \ref{Prop: 1-1 case totally semistable and codim 1} as $v'^2\neq 2$ if $\HH_{\WW}$ contains an effective spherical class.  Then for $E_1\in M^s_{\sigma_0}(v')$ and $E_2\in M^s_{\sigma_0}(2u_2)$, the $\P^1$ worth of extensions of the form $$0\to E_1\to E\to E_2\to 0$$ is contracted by $\WW$, and varying $E_1$ and $E_2$ in their moduli sweeps out a divisor in $M_{\sigma_+}(v)$.  Moreover, as $\det(E_2)=2L'$ as above, we may choose $E_1$ to have either determinant in $M_{\sigma_0}^s(v')$ to get a divisorial contraction on each irreducible component $M_{\sigma_+}(v,L_i),i=1,2$.  The same argument gives a divisorial contraction on $M_{\sigma_+}(v,L)$ where $L\equiv D+(\frac{\rk v}{2}+1)K_X\pmod 2$ if $v^2=6$, as $v'^2=2$ and $M_{\sigma_0}^s(v',L-2L')\neq\varnothing$ by Lemma \ref{Lem:isotropic totally semistable divisorial contraction l=1}.

It remains to consider the case when $v^2=4$ and the case when $v^2=6$ and $L\equiv D+\frac{\rk v}{2}K_X\pmod 2$  with $w$ a spherical class.  If $v^2=4$ so that $v'^2=0$, then as $\langle v',u_2\rangle=1$ we must have $v'=u_1$ and $\langle u_1,u_2\rangle=1$.  So $v=w+3u_2$ with $w$ spherical.  Take the unique $\sigma_0$-stable object $S$ of class $w$, let $E_1\in M^s_{\sigma_0}(2u_2)$, and $E_2\in M^s_{\sigma_0}(u_2)$, and consider extensions of the form $$0\to 0\to S\to E\to E_1\oplus E_2\to 0.$$  These extensions move in a one-dimensional family by \cite[Lemma 6.3]{CH15} and are $\sigma_+$-stable by \cite[Lemma 6.1]{CH15}.  For fixed $E_i$, this curve of extensions is contracted by $\WW$ and varying the $E_i$ sweeps out a divisor.  Moreover, although $E_1$ and $S$ can each only have one of the two possible determinants, $E_2$ can have determinant either $L'$ or $L'+K_X$, giving a divisorial contraction in each component as before.

If instead $v^2=6$ and $L\equiv D+\frac{\rk v}{2}K_X\pmod 2$, then $v=w+4u_2$.  Instead of the above decomposition ($v=(v-2u_2)+2u_2$), we use a different one, $v=w+v''$ where $v'':=2u_2+2u_2$.  Indeed, take the unique $\sigma_0$-stable object $S$ of class $w$, and two non-isomorphic objects $E_1,E_2\in M^s_{\sigma_0}(2u_2)$.  Then by \cite[Lemmas 6.1-6.3]{CH15} the extensions of the form $$0\to S\to E\to E_1\oplus E_2\to 0$$ are $\sigma_+$-stable and move in a two-dimensional family contracted to the same point by $\WW$.  Varying $(E_1,E_2)\in (M^s_{\sigma_0}(2u_2)\times M^s_{\sigma_0}(2u_2))\backslash\Delta$ sweeps out a contracted divisor in $M_{\sigma_+}(v)$.

For the second claim for such $\HH_\WW$, we must consider $v^2=2$ and $v^2=1$.  The first case occurs when $w$ is spherical and has been dealt with in Lemma \ref{Lem:isotropic totally semistable divisorial contraction l=1}. {\color{red}It seems to be totally semis-stable.} The second entails $w$ being exceptional, and we must have $v=w+u_2$.  Letting $F\in M_{\sigma_0}^s(w)$ and $G\in M_{\sigma_0}^s(u_2)$ and considering non-trivial extensions $$0\to F\to E\to G\to 0,$$ which are easily seen to be unique, we sweep out a non-contracted divisor in each of $M_{\sigma_+}(v,L)$ and $M_{\sigma_+}(v,L+K_X)$.  

Finally, we suppose that $\HH_{\WW}$ contains no effective spherical or exceptional class.  We note that since $\langle v,u_2\rangle=1$, if $v^2$ were odd, then $\frac{v^2+1}{2}u_2-v$ would be an exceptional class, so $v^2$ must be even.  Moreover, $\langle v,u_2\rangle=1$ implies that $v$ and $u_2$ generate $\HH_{\WW}$, so we may write $u_1=xv+yu_2$, from which we conclude that $\HH_{\WW}=\Z u_1+\Z u_2,\langle u_1,u_2\rangle=1$, and $v=u_1+\frac{v^2}{2}u_2$.  If $v^2\geq 3$, then in-fact $v^2\geq 4$ in this case, so we may write $v=v'+2u_2$ and as above we get a divisor swept out by contracted $\P^1$'s of extensions $$0\to E_1\to E\to E_2\to 0$$ with $E_2\in M_{\sigma_0}^s(2u_2)$ and $E_1\in M_{\sigma_0}^s(v')$, the latter of which is non-empty by Proposition \ref{Prop: 1-1 case totally semistable and codim 1} if $v'^2>0$.  If instead $v'^2=0$, then $v'=u_1$ and $M_{\sigma_0}^s(u_1)\neq\varnothing$ because $C_{\WW}=P_{\HH}$ under the assumptions on $\HH_{\WW}$.  The final option to consider is $v^2=2$, in which case the only possibility for a destabilizing exact sequence is $$0\to E_1\to E\to E_2\to 0$$ for $E_i\in M^s_{\sigma_0}(u_i)$ which span a divisor which is not contracted as $\ext^1(E_2,E_1)=1$.
\end{proof}

\begin{Lem}\label{Lem:isotropic divisorial 1-1 2}
Suppose that $\langle v,w\rangle=0$ for an effective spherical or exceptional class $w$.  
\textcolor{red}{Assume that $L \not \equiv D+\frac{\rk v}{2}K_X (\mod 2)$ if $v^2=2$, where $D$ is a nodal cycle.} 
If $w^2=-2$ and $\langle v,u_2\rangle>1$, then $\WW$ induces a divisorial contraction, while if $w^2=-2$ and $\langle v,u_2\rangle=1$ or $w^2=-1$, then $\codim(M_{\sigma_+}(v,L)\backslash M^s_{\sigma_0}(v,L))=1$ but this divisor is not contracted. 
\end{Lem}
\begin{proof}
Consider the Mukai vector $a:=v-w$.  Then $$a^2=v^2+w^2>w^2, \langle a,v \rangle=v^2>0,
\mbox{ and }\langle a,w\rangle=-w^2.$$  If $a^2>0$, then since $\langle a,w\rangle>0$, $M^s_{\sigma_0}(a)\neq\varnothing$ by Proposition \ref{Prop: 1-1 case totally semistable and codim 1}, so the generic member of $M_{\sigma_+}(v)\backslash M^s_{\sigma_0}(v)$ fits into a short exact sequence $$0\to E_1\to E\to E_2\to 0,$$ where $E_1\in M^s_{\sigma_0}(w)$ and $E_2\in M^s_{\sigma_0}(a)$.  If $w^2=-2$, then $E_1$ is unique and such extensions span a $\P^1$ which gets contracted by $\WW$.  If $w^2=-1$, then $E_1\in \{E_0,E_0(K_X)\}$ and there is a unique such indecomposable extension.  Thus the divisor gets contracted if $w$ is an effective spherical class, while the divisor $M_{\sigma_+}(v)\backslash M^s_{\sigma_0}(v)$ is not contracted if $w^2=-1$.  

It remains to consider when $a^2=0$, i.e $v^2=-w^2$.  The fact that $a$ is an effective isotropic class such that $\langle a,w\rangle>0$ implies that $a=ku_2$.  If $w^2=-1$, then $k=\langle u_2,w\rangle=1$, in which case $v=w+u_2$ and the claim follows from Lemma \ref{Lem:isotropic divisorial l=1 1}.  On the other hand, if $w^2=-2$, then $$2=-w^2=\langle a,w\rangle=\langle ku_2,w\rangle=k\langle u_2,w\rangle,$$ so either $v=w+2u_2$ and $\langle u_2,w\rangle=1$ or $v=w+u_2$ and $\langle u_2,w\rangle=2$.  In the first case $\langle v,u_2\rangle=1$, which was excluded by hypothesis (and has already been discussed in Lemmas \ref{Lem:isotropic totally semistable divisorial contraction l=1} and \ref{Lem:isotropic divisorial l=1 1}).  In the second case, we again see that the generic member of $M_{\sigma_+}(v)\backslash M^s_{\sigma_0}(v)$ lies on a contracted $\P^1$ worth of extensions fitting into a short sequence $$0\to E_1\to E\to E_2\to 0$$ with $E_1\in M^s_{\sigma_0}(w)$ and $E_2\in M^s_{\sigma_0}(u_2)$.  These span a divisor which is thus contracted.  
\end{proof}

\begin{Lem}
Suppose that $v^2\geq 3$ and $\langle v,u\rangle=1$ for an isotropic $u$.  Then $\WW$ induces a divisorial contraction.
\end{Lem}
\begin{proof}
The class $u$ is clearly effective, so by Lemma \ref{Lem:isotropic divisorial l=1 1}, we only need to consider $u=u_1$.  Moreover, if $\HH_\WW$ contains no spherical or exceptional classes, then the corresponding argument for $u_2$ works equally well for $u_1$, so we may assume that $\HH_\WW$ contains a spherical or exceptional class $w$ and $u_1=u_2+nw$, where $n>0$ and $n=\langle u_2,w\rangle$ if $w^2=-2$ or $n=2\langle u_2,w\rangle$ if $w^2=-1$. 

If $\langle w,v\rangle\geq 0$, then $$1=\langle v,u_1\rangle=\langle v,u_2\rangle+n\langle v,w\rangle\geq\langle v,u_2\rangle>0,$$ from which it follows that $\langle v,u_2\rangle=1$ and $\langle v,w\rangle=0$.  But this is a contradiction as then $v^2=-w^2<3$, contrary to hypothesis.

Thus $\langle v,w\rangle<0$, so $\WW$ is totally semistable for $v$, and $\WW$ induces a divisorial contraction for $v$ if and only if it induces a divisorial contraction for $v-nw=\Phi_*(v)$, where $\Phi$ is the equivalence in \eqref{eq:iso:Phi}, since the spherical/exceptional twist preserves S-equivalence.  But then $1=\langle \Phi_*(v),u_2\rangle=\langle v,u_1\rangle$, as $u_2=\Phi_*(u_1)$, so the result follows from Lemma \ref{Lem:isotropic divisorial l=1 1}.
\end{proof}

\subsection{Non-minimal case}

Assume that $\phi_{\sigma_+}(w)<\phi_{\sigma_+}(v)$, and hence
$\phi_{\sigma_-}(w)>\phi_{\sigma_-}(v)$, and denote by $E_0\in M^s_{\sigma_0}(w)$, unique if $w^2=-2$.
We set 
$\TT_1:=\langle E_0,E_0(K_X) \rangle$ if $w^2=-1$ (resp. $\TT_1:=\langle E_0\rangle$ if $w^2=-2$) and
$\FF_1$ is the full subcategory of 
$\PP(1)$ generated by $\sigma_0$-stable objects $E$
with $\phi_{\sigma_-}(E)<\phi_{\sigma_-}(E_0)$.
We also set
$\TT_1^*$ is the full subcategory of 
$\PP(1)$ generated by $\sigma_0$-stable objects $E$
with $\phi_{\sigma_+}(E)>\phi_{\sigma_+}(E_0)$
and
$\FF_1^*:=\langle E_0,E_0(K_X) \rangle$ (resp. $\FF_1^*:=\langle E_0\rangle$ if $w^2=-2$).
We set $\AA_0=\PP(1)$,
$\AA_1=\langle \TT_1[-1],\FF_1 \rangle$ and
$\AA_1^*:=\langle \TT_1^*,\FF_1^*[1] \rangle$.
 Let $\Phi:{\bf D}(X) \to {\bf D}(X)$ be the equivalence
defined by $E_0$.
Then we have equivalences
\begin{equation}
\begin{split}
\Phi:& \AA_0 \isomor \AA_1\\
\Phi^{-1}:&\AA_0 \isomor \AA_1^*.
\end{split}
\end{equation}
The proof is similar to that for the non-isotropic case.

Write $n:=-2\langle v,w \rangle$ (resp $n:=-\langle v,w\rangle$), and assume that $n>0$.
Then we have isomorphisms
\begin{equation}\label{eq:iso:Phi}
\begin{matrix}
\Phi:& \MM_{\sigma_+}(v)& \isomor& \MM_{\sigma_-}(v-nw),\\
\Phi:& \MM_{\sigma_+}(v-nw)& \isomor& \MM_{\sigma_-}(v).\\
\end{matrix}
\end{equation}


By Proposition \ref{Prop:isotropic-classification},
we have a birational map
$$ \MM_{\sigma_-}(v-nw) \dashrightarrow\MM_{\sigma_+}(v-nw),$$ in codimension 1
unless either 
(i) $\langle v-nw, u \rangle=1,2$, $\ell(u)=2$, or (ii) $v^2=2$ and
$\langle v-nw,u \rangle=1=\ell(u)=1$,
or (iii) $\langle v-nw,u \rangle=\ell(u)=1$.
We note that the remaining case of Proposition
\ref{Prop:isotropic-classification}
does not occur by $\langle v,w \rangle \ne 0$.
For (ii) or (iii), we also have a birational map, but the stability changes
along a codimension 1 locus.
For case (i), there is an (anti-) equivalence $\phi'$ of ${\bf D}(X)$ which induces
an isomorphism 
$$\Phi': \MM_{\sigma_-}(v-nw) \to \MM_{\sigma_+}(v-nw).$$
If $X$ is unnodal, then we also have $\Phi'$ by Proposition \ref{prop:LGUK3}. 
Precomposing and postcomposing with the isomorphisms from \eqref{eq:iso:Phi}, we get a birational map
$$
\Phi \circ \Phi:\MM_{\sigma_+}(v) \dashrightarrow \MM_{\sigma_-}(v).
$$

\begin{Rem}
If $\langle v, w \rangle=0$, then
$\Phi$ also induces an isomorphism
$$\Phi:\MM_{\sigma_+}(v) \to \MM_{\sigma_-}(v).$$
\end{Rem}







\section{Flopping walls}

\begin{Prop} \label{prop:flops}
Assume $v$ is primitive and that $\WW$ induces neither a divisorial contraction nor a $\P^1$-fibration.  If either
\begin{enumerate}
\item \label{enum:sum2positive}
$v^2\geq 3$ and $v$ can be written as the sum 
$v = a_1 + a_2$ with $a_i\in P_\HH$, or 
\item\label{enum:exceptional} there exists an exceptional class $w$ and either
\begin{enumerate}
\item\label{enum:exceptionalflop1}
$0< \langle  w,v\rangle\leq\frac{v^2}{2}$, or
\item\label{enum:exceptionalflop2}
$\langle v,w\rangle=0$ and $v^2\geq 3$; or
\end{enumerate}
\item\label{enum:spherical} there exists a spherical class $w$ and either
\begin{enumerate}
\item\label{enum:sphericalflop1}
$0 < \langle w, v\rangle < \frac{v^2}2$, or
\item\label{enum:sphericalflop2}
$\langle w,v\rangle=\frac{v^2}{2}$ and $\HH_{\WW}$ falls into subcase \ref{enum:TwoSpherical} of case \ref{enum:TwoNegative} in Proposition \ref{Prop:lattice classification},
\end{enumerate}
\end{enumerate}
then $\WW$ induces a small contraction.
\end{Prop}


\begin{proof}
We consider case \ref{enum:sum2positive} first, so $v=a_1+a_2$ with $a_i\in P_{\HH}$.  Using \cite[Lemma 9.2]{BM14b}, we may assume that the parallelogram with vertices $0,a_1,v,a_2$ does not contain any lattice point other than its vertices.  In particular, the $a_i$ are primitive, and without loss of generality, we may assume that $\phi^+(a_1)<\phi^+(a_2)$.  By Theorem \ref{Thm:generic moduli spaces}, there exist $\sigma_+$-stable objects $A_i$ with $v(A_i)=a_i$.  If $a_i^2>0$ for each $i$, then the signature of $\HH$ forces $\langle a_1,a_2\rangle\geq 2$ so that $\ext^1(A_2,A_1)\geq 2$.  If, say, $a_1^2=0$, then by part \ref{thm:Classification,Divisorial} of Theorem \ref{classification of walls} and the assumptions that $v^2\geq 3$ and that $\WW$ does not induce a divisorial contraction we must have either $\ell(a_1)=2$ and $\langle v,a_1\rangle \geq 3$ or $\ell(a_1)=1$ and $\langle v,a_1\rangle\geq 2$.  So $\langle a_2,a_1\rangle\geq 2$ and again $\ext^1(A_2,A_1)\geq 2$.  By \cite[Lemma 9.3]{BM14b}, any nontrivial extension $$0\to A_1\to E\to A_2\to 0$$ is $\sigma_+$-stable of class $v$.  All such extensions are non-isomorphic but $S$-equivalent with respect to $\sigma_0$, giving a projective space of positive dimension contracted by $\pi^+$.

We move on to case \ref{enum:exceptional} and begin with subcase \ref{enum:exceptionalflop1}.  
{\color{red}
It may be needed to add the following:
Assume that $v$ is minimal. Then $\langle v,w \rangle>0$ means
$w$ is effective. Since $(v-w)^2 \geq -1$, if $-(v-w)$ is effective, then 
$\langle v,-(v-w) \rangle \ge 0$. Hence $\langle v,w \rangle \geq v^2$.
Therefore $v-w$ is effective.}
First assume that $w$ is effective.  From the assumptions, we observe that $\langle w, v-w\rangle=\langle w,v\rangle+1\geq 2$ and $(v-w)^2\geq -1$.  As in the proof of \cite[Proposition 9.1]{BM14b}, we consider the parallelogram $\mathbf{P}$ with vertices $0,w,v,v-w$ and the function $f(a)=a^2$ on $\mathbf{P}$, and the same argument as given there shows that $f(a)>-1$ unless $a\in\{w,v-w\}$.  It follows that if $\mathbf{P}$ contains any lattice point $a$ other than its vertices, then both $a^2\geq 0$ and $(v-a)^2\geq 0$.  So $v$ is the sum of two positive classes, and we are in case \ref{enum:sum2positive}.  It is easy to see that indeed $v$ satisfies the extra condition $v^2\geq 3$ if such an $a$ is isotropic.  We may therefore assume that no such lattice points exist.  Let $S$ be a $\sigma_+$-stable object of class $w$ and $F$ any $\sigma_+$-stable object of class $v-w$.  Then, assuming $\phi^+(w)<\phi^+(v-w)$ without loss of generality, we get $\ext^1(F,S)\geq2$ giving a positive dimensional projective space worth of $\sigma_+$-stable extensions that get contracted by $\pi^+$.

In case \ref{enum:exceptionalflop1} it remains to consider the possibility that $w$ is not effective.  Then $z=-w$ is effective and $\langle v,z\rangle<0$ so that $v$ is not minimal.  Consider the composition of spherical and exceptional twists as in Proposition \ref{Prop:NonMinimalIsomorphism} or Proposition \ref{Prop:CompositionSphericalExceptional} and denote it by $\Phi$.  Then $\Phi_*(v)$ is minimal, and as $\langle \Phi_*(w),\Phi_*(v)\rangle=\langle w,v\rangle$, we see that $\Phi_*(w)$ is an effective exceptional class and satisfies the same inequality in the hypothesis of \ref{enum:exceptionalflop1} for $\Phi_*(v)$.  Moreover, depending on the parity of the index of the chamber occupied by $v$, we either get $M_{\sigma_\pm}(v)\cong M_{\sigma_\pm}(\Phi_*(v))$ or $M_{\sigma_{\pm}}(v)\cong M_{\sigma_\mp}(\Phi_*(v))$, and the S-equivalence class of $E\in M_{\sigma_+}(v)$ is determined by that of $\Phi(E)\in M_{\sigma_\pm}(\Phi_*(v))$, respectively.  Thus the result follows from the work of the previous paragraph.

Now let us consider subcase \ref{enum:exceptionalflop2}.  Since $\langle v,w\rangle=0$, we may assume that $w$ is effective.  As $v^2\geq3$ by assumption, $(v-2w)^2\geq-1$ and $\langle w,v-2w\rangle=2$.  Let $S,S(K_X)$ be the two $\sigma_+$-stable exceptional objects of class $w$ and let $F$ be a $\sigma_+$-stable object of class $v-2w$.  As in the previous cases, we may assume that the parallelogram with vertices $0,w,v,v-w$ has no additional lattices points so that therefore the parallelogram with vertices $0,2w,v,v-2w$ has no other lattices points than $0,w,2w,v,v-w,v-2w$.  Without loss of generality we may assume that $\phi^+(w)>\phi^+(v)$.  Then for any extension $$0\to F\to E\to S\oplus S(K_X)\to 0$$ corresponding to non-zero extensions in each of $\Ext^1(S,F)$ and $\Ext^1(S(K_X),F)$, $E$ satisfies $$\Hom(S,E)=\Hom(S(K_X),E)=0.$$  It follows that $E$ is $\sigma_+$-stable of class $v$.  Indeed, if not, then the class of the maximal destabilizing subobject $A$ would satisfy $\phi^+(v(A))>\phi^+(v)$ and thus must either be $w$ or $2w$.  But then we would get $\Hom(S,E)\neq 0$ or $\Hom(S(K_X),E)\neq 0$, a contradiction.  Thus we get a $\P^1\times\P^1$ worth of non-isomorphic $\sigma_+$-stable objects of class $v$ that gets contracted by $\pi^+$.

Finally, we move on to \ref{enum:spherical} and deal with both subcases at the same time.  The proof proceeds exactly as in \cite[Lemma 9.1, case (b)]{BM14b} persuant to the following remarks.  First, if there is a lattice point $a$ in the parallelogram with vertices $0,w,v,v-w$ that satisfies $a^2=-1$, then we are in case \ref{enum:exceptional}, so we can discount this possibility.  Second, in subcase \ref{enum:sphericalflop2} the assumption that $\HH_\WW$ falls into subcase \ref{enum:TwoSpherical} of case \ref{enum:TwoNegative} of Proposition \ref{Prop:lattice classification} guarantees that all $(-2)$-classes are spherical.  Thus, the class $v-w$, which satisfies $(v-w)^2=-2$, is necessarily a spherical class.  The argument of \cite[Lemma 9.1, case (b)]{BM14b} then carries through without change.
\end{proof}
\todo{The only thing we would need to do in the above proof to allow $v$ to be non-primitive is to show that the extensions constructed are not $S$-equivalent with respect to $\sigma_+$.  A small modification to Lemmas 9.2 and 9.3 in \cite{BM14b} (which I've included at the end of the source file) gives $\sigma_+$-semistable extensions.  Now, we can assume that $\WW$ is a fake wall for all $m'v_0$ where $m'<m$ and $v=mv_0$.  Indeed, any curve contracted by $\pi^+$ for $m'v_0$ can be easily embedded into $M_{\sigma_+}(v)$ to give another curve contracted by $\pi^+$.  I believe that under this assumption, we can show that the above extensions are distinct with respect to $S$-equivalence.  But let's move on for the moment.}

Now we prove the converse to Proposition \ref{prop:flops}.
\begin{Prop}\label{prop: fake or non-walls}
Assume that $v$ is primitive and that $\WW$ induces neither a divisorial contraction nor a $\P^1$-fibration.  Assume further that none of the conditions \ref{enum:sum2positive}-\ref{enum:spherical} in Proposition \ref{prop:flops} are satisfied.  Then $\WW$ is either a fake wall, or not a wall.
\end{Prop}
\begin{proof}
We consider first the case that $v$ is minimal in its $G_{\HH}$-orbit.  Furthermore, we assume for now that $v^2\geq 3$ and prove that in this case every $\sigma_+$-stable object $E$ of class $v$ is $\sigma_0$-stable.  If not, then some such $E$ is strictly $\sigma_0$-semistable, and thus $\sigma_-$-unstable.  Let $a_1,\ldots,a_n$ be the Mukai vectors of the HN-filtration factors of $E$ with respect to $\sigma_-$.  By assumption on the failure of condition \ref{enum:sum2positive}, the $a_i$ cannot all be in $P_{\HH}$, so $E$ must have a destabilizing spherical or exceptional subobject or quotient $\tilde{S}$ with $v(\tilde{S})=\tilde{w}$.  

If there is only one $\sigma_0$-stable spherical or exceptional object (in the latter case, uniqueness is only up to $-\otimes\OO(K_X)$ of course), then clearly $v-\tilde{w}\in P_{\HH}$, so $v^2-2\langle \tilde{w},v\rangle+w^2\geq 0$, contradicting the assumption about the failure of conditions \ref{enum:exceptional} and \ref{enum:spherical} of Proposition \ref{prop:flops}.


Now suppose instead that there are two $\sigma_0$-stable spherical/exceptional objects with Mukai vectors $w_1,w_2$.  We must have $v-\tilde{w}\in C_{\WW}$, and moreover, by \cite[Lemma 4.6]{Yos16b} any stable factor of $\tilde{S}$ must also be spherical or exceptional, so $v-w_1$ or $v-w_2$ must be effective as well.  The assumption about failure of conditions \ref{enum:exceptional} and \ref{enum:spherical} in addition to the minimality assumption on $v$ force $\langle v,w_i\rangle>\frac{v^2}{2}$, and thus that $(v-w_i)^2<w_i^2$, for $i=1,2$.  But then $v$ must lie above the concave up hyperbola $(v-w_1)^2=w_1^2$ and below the concave down hyperbola $(v-w_2)^2=w_2^2$.  In case $w_1^2=w_2^2$, these two hyperbolas intersect at 0 and $w_1+w_2$, while if, say, $-1=w_1^2\neq w_2^2=-2$, then we must have $(v-w_1)^2\leq -2$ and $(v-w_2)^2<-2$.  Similarly to the previous case, $v$ must lie above or on the hyperbola $(v-w_1)^2=-2$ and below the concave down hyperbola $(v-w_2)^2=-2$.  One can easily check that these hyperbola intersect at two points inside the unit square, so in either case, writing $v=xw_1+yw_2$, we must have $x,y<1$.  But then neither $v-w_1$ nor $v-w_2$ can be effective, a contradiction.

If $v$ is not minimal, then $v\in\CC_n$ for some $0\neq n\in\Z$ and there exists a minimal class $v_0$ in the same orbit.  By Propositon \ref{Prop:NonMinimalIsomorphism}, if $n$ is even then $M_{\sigma_+}(v)\cong M_{\sigma_+}(v_0)$, induced by a sequence of spherical/exceptional twists, and if $n$ is odd, then $M_{\sigma_+}(v)\cong M_{\sigma_+}(v_1)\cong M_{\sigma_-}(v_0)$.  Since the assumptions of the proposition are invariant under $G_{\HH}$, the same asumptions apply to $v_0$, so every $\sigma_\pm$-stable object of class $v_0$ is $\sigma_0$-stable.  If $\Phi$ is the sequence of spherical/exceptional twists used in this isomorphism, then from the definition of a spherical/exceptional twist, it is easy to see that the $S$-equivalence class of $\Phi(E_0)$ is determined by that of $E_0$.  Since this equivalence is trivial on $M_{\sigma_{\pm}}(v_0)$, it must be trivial on $M_{\sigma_+}(v)$ as well, implying that $\pi^+$ is an isomorphism, as claimed. 
\end{proof}

\section{LGU on the covering K3 surface}

\subsection{Geometric stability conditions on a wall}


\begin{Lem}\label{lem:W-general}
Let $W$ be a wall for $v$ defined by a primitive isotropic vector $u$ with
$\ell(u)=1$.
\begin{enumerate}
\item[(1)]
For a general stability condition $\sigma$ on $W$,
there is a Fourier-Mukai transform $\Phi:{\bf D}(X) \to {\bf D}(X)$
such that $\Phi(\sigma)$ is geometric.
\item[(2)]
Let $w$ be a primitive and isotropic Mukai vector with
$\ell(w)=2$.
Then 
for a general stability condition $\sigma$ on $W$, 
$\sigma$ is general with respect to $w$.
\end{enumerate}
\end{Lem}

\begin{proof}
(1) Let $W$ be a wall for $v$ defined by a Mukai vector $u$. 
Assume that $W$ is also a wall for $\varrho_X$.
Let $\sigma$ be a general stability condition on $W$.
Then there is a Fourier-Mukai transform $\Phi:{\bf D}(X) \to {\bf D}(X)$
such that $\Phi(\sigma) \in \overline{U(X)}$
and the wall for $\varrho_X$ is defined by $\OO_C$ or 
$E_0$, where $E_0$ is a spherical or an exceptional bundle.
We shall replace $\sigma$ by $\Phi(\sigma)$.
By the action of $g \in \widetilde{GL}_2^+(\R)$,
we may also assume that 
$Z_{\sigma}=\langle e^{\beta+\sqrt{-1}\omega},\bullet \rangle$.
Then  the wall $W$ is (i) $(c_1(E_0)-\rk E_0 \beta, \omega)=0$ or
(ii) $(C,\omega)=0$.
In the first case, 
$c_1(v) \rk E_0-c_1(E_0)\rk v=0$.
Indeed if $c_1(v) \rk E_0-c_1(E_0)\rk v \ne 0$, then
there is $\omega$ with $(c_1(v) \rk E_0-c_1(E_0)\rk v,\omega) \ne 0$.
For a general $\beta$ satisfying
$(\beta,\omega)=(c_1(E_0),\omega)/\rk E_0$,
$(\beta,\omega)$ does not lie on the wall for $v$.
Thus $c_1(v) \rk E_0-c_1(E_0)\rk v=0$.
Then $\varrho_X \in \HH$, which implies $\HH$ contains
an isotropic vector $u'$ with $\ell(u')=2$.
Since $v(E_0) \in \HH$, Remark \label{Rem:negative stable classes} implies
$\ell(u)=2$, which is a contradiction.

We treat the secind case.
If $\rk v \ne 0$, then
we can take $\beta$ such that $Z_\sigma(u) \not \in \R Z_\sigma(v)$.
If $\rk v=0$, then we can take $\omega \in C^\perp$
such that $(c_1(v),\omega) \ne 0$ by $v^2 \geq 0$.
Then we can find $\beta$ such that
$Z_\sigma(u) \not \in \R Z_\sigma(v)$.

(2) Since $\ell(w)=2$, there is an autoequivalence
$\Phi:{\bf D}(X) \to {\bf D}(X)$
such that $\Phi(w)=\varrho_X$.
Hence the claim follows from (1). 
\end{proof}  
 


\subsection{Fourier-Mukai transform associated to
Uhlenbeck contraction}

Let $X_1$ be a K3 surface and $u_1$ 
a primitive and isotropic Mukai vector on $X_1$.
For a general stability condition
$\sigma$, we set $X_2:=M_\sigma(u_1)$.
$X_2$ is a K3 surface.
Let $\EE \in {\bf D}^{1_{X_1} \times \alpha}(X_1 \times X_2)$ be a
universal object as a twisted object, where
$\alpha$ is a 2-cocycle of $\OO_{X_2}^{\times}$. 
We set
\begin{equation}
\Phi:=\Phi_{X_1 \to X_2}^{\EE^{\vee}},\;
\Psi:=\Phi_{X_1 \to X_2}^{\EE}.
\end{equation}
For a smooth variety $Y$,
we also set
\begin{equation}
D_Y(E):=E^{\vee}={\bf R}\HH om_{\OO_Y}(E,\OO_Y),\; E \in {\bf D}(Y).
\end{equation}
Let $v_1$ be a Mukai vector such that $\langle u_1,v_1 \rangle=2$.


\begin{Prop}\label{prop:uhl}
There is an object $\FF \in {\bf D}(X_1 \times X_1)$
which induces a Fourier-Mukai transform
$\Phi_{X_1 \to X_1}^{\FF^{\vee}}:{\bf D}(X_1) \to {\bf D}(X_1)$
such that
\begin{enumerate}
\item
$\FF_{|X_1 \times \{ x \}}$ $(x \in X_1)$ is a $\sigma$-stable object.
\item
\begin{equation}
D_{X_1} \circ \Phi_{X_1 \to X_1}^{\FF^{\vee}}(x)
=-(x+\tfrac{v_1^2}{2} \langle x,u_1 \rangle u_1-\langle x,v_1 \rangle u_1-
\langle x,u_1 \rangle v_1).
\end{equation}
\end{enumerate} 
\end{Prop}




\begin{proof}
For $F \in {\bf D}(X_1)$ with $v(F)=v_1$,
we set
\begin{equation}
N:=\det \Phi(F).
\end{equation}
We have the following commutative diagram:
\begin{equation}
\begin{CD}
{\bf D}(X_1) @>{\Phi}>> {\bf D}^{\alpha^{-1}}(X_2) @>{D_{X_2}}>> 
{\bf D}^{\alpha}(X_2)  
@>{\otimes N}>>{\bf D}^{\alpha^{-1}}(X_2)@>{\Psi}>>  {\bf D}(X_1).
\end{CD}
\end{equation}

We define a (contravariant)-equivalence
${\bf D}(X_1) \to {\bf D}(X_1)$ by
\begin{equation}
\Xi:=\Psi \circ(\otimes N) \circ D_{X_2} \circ \Phi.
\end{equation}
Thus
$D_{X_1} \circ \Xi$ is an autoequivalence of ${\bf D}(X_1)$. 
Since
$\Xi=D_{X_1} \circ \Phi \circ(\otimes N^{\vee})  \circ \Phi$
up to shift,
$D_{X_1} \circ \Xi$ defines a Fourier-Mukai transform
$\Phi_{X_1 \to X_1}^{\FF^{\vee}}:{\bf D}(X_1) \to {\bf D}(X_1)$.


We set
$v_2:=v(\Phi(F))=(2,\xi,a)$.
Then
$$
(\Q v_2+\Q \varrho_{X_2})^\perp=\{e^{\xi/2}(0,D,0) \mid D \in \NS(X_2)_{\Q}\}.
$$
Under $(\otimes N) \circ D_{X_2}$, we have
\begin{equation}
\begin{matrix}
v_2 & \mapsto & v_2\\
\varrho_{X_2} & \mapsto & \varrho_{X_2}\\
e^{\xi/2}(0,D,0) & \mapsto & -e^{\xi/2}(0,D,0).
\end{matrix}
\end{equation}
Hence
\begin{equation}
\Xi_{|(\Q u_1+\Q v_1)}=1_{(\Q u_1+\Q v_1)},\;
\Xi_{|(\Q u_1+\Q v_1)^\perp}=-1_{(\Q u_1+\Q v_1)^\perp}.
\end{equation}
Then it is easy to see that
\begin{equation}
\Xi(x)=-(x+\tfrac{v_1^2}{2} \langle x,u_1 \rangle u_1-
\langle x,v_1 \rangle u_1-
\langle x,u_1 \rangle v_1).
\end{equation}

Since $\sigma$ is geometric and $\EE$ is a family of $\sigma$-stable objects,
$\sigma':=\Phi \circ(\otimes N^{\vee}) \circ \Phi(\sigma)$ is also geometric.
In particular, $k_x$ $(x \in X_1)$ is $\sigma'$-stable.
Then $\FF_{|X_1 \times \{ x\}}$ is $\sigma$-stable.  
\end{proof}



\subsection{Fourier-Mukai transform on an Enriques surface}

Let $\sigma$ be a geomeric stability condition on $X$ which is
general with 
respect to a primitive and isotropic Mukai vector $v_0$ with
$\ell(v_0)=2$.
We write $\varpi^*(v_0)=2 \tilde{v}_0$.
Let $\sigma'$ be the stability condition on $\widetilde{X}$
such that $E \in {\bf D}(\widetilde{X})$ is $\sigma'$-semi-stable
if and only if $\varpi_*(E)$ is $\sigma$-semi-stable.
\begin{Lem}\label{lem:v_0:general}
\begin{enumerate}
\item[(1)]
$M_{\sigma'}(\tilde{v}_0)$ conststs of stable objects.
\item[(2)]
$\iota$ acts freely on $M_{\sigma'}(\tilde{v}_0)$.
\end{enumerate}
\end{Lem}

\begin{proof}
(1) For $E \in M_{\sigma'}(\tilde{v}_0)$,
$\varpi_*(E)$ is a $\sigma$-semi-stable object
with $v(\varpi_*(E))=v_0$.
Since $\sigma$ is general,
$\varpi_*(E)$ is $\sigma$-stable.
If $E$ contains a subobject $E_1$ with
$\phi_{\sigma'}(E_1)=\phi_{\sigma'}(E)$, then
$\varpi_*(E_1)$ is a subobject of $\varpi_*(E)$
with $\phi_\sigma(\varpi_*(E_1))=\phi_\sigma(\varpi_*(E))$.
Hence $E_1=E$, which implies our claim.

(2)
If $\iota^*(E) \cong E$, then
$E \cong \varpi^*(F)$.
Then $v_0=v(\varpi_*(E))=2v(F)$, which is a contradiction.
\end{proof}  





\begin{Prop}\label{prop:Enriques-refl}
Let $u$ and $v$ be Mukai vectors such that
$u$ is primitive and isotropic,
$\langle u,v \rangle=1$ and $\ell(u)=1$.
We set
$$
v_0:=-(\varrho_X+2v^2 \langle \varrho_X,u \rangle u
-2\langle \varrho_X,v \rangle u-2
\langle \varrho_X,u \rangle v).
$$
Assume that $\sigma$ is a geometric stability condition
which is general with respect to $v_0$.
Then there is an autoequivalence 
$\Phi_{X \to X}^{\EE^{\vee}}:{\bf D}(X) \to {\bf D}(X)$
such that 
\begin{enumerate}
\item[(1)]
$\EE_{|X \times \{x \}}$ is $\sigma$-stable
\item[(2)]
\begin{equation}
D_X \circ \Phi_{X \to X}^{\EE^{\vee}}(x)=
-(x+2v^2 \langle x,u \rangle u-2\langle x,v \rangle u-
2\langle x,u \rangle v).
\end{equation}
\end{enumerate}
\end{Prop}


\begin{proof}
We note that $v_0$ is a primitive and isotropic Mukai vector with
$\ell(v_0)=2$.
Let $\sigma'$ be the stability condition on $X_1:=\widetilde{X}$
induced by $\sigma$.
We set $v_1:=\varpi^*(v)$ and $u_1:=\varpi^*(u)$.
By Lemma \ref{lem:v_0:general},
we have a chamber $U$ for $\tilde{v}_0$ containing $\sigma'$.
Let $\sigma_1$ be a stability condition in $U$
which is general
with respect to $u_1$. 
By Proposition\ref{prop:uhl}, we have an object
$\FF \in {\bf D}(\widetilde{X} \times \widetilde{X})$
which satisfies (i) and (ii) of Proposition\ref{prop:uhl}.

By the definition of $\sigma'$,
$\iota^*(\EE_{|X \times \{ x \}})$ is also $\sigma'$-stable,
and hence  
we have  
an isomorphism $\tau:\widetilde{X} \to \widetilde{X}$ 
such that
\begin{equation}
(\iota \times 1_{\widetilde{X}})^*(\EE) \cong
(1_{\widetilde{X}} \times \tau)^*(\EE) \otimes L,
\end{equation}
where $L$ is a line bundle on $\widetilde{X}$.
Then we have a commutative diagram

\begin{equation}
\begin{CD}
{\bf D}(\widetilde{X}) @>{\Phi_{\widetilde{X} \to 
\widetilde{X}}^{\FF^{\vee}}}>> {\bf D}(\widetilde{X})\\
@V{\iota^*}VV @VV{(\otimes L^{\vee}) \circ \tau^*}V \\
{\bf D}(\widetilde{X}) @>>{\Phi_{\widetilde{X} \to 
\widetilde{X}}^{\FF^{\vee}}}> {\bf D}(\widetilde{X}).\\
\end{CD}
\end{equation}

By Proposition\ref{prop:uhl}, we see that
\begin{equation}
\iota^* \circ \Phi_{\widetilde{X} \to \widetilde{X}}^{\FF^{\vee}}(x)
=\Phi_{\widetilde{X} \to \widetilde{X}}^{\FF^{\vee}} \circ \iota^*(x),\;
 x \in H^*(\widetilde{X},\Z).
\end{equation}
Hence we get
\begin{equation}
\iota(x)=(\otimes L^{\vee}) \circ \tau^*(x),\;
x \in H^*(\widetilde{X},\Z),
\end{equation}
which implies $L=\OO_{\widetilde{X}}$
and $\iota=\tau$ on $H^*(X,\Z)$.
By the Torelli theorem,
$\iota=\tau$.
Therefore $\Phi_{\widetilde{X} \to \widetilde{X}}^{\FF^{\vee}}$ and 
$\iota$ are commutative.
Then there is $\EE \in {\bf D}(X \times X)$ such that
\begin{equation}
(\varpi \times 1_{\widetilde{X}})_*(\FF)
\cong (1_X \times \varpi )^*(\EE).
\end{equation}

$\Phi_{X \to X}^{\EE^{\vee}}$ defines a desired 
Fourier-Mukai transform.
\end{proof}


\begin{Prop}\label{prop:LGUK3}
Assume that $X$ is unnodal.
Let $W$ be a wall for $v$ defined by an isotropic Mukai vector
$u$ such that $\langle u,v \rangle=1$ and $\ell(u)=1$.
Let $\sigma \in W$ be a geometric stability condition.
Assume that $\sigma_\pm$ are separated by $W$.
Then 
$D_X \circ \Phi_{X \to X}^{\EE^{\vee}}$ induces an isomorphism
$M_{\sigma_+}(v) \cong M_{\sigma_-}(v)$.
\end{Prop}

\begin{proof}
By Lemma \ref{lem:W-general} (2),
by purturbing $\sigma$ in $W$,
we may assume that $\sigma$
is general with respect to $v_0$ in Proposition \ref{prop:Enriques-refl}.
Then $\Phi_{X \to X}^{\EE^{\vee}}$ in Proposition \ref{prop:Enriques-refl}
is defined. In order to study the stability,
we first assume that $\Pic(\widetilde{X})=\Pic(X)$, where
$\pi:\widetilde{X} \to X$ is the covering K3 surface.
Then $H^*_{\alg}(\widetilde{X},{\Bbb Z})=2 \NS(X) \oplus \langle 2 \rangle
\oplus \langle -2 \rangle$.
We set $X_1:=\widetilde{X}$ and
$\pi^*(v)=v_1,\pi^*(u)=u_1$.
Then there is no $w$ with $\langle w,u_1 \rangle=1$.
In paricular for $X_2:=M_\sigma(u_1)$,
$\alpha$ defines a non-trivial Brauer class.
By our assumption, $\pi^*(\sigma_\pm)$ are general with respect to $\pi^*(v)$. 
Assume that $F \in \MM_{\sigma_+}(v)$.
Then for the equivalence $\Phi$ in the proof of Proposition \ref{prop:uhl},
$\Phi(\pi^*(F))$ is a $\mu$-stable $\alpha^{-1}$-twisted sheaf.
Since the $\mu$-stability is independent of the choice of the B-field,
$\Xi(\pi^*(\sigma_+))$ and $\pi^*(\sigma_-)$ belong to the same chamber.
Therefore $\Xi$ induces an isomorphism
$\MM_{\pi^*(\sigma_+)}(v_1) \to \MM_{\pi^*(\sigma_-)}(v_1)$.  
Hence $D_X \circ \Phi_{X \to X}^{\EE^{\vee}}$ induces
an isomorphism $\MM_{\sigma_+}(v) \to \MM_{\sigma_-}(v)$.
Thus we get the claim in this case.
Since the structure of walls is the same if $X$ is unnodal,
the same claim also holds. Therefore Proposition \ref{prop:LGUK3} holds.
\end{proof}




\section{Main theorems}
\begin{proof}[Proof of Theorem \ref{Thm:MainTheorem1}]
We may connect $\sigma$ and $\tau$ by a path which intersects walls in points that lie on no other walls.  As the set of walls is locally finite, the path will intersect only finitely many walls, and thus for the purpose of proving both parts of the theorem, it suffices to consider one wall $\WW$, a generic stability condition $\sigma_0\in\WW$, and nearby stability conditions $\sigma_\pm$.  

As part (a) follows from part (b), except possibly for the fourth type of divisorial contraction, we only need to prove part (a) directly in this case.  But as $\WW$ is not totally semistable in this case, $M_{\sigma_0}(v)$ is a nonempty open subset which induces a birational map $M_{\sigma_+}(v)\dashrightarrow M_{\sigma_-}(v)$, as claimed.

Moving on to part (b), suppose that $\langle v,w\rangle<0$ for an effective spherical or exceptional class $w\in\HH_\WW$.  Then denoting by $v_0$ the minimal Mukai vector in the $G_\HH$-orbit of $v$, with respect to which $v=v_n$, and letting $\Phi^\pm$ be the sequence of spherical or exceptional twists giving the isomorphism $M_{\sigma_\pm}(v)\to M_{\sigma_\pm}(v_0)$ if $n$ is even (resp., $M_{\sigma\pm}(v)\to M_{\sigma_\mp}(v_0)$ if $n$ is odd), we see that it suffices to prove the theorem in the case that $v$ is minimal.  Indeed, if $\Phi$ is an autoequivalence inducing a birational map $M_{\sigma_+}(v_0)\dashrightarrow M_{\sigma_-}(v_0)$ as in the theorem, then $(\Phi^-)^{-1}\circ\Phi\circ\Phi^+$ proves the theorem for $v$ if $n$ is even (resp., $(\Phi^-)^{-1}\circ\Phi^{-1}\circ\Phi^+$ if $n$ is odd).

Thus we may assume that $\langle v,w\rangle\geq 0$ for all effective spherical and exceptional classes $w\in\HH_\WW$, and we break up the proof into cases according to Theorem \ref{classification of walls}.  We first consider the case that $\WW$ is a flopping wall or a fake wall.  If $\codim(M_{\sigma_+}(v)\backslash M_{\sigma_0}^s(v))\geq 2$, then we may take $U$ to be the open subset of $\sigma_0$-stable objects, so there is nothing to prove (i.e. we just take $\Phi=\Id$).  By Theorem \ref{classification of walls}, if $v$ is minimal as assumed, then $\WW$ cannot be a totally semistable flopping or fake wall, so we only need to consider the case that $\codim(M_{\sigma_+}(v)\backslash M_{\sigma_0}^s(v))=1$, which can only occur when $\langle v,w\rangle=0$ for an effective spherical or exceptional class $w\in\HH_\WW$ by Lemma \ref{Lem:non-isotropic no totally semistable wall}, Proposition \ref{Prop:LGU walls of low codimension}, and Proposition \ref{Prop: 1-1 case totally semistable and codim 1}.  For the purpose of part (b), we can prove this case along with that of Brill-Noether type divisorial contractions and both types of $\P^1$-fibrations.

We assume, consequently, that $v$ is minimal in its $G_\HH$-orbit and that there exists an effective spherical or exceptional class $w\in\HH_\WW$ such that $\langle v,w\rangle=0$.  Without loss of generality, we may assume that $w=w_0=v(T_0^+)$ as in Proposition \ref{Prop:NonMinimalIsomorphism} and Proposition \ref{Prop:CompositionSphericalExceptional} and the discussions following them.  But we showed there that the spherical/exceptional twist $R_{T_0^+}$ induces an isomorphism $M_{\sigma_+}(v)\to M_{\sigma_-}(v)$. 

Now we consider the case that $\WW$ induces a divisorial contraction but $\langle v,w\rangle>0$ for all effective spherical/exceptional classes $w\in\HH_\WW$, i.e. divisorial contractions of Hilbert-Chow, LGU, Name? walls.

\textbf{Hilbert-Chow:} Here we assume $\HH_\WW$ contains an isotropic vector $u_2$ such that $\langle v,u_2\rangle=1$.  Then by Proposition \ref{Prop:Uhlenbeck morphism} and Lemma \ref{Lem: Hilbert-Chow} we may identify $M_{\sigma_+}(v)$ with the Hilbert scheme of $\frac{v^2+1}{2}$ points considered as the moduli space parametrizing the shifts of twisted ideal sheaves $I_Z(L)$ for some $L\in\Pic(X)$ and $\ell(Z)=\frac{v^2+1}{2}$.  Up to applying the autoequivalence $\otimes L$, we may assume $L=\OO_X$.  Moreover, $\sigma_-$-stable objects are the shifts $F^{\vee}[2]$ of an ideal sheaf $F=I_Z$.  But then $\Phi(\blank):=(\blank)^{\vee}[3]$ provides the required autoequivalence.
\todo{Check about this.  I'm worried about the cohomological degree of these duals}

\textbf{LGU:} As above, we identify $M_{\sigma_+}(v)$ with the moduli space $M_{\omega}(-v)$ of shifts $F[1]$ of $\omega$-Gieseker stable sheaves $F$.  Writing $-v=(2,c,s)$ and choosing $L\in\Pic(X)$ with $c_1(L)=c$, we get that $\Phi(\blank):=(\blank)^{\vee}\otimes \OO(L)[3]$ is the required autoequivalence as $\Phi(F[1])=F^{\vee}\otimes \OO(L)[2]$ which is an object of $M_{\sigma_-}(v)$.

\textbf{LGUK3:} \todo{I need to think about whether my argument inducing from the covering K3 works.}
We need to assume that $X$ is unnodal.
Then the claim follows from Proposition \ref{prop:LGUK3}.


\end{proof}

{\color{red}
\section{Picard groups of moduli spaces}

We shall prove Corollary \ref{Cor:Picard}.
For the computation of the Picard groups of odd rank cases,
we can use deformation of Enriques surfaces, since
$\theta_v$ is well-defined for a relative moduli space over
a family of Enriques surfaces.
Thus we may assume that $\Pic(X)=\Pic(\widetilde{X})$, where $\widetilde{X}$
is the covering K3 surface.
By Theorem \ref{Thm:application1} and \ref{Thm:application2}, 
it is sufficient to prove that $\theta_{v,\sigma}$ is an isomorphism 
for a special pair of $v$ and $\sigma$.
We may assume that $v=(1,0,\tfrac{1}{2}-n)$ or 
$v=(2,D,a)$ with
$(D,\eta)=1$ for a divisor $\eta$ with $\eta^2=0$.
For the first case,
$\theta_{v,\sigma}$ is an isomorphism where
$M_\sigma(v)=\Hilb_X^n$ $(n \geq 2)$.
So we shall treat the second case. 
Then replacing $v$ by $v \exp(k \eta)$, we may assume that
$D^2=-2,-4$.
Since $X$ is unnodal,
we can take an ample divisor $H$ with $(D,H)=0$.
We set $n:=\frac{D^2}{2}+1-a$.
Then 
$$
v=(2,D,\tfrac{D^2}{2}+1-n)=v(\OO_X(D))+v(I_Z),\;
I_Z \in \Hilb_X^n.
$$
We take $\beta_0 \in \Pic(X)_{\Q}$
with $(\beta_0,D)=\frac{D^2}{2}+n$.
Then $\chi(\OO_X(D-\beta_0))=\chi(I_{\ZZ_t}(-\beta_0))$.
We take $\beta  \in \Pic(X)_{\Q}$ in a neighborhood of $\beta_0$
such that $\chi(\OO_X(D-\beta))<\chi(I_{\ZZ_t}(-\beta))$.
We take a stability condition $\sigma$ such that $M_\sigma(v)$ is the 
moduli space of $\beta$-twisted stable sheaves $M_H^\beta(v)$.
For a non-trivial extension
$$
0 \to \OO_X(D) \to E \to I_{\ZZ_t} \to 0,\; t \in \Hilb_X^n
$$
$E$ is a $\beta$-twisted stable sheaf.
Since $H^0(X,\OO_X(\pm D))=H^0(X,\OO_X(\pm D+K_X))=0$,
$\VV:=\Ext^1_{p_{\Hilb_X^n}}(I_{\ZZ},\OO_X(D) \boxtimes \OO_{\Hilb_X^n})$ 
is a locally free sheaf 
on $\Hilb_X^n$ of rank $n-1-\tfrac{D^2}{2}$.
For $n>0$, let $P:=\P(\VV^{\vee}) \to \Hilb_X^n$ be the projective bundle
parametrizing non-trivial extensions
of $I_{\ZZ_t}$ by $\OO_X(D)$ and 
\begin{equation}\label{eq:beta}
0 \to \OO_X(D) \boxtimes \OO_P(\lambda) \to \EE \to
I_{\ZZ} \to 0 
\end{equation}
the universal family of extensions, where
$\OO_P(\lambda)$ is the tautological line bundle on $P$.
Hence we have a morphism 
$\psi:P \to M_\sigma(v)$.


\begin{Lem}
$\theta_{v,\sigma}$ is an isomorphism if $n \geq 2$, or $n=1$ and $D^2=-4$. 
\end{Lem}

\begin{proof}
We fix $\eta \in \Pic(X)$ with $(\eta,D)=1$. 
For $u=x v(\OO_X)+y(0,0,1)+(0,\xi,0)$,
\begin{equation}
u-\langle v,u \rangle(0,\eta,0)=
x v(\OO_X)+b(0,0,1)+(0,\xi-(\xi,D)\eta,0)+
(x(2+\tfrac{D^2}{2}-n)+2y)(0,\eta,0).
\end{equation} 

We set 
$\theta_0(\alpha):=\det p_{\Hilb_X^n !}(I_{\ZZ} \otimes p_X^*(\alpha^{\vee}))$.
Then $\theta_0(k_p)=\OO_{\Hilb_X^n}$.
We have an injective homomorphism
$\Pic(X) \to \Pic(\Hilb_X^n)$ 
by sending $D \in \Pic(X)$ to $\theta_0(\OO_X(D)-\OO_X)$.
We regard $\Pic(X)$ as a subgroup of $\Pic(\Hilb_X^n)$
by this homomorphism.
If $n \geq 2$, then
$\Pic(\Hilb_X^n)=\Z \delta \oplus \Pic(X)$ with 
$\delta:=\theta_0(\OO_X)$.
By \eqref{eq:beta}, we have
\begin{equation}
\begin{split}
p_{P!}(\EE \otimes p_X^*(u^{\vee}))=&
-\langle v(\OO_X(D)),v(u) \rangle\lambda
+\theta_0(u)\\
=&
x(\delta+(n-1)\lambda+(2+\tfrac{D^2}{2}-n)\eta)
-y(\lambda-2\eta)+(\xi-(\xi,D)\eta).
\end{split}
\end{equation}
Since
$$
\Pic(P)=\Pic(X) \oplus \Z \delta \oplus \Z \lambda=
D^\perp \oplus \Z \eta \oplus \Z \delta \oplus \Z \lambda,
$$
$$
\psi^* \circ \theta_{v,\sigma}:K(X)_v \to \Pic(M_H^\beta(v)) \to \Pic(P)
$$
is an isomorphism.

If $n=1$ and $D^2=-4$, then $\rk \VV=n+1=2$ and
$\Pic(P)=\Pic(X) \oplus \Z \lambda$.
Since  
$2+\tfrac{D^2}{2}-n=-1$, 
\begin{equation}
p_{P!}(\EE \otimes p_X^*(u^{\vee}))=
-x \eta-y(\lambda-2\eta)+(\xi-(\xi,D)\eta).
\end{equation}
Hence $\psi^* \circ \theta_{v,\sigma}$
is also an isomorphism.


Therefore $\theta_{v,\sigma} (K(X)_v)$ is a direct summand of $\Pic(M_\sigma(v))$.
Since the torsion submodule of $\Pic(M_\sigma(v))$ is
$\Z /2 \Z$ and $\rk \Pic(M_\sigma(v))=\rk K(X)_v$,
$\theta_{v,\sigma}$ is an isomorphism.
\end{proof}

}


\section{Appendix}\label{App: exceptional case}




\subsection{A connected component of $M_\omega(v,K_X)$ $(v=(2,0,-1))$}



For a Mukai vector $v=(2,0,-1)$, we shall describe a connected component
of $M_\omega(v,K_X)$, which is a refinement of \cite[Remark 2.3]{Yos16a}.

 
Let $\varpi:\widetilde{X} \to X$ be the covering K3 surface.
We set $E_0:=\OO_X \oplus \OO_X(K_X)$.
Let $\iota$ be the covering involution of $\widetilde{X}$.
\begin{Lem}\label{lem:ext^2}
In the open subscheme $M_\omega(v,K_X)^s$ of stable sheaves,
the singular locus is 
\begin{equation}\label{eq:sing}
\{ \varpi_*(I_W(D)) \mid I_Z \in \Hilb^n(\widetilde{X}),\;\iota(D)=-D, n=(D^2)/2+2 \},
\end{equation}
where $D=0$ or $(D^2)=-4$.
In particular, the singular locus consists of isolated singular points in 
the open subscheme of locally free sheaves, and 
an irreducible 4-dimensional subscheme. 
\end{Lem}

\begin{proof}
If $M_\omega(v,K_X)^s$ is singular at $E$, then 
$E(K_X) \cong E$, which implies $E= \varpi_*(I_W(D))$, where 
$I_W$ is the ideal sheaf of a 0-dimensional subscheme $W$ of  $\widetilde{X}$
and $D$ is a divisor on $\widetilde{X}$ (cf. \cite[Lem. 2.13]{Yamada}). 
Since $\varpi^*(\varpi_*(I_W(D)))=I_W(D) \oplus I_{\iota(W)}(\iota(D))$,
$D+\iota(D)=0$ and $\deg W=(D^2)/2+2$.
Hence we see that $(D,\varpi^*(\omega))=0$ and $(D^2) \geq -4$.
Since $D$ is not effective,
we get $D=0$ or $D^2=-4$. 
Therefore 
$$
\dim \{E \in M_\omega(v,K_X)^s \mid E(K_X) \cong E \} \leq 4.
$$
Since $\dim M_\omega(v,K_X)^s=5$,  \eqref{eq:sing} is the singular locus
of $M_\omega(v,K_X)^s$.
In the open subscheme of locally free sheaves,
$Z= \emptyset$, and hence
the singular locus is isolated.
If $D=0$ and $\# W=2$, then 
$\varpi_*(I_W)$ is a non-locally free sheaf
with $\varpi_*(I_W)^{\vee \vee} \cong E_0$,
which gives an irreducible component 
$$
\{ \varpi_*(I_W) \in M_\omega(v,K_X)^s \mid I_W \in \Hilb^2(\widetilde{X}) \}
$$
of dimension 4. 
\end{proof}

\begin{Cor}
$M_\omega(v,K_X)^s$ is normal on the open subscheme of locally free
sheaves.
\end{Cor}

Let us study the 4-dimensional component of the singular locus.
We have an injectve morphism $X \to \Hilb^2(\widetilde{X})$ by sending
$I_z$ $(z \in X)$ to $I_{\varpi^{-1}(z)}$.
The image is the $\iota$-invariant Hilbert scheme $\Hilb^2(\widetilde{X})^{\iota}$.
We shall identify $X$ with  $\Hilb^2(\widetilde{X})^{\iota}$.
We have a morphism
$$
\begin{matrix}
\Hilb^2(\widetilde{X}) & \to & M_\omega(v,K_X)\\
I_W & \mapsto & \varpi_*(I_W),
\end{matrix}
$$
 which is a double covering to
its image.
If $I_W \in \Hilb_{\widetilde{X}}^2 \setminus X$, then
$\varpi_*(I_W)$ is a stable non-locally free sheaf and
the fiber over $\varpi_*(I_W)$ is $\{ I_W, I_{\iota(W)} \}$.
If $I_W=I_{\varpi^{-1}(z)}$, then 
$\varpi_*(I_W)=I_z \otimes E_0$, which is a properly semi-stable
sheaf.

For $E=\varpi_*(I_W)$, we have $-\chi(E,E)/2=2$, and hence
by using
\cite[Fact 2.4]{Yamada} and the paragraph in front of \cite[Lem. 2.13]{Yamada},
we see that 
$M_\omega(v,K_X)^s$ is analytic locally defined by 
a hypersurface  $F(t_1,t_2,...,t_6)=0$ in $(\C^6,0)$
such that 
$$
F(t_1,t_2,...,t_6)=\sum_{i=1}^n t_i^2+G(t_1,t_2,...,t_6),\;\;
n \geq 2,\; G(t_1,t_2,...,t_6) \in (t_1,t_2,...,t_6)^3.
$$
Since the singular locus is 4-dimensional,
$n=2$. 
 Therefore there are at most two irreducible components intersecting
along the 4-dimensional singular locus.

We shall prove that there are exactly two irreducible components $M_0$ and $M_1$
intersecting along the 4-dimensional singular locus. 

 


Let $\psi:M_\omega(v,K_X) \to N$ be the contraction map to the 
Uhlenbeck compactification.
\begin{Lem}\label{lem:M_0}
There is an irreducible component
$M_0$ of $M_\omega(v,K_X)$ 
such that $M_0$ contains a $\mu$-stable locally free sheaf and
$\psi(M_0)$ contains $M_\omega(v',K_X) \times S^2 X$,
where $v'=v(E_0)$.
\end{Lem}

\begin{proof}
Let $\MM_0$ be an irreducible component of the stack $\MM_\omega(v,K_X)^{\mu ss}$
of $\mu$-semi-stable sheaves $E$ with $v(E)=v$ and $c_1(E)=K_X$ 
containing all locally free sheaves fitting in
\begin{equation}
0 \to \OO_X \to F \to I_Z (K_X) \to 0,
\end{equation}
where $I_Z \in \Hilb^2(X)$ (see that proof of \cite[Lem. 2.8]{Yos16a}).
Then  
$$
\MM_0':=\{ E \in \MM_0 \mid \text{$E$ is a $\mu$-stable locally free sheaf}\}
$$
is an open and dense substack of $\MM_0$.
Let $M_0$ be the irreducible component of $M_\omega(v,K_X)$
containing the associated coarse moduli scheme of $\MM_0'$.
We shall prove that $\psi(M_0)$ contains $M_\omega(v',K_X) \times S^2 X$,
where $v'=v(E_0)$.
Let $F$ be a locally free sheaves fitting in
\begin{equation}
0 \to \OO_X \to F \to I_Z (K_X) \to 0,
\end{equation}
where $I_Z \in \Hilb^2(X)$.
Then we have a deformation $\FF_t$ over a curve $T$ such that $\FF_0=F$ and
$\FF_t$ ($0 \ne t \in T$) are $\mu$-stable locally free sheaves.
Then there is a family of semi-stable sheaves
$\FF_t'$ such that $\FF_t=\FF_t'$ for $t \ne 0$.
Thus we have a morphism $T \to M_0$.
Since $\psi$ is extended to the family of $\mu$-semi-stable sheaves
$\FF_t$, 
\begin{equation}
\psi(\FF_0')=\lim_{t \to 0} \psi(\FF_t)=\psi(\FF_0)=
(E_0, [Z]) \in M_\omega(v',K_X) \times S^2 X,
\end{equation}
where 
$[Z]$ is the 0-cycle defined by $Z$.
Hence our claim holds.
\end{proof}




For the construction of another irreducible component $M_1$,
we prepare the following lemma.
\begin{Lem}
For a non-trivial extension
\begin{equation}\label{eq:(2,K,-1)}
0 \to I_Z \to E \to \OO_X(K_X) \to 0,
\end{equation}
$E$ is a semi-stable sheaf, where
$Z \in \Hilb^2(X)$.
\end{Lem}

\begin{proof}
Let $I$ be a subsheaf with $c_1(I)=0,K_X$.
If $c_1(I)=0$, then $I$ is a subsheaf of $I_Z$.
If $c_1(I)=K_X$, then we have an injective homomorphism
$\phi:I \to \OO_X(K_X)$.
Since \eqref{eq:(2,K,-1)} is a non-trivial extension,
$\phi$ is not an isomorphism.
Hence $E$ is semi-stable. 
\end{proof}


We note that 
$I_Z[1]$ is the cone of $\OO_X \to \OO_Z$.
By the triangle
\begin{equation}
\OO_Z \to I_Z[1] \to \OO_X[1] \to \OO_Z[1] 
\end{equation}
and $\Hom(\OO_X(K_X),\OO_X[1])=0$,
the extension class $e \in \Hom(\OO_X(K_X),I_Z[1])$
is represented as a homomorphism $f$ of complexes: 
\begin{equation}\label{eq:cone(f)}
\begin{CD}
@. \OO_X\\
@. @VVV\\
\OO_X(K_X) @>>> \OO_Z
\end{CD}
\end{equation}
Thus $E=\Cone(f)[-1]=\ker(\OO_X \oplus \OO_X(K_X) \to \OO_Z)$.
Then $E$ is properly semi-stable if and only if
$\OO_X(K_X) \to \OO_Z$ is not surjective.

Since $\Hom(\OO_X(K_X),I_Z)=0$, we have a family
of semi-stable sheaves 
\begin{equation}
0 \to I_{\ZZ} \otimes \OO_{\P}(1) \to \EE \to \OO_{\P \times X}(K_X) \to 0
\end{equation}
on the projective bundle
$$
q:\P:=\P(\Ext^1_p(\OO_{\Hilb^2(X) \times X}(K_X),I_{\ZZ})) \to \Hilb^2(X),
$$
where $\ZZ$ is the universal family on $\Hilb^2(X) \times X$
and $p:\Hilb^2(X) \times X \to \Hilb^2(X)$ is the projection.
Then we have a morphism 
\begin{equation}
\begin{matrix}
g:& \P & \to & M_\omega(v,K_X)\\
& z & \mapsto & \EE_z, \\
\end{matrix}
\end{equation}
 which is injective
on the stable locus. 
Indeed for a stable sheaf $E$,  the exact sequence \eqref{eq:(2,K,-1)}
is uniquely determined by $E$.
Let us consider the locus of properly semi-stable sheaves.
Assume that $Z=\{z_1,z_2 \}$ $(z_1 \ne z_2)$.
If $\OO_X(K_X) \to \OO_Z$ is not surjective at $z_1$, then
$E$ fits in an exact sequence
$$
0 \to I_{z_2}(K_X) \to E \to I_{z_1} \to 0
$$ 
which gives the point $I_{z_2}(K_X) \oplus I_{z_1}$ of $M_\omega(v,K_X)$.
Therefore $g$ is injective if $z_1 \ne z_2$.
If $Z=\{z \}$ with a non-reduced structure
and $\OO_X(K_X) \to \OO_Z$ is not surjective, then
the image is $I_z \oplus I_z(K_X)$.
Thus the image is independent of the choice of a scheme structure of $Z$, and
$g$ is not injective on the fiber.
\begin{Lem}\label{lem:M_1}
We set $M_1:=g(\P)$.
Then $M_1=\psi^{-1}(M_\omega(v',K_X) \times S^2 X)$ and 
is an irreducible component of $M_\omega(v,K_X)$. 
\end{Lem}

\begin{proof}
We already proved that the set  
$$
\{I_{z_1} \oplus I_{z_2}(K_X) \mid z_1,z_2 \in X\}
$$
of properly semi-stable
sheaves is contained in $M_1$.
If $E \in M_\omega(v,K_X)^s$ satisfies
$\psi(E) \in M_\omega(v',K_X) \times S^2 X$, then
$E$ is a properly $\mu$-semi-stable sheaf with 
$E^{\vee \vee}=E_0$, and hence $E$ fits in the exact sequence
\eqref{eq:(2,K,-1)}.
Therefore the claim holds.
\end{proof}


Let us show that $M_0 \cup M_1$ is a connected component of
$M_\omega(v,K_X)$.

 
\begin{Lem}\label{lem:normal}
The local deformation space is smooth at $E$ if
$E$ is $S$-equivalent to $I_{z_1} \oplus I_{z_2} (K_X)$ ($z_1 \ne z_2$).
Hence $I_{z_1} \oplus I_{z_2} (K_X)$ ($z_1 \ne z_2$)
is contained in the normal open subscheme
$$
M_\omega(v,K_X)^*:=\{ E \in M_\omega(v,K_X) \mid \Ext^2(E,E)=0\}
$$
of $M_\omega(v,K_X)$.
\end{Lem}

\begin{proof}
If $z_1 \ne z_2$, then
$$
\Ext^2(I_{z_1},I_{z_1})=\Ext^2(I_{z_1}, I_{z_2} (K_X))=
\Ext^2(I_{z_2}(K_X),I_{z_1})=\Ext^2(I_{z_2}(K_X),I_{z_2}(K_X))=0.
$$
Hence $\Ext^2(E,E)=0$.
Let $Q$ be an open subscheme of a suitable quot-scheme
$\Quot_{\OO_X(-mH)^{\oplus N}/X}$
parameterizing quotients
$\OO_X(-mH)^{\oplus N} \to E$
such that $M_\omega(v,K_X)$ is a GIT-quotient of
$Q$ by $PGL(N)$.
Let $Q^*$ be the open subscheme of $Q$
such that $\Ext^2(E,E)=0$. Then $Q^*$ is smooth.
Hence the open subscheme $M_\omega(v,K_X)^*=Q^*/PGL(N)$ of
$M_\omega(v,K_X)$ is a normal scheme.  
\end{proof}

\begin{Lem}\label{lem:M^*} 
$\P \times_{M_\omega(v,K_X)} M_\omega(v,K_X)^* 
\to M_\omega(v,K_X)^*$ is a closed immersion whose
image is a connected component of $M_\omega(v,K_X)^*$.
In particular $M_\omega(v,K_X)^* $ is smooth.
\end{Lem}

\begin{proof}
Since $g$ is a proper morphism,
$g':\P \times_{M_\omega(v,K_X)} M_\omega(v,K_X)^*  \to M_\omega(v,K_X)^*$ is also proper.
We note that the image of $g'$ is $M_1^*:=M_1 \cap M_\omega(v,K_X)^*$,
which is a connected component of $M_\omega(v,K_X)^*$ by Lemma \ref{lem:normal}.
If $E$ is $S$-equivalent to $I_{z} \oplus I_z(K_X)$, then
we see that $\Ext^2(E,E) \ne 0$. Hence $E$ is not contained in 
$M_\omega(v,K_X)^*$.
Therefore $g'$ is injective, which implies
$g'$ is a finite map between normal schemes.
Obviously $g':\P \times_{M_\omega(v,K_X)} M_\omega(v,K_X)^*  \to M_1^*$ 
is a birational map, which implies $g'$ is an isomorphism.
\end{proof}



\begin{Prop}\label{prop:connected}
\begin{enumerate}
\item[(1)]
The singular locus of $M_\omega(v,K_X)$ consists of a 4-dmensional subscheme
$$
S_1:=\{ \varpi_*(I_W) \mid I_W \in \Hilb^2(\widetilde{X}) \} 
$$
and a finite set of points 
$$
S_2:=\{\varpi_*({\cal O}_{\widetilde{X}}(D)) \mid \iota(D)=-D, (D^2)=-4\}.
$$
\item[(2)]
$M_0$ and $M_1$
intersect along $S_1$.
\item[(3)]
$M_0 \cup M_1$ is a connected component of $M_\omega(v,K_X)$.
\end{enumerate}
\end{Prop}

\begin{proof}
(1) follows from Lemma \ref{lem:ext^2} and Lemma \ref{lem:M^*}. 
By Lemma \ref{lem:M_0} and Lemma \ref{lem:M_1},
$M_0$ intersects $M_1$ along the 4-dimensional singular locus.
By the description of the singular locus,
$M_0 \cup M_1$ is a connected component of
$M_\omega(v,K_X)$.
\end{proof}


\begin{Rem}
$M_0 \cap M_1 \to M_\omega(v',K_X) \times S^2 X$ is a double cover.
Indeed for $\varpi_*(I_{w_1,w_2})$ $(w_1 \ne w_2)$,
$\psi^{-1}(\psi(\varpi_*(I_{w_1,w_2})))=\{\varpi_*(I_{w_1,w_2}),\varpi_*(I_{w_1,\iota(w_2)})\}$.
\end{Rem}









\subsection{A remark on $M_\omega(v,0)$ $(v=(2,0,-1))$}
\begin{Lem}
If $E \in M_\omega(v,0)$ is not a $\mu$-stable locally free sheaf, then 
$E$ is $S$-equivalent to $I_{z_1} \oplus I_{z_2}$ or
$I_{z_1}(K_X) \oplus I_{z_2}(K_X)$.  
\end{Lem}

\begin{proof}
If $E$ is not locally free, then we see that
$\chi(E^{\vee \vee})>0$.
Hence $H^0(E^{\vee \vee}) \ne 0$ or
$\Hom(E^{\vee \vee},\OO_X(K_X)) \ne 0$.
Then there is an exact sequence
$$
0 \to I_Z(D) \to E \to I_W(D) \to 0
$$
such that $D=0,K_X$.
By the semi-stability of $E$,
$\deg W=0,1$.
If $\deg W=0$, then
$E^{\vee \vee}=\OO_X(D)^{\oplus 2}$.
In this case, there is an exact sequence
$$
0 \to I_{z_1}(D) \to E \to I_{z_2}(D) \to 0.
$$
Indeed for a torsion free sheaf $F$ with
$E \subset F \subset E^{\vee \vee}$ and $F/E=k_{z_1}$,
$\Hom(\OO_X(D),F) \ne 0$. 
Hence $I_{z_1}(D)$ is a subsheaf of $E$.
Therefore $E$ is $S$-equivalent to $I_{z_1}(D)\oplus I_{z_2}(D)$.
\end{proof}


Thus the complement of the open subset of $\mu$-stable locally free sheaves
in $M_\omega(v,0)$ is parametrized by
two copies of $S^2 X$, and this locus is not contracted by the
morphism to the Uhlenbeck compactification.
 



\subsubsection{}



In this subsection, we shall prove the following claim. 
\begin{Prop}\label{prop:irred-comp:v^2=2}
Let $v:=(r,\xi,a)$ be a Mukai vector such that $v^2=2$.
Assume that $L \equiv D+\frac{r}{2}K_X \mod 2$, where $D$ is a nodal cycle.
Then $M_\omega(v,L)$ has two irreducible components.
\end{Prop}



\begin{Lem}\label{lem:e-poly}
Let $v:=(r,\xi,a)$ be a Mukai vector such that $v^2$ is even.
\begin{enumerate}
\item[(1)]
There is $v'=(2,L',a')$ such that
$e(M_\omega(v,L))=e(M_\omega(v',L'))$,
where $L'+K_X \equiv L+\frac{r}{2}K_X \mod 2$.
\item[(2)]
For a smooth rational curve $C$,
we have an equality
\begin{equation}\label{eq:(-2)}
e(M_\omega(v,L))= e(M_{\omega'}(v',L')),
\end{equation}
where $\sigma'=\Phi(\sigma)$, 
$v'=(r,\xi+(C,\xi)C,a)$ and $L'=L+(L,C)C$.
\end{enumerate}
\end{Lem}

\begin{proof}
Let $\Stab(X)^*$ be the connected component of $\Stab(X)$
containing geometric stability conditions.

(1)
Then the $(-1)$-reflection $R_{\OO_X}$ associated to
$\OO_X$, $\Stab(X)^*$ is preserved, and
we have an isomorphism
\begin{equation}
M_\sigma(v,L))\cong M_{\sigma'}(v',L').
\end{equation}
where $\sigma'=R_{\OO_X}(\sigma)$, 
$v'=R_{\OO_X}(v)=(-2a,\xi,-r/2)$ 
and $L'=L+\tfrac{r+2a}{2}K_X$.
By \cite{Nue14a},
we have
\begin{equation}
e(M_\omega(v,L))= e(M_{\omega'}(v',L')).
\end{equation}
By similar arguments as in \cite{Nue14a} or \cite{Yos16a},
we get (1).

(2)
Let $\Phi:{\bf D}(X) \to {\bf D}(X)$ be the twist functor
associated to a spherical object $\OO_C(-1)$.
By the action of $\Phi$,
$\Stab(X)^*$ is preserved, and
we have an equality
\begin{equation}
M_\sigma(v,L) \cong M_{\sigma'}(v',L)
\end{equation}
where $\sigma'=\Phi(\sigma)$, 
$v'=(r,\xi+(C,\xi)C,a)$ and $L'=L+(L,C)C$.
By using \cite{Nue14a} again,
we get \eqref{eq:(-2)}.
\end{proof}


\begin{Lem}\label{lem:rank2}
Let $v=(r,\xi,a)$ be a Mukai vector with
$v^2=2$ and $L$ a divisor such that $L \equiv D+\frac{r}{2}K_X \mod 2$,
where $D$
is a nodal cycle.
Then there is a Mukai vector $v'=(2,\xi',0)$
and an elliptic fibration $\pi:X \to \P^1$
such that 
\begin{enumerate}
\item
$e(M_\omega(v,L))=e(M_{\omega'}(v',L'))$,
\item
$(\xi',C)=2$ for a general fiber of $\pi$
and $L' \equiv D' +K_X \mod 2$,
where $D'$ is a nodal cycle.
\end{enumerate}
\end{Lem}

\begin{proof}
For a primitive Mukai vector $v$ such that $\ell(v)=1$ and $v^2$ is even,
we take $v'=(2,\xi',a')$ and $L'$ satisfying
Lemma \ref{lem:e-poly}.
By $\ell(v)=1$, we can find an isotropic Mukai vector
$\eta$ such that
$(\xi'+2\lambda,\eta)=1$, where $\lambda \in \NS(X)$.   
Replacing $v'$ by $v' e^{k \eta+\lambda}$, we may assume that 
$({\xi'}^2)=2$ and $a'=0$.
Then $\xi'$ is effective or $-\xi'$ is effective.
Since $v' e^{-\xi'}=(2,-\xi',a')$, we may assume that $\xi'$ is effective.
Since $L \equiv D+\frac{r}{2}K_X \mod 2$ for a nodal cycle $D$, 
By using Lemma \ref{lem:nodal} and \eqref{eq:(-2)}, 
we may assume that $v'$ and $L'$ 
satisfy
$L' \equiv D'+K_X \mod 2$ for a nodal cycle $D'$.

Then for an isotropic divisor $\eta$ with $(\xi',\eta)=1$,
Riemann-Roch theorem implies
$\eta$ is effective.
Let $\eta=f+\sum_i C_i$ be a decomposition of $\eta$ such that 
$f$ is a nef and isotropic divisor, and $C_i$ are smooth rational curves.
Then $(\xi',f), (\xi',C_i) \geq 0$ imply that
$(f,\xi')=1$ by using Hodge index theorem.
By the linear system
$|2f|$, we have an elliptic fibration $\pi:X \to \P^1$
satisfying our requirements. 
\end{proof}


\begin{Lem}\label{lem:nodal}
For a nodal cycle $D$ and $D'$, 
 $D+(D,D')D' \equiv D'' \mod 2$, where $D''$ is a nodal cycle. 
\end{Lem}

\begin{proof}
For nodal cycles $D$ and $D'$,
let $E$ be a stable vector bundle with $v(E)=(2,D+K_X,0)$
and
$E'$ a stable vector bundle with $v(E')=(2,D'+K_X,0)$.
Then for $E(nH)$ $(n \gg 0)$,
$$
F:=\ker (\Hom(E',E(nH)) \otimes E' \to E(nH))
$$
is a stable vector bundle
with $v(F)=e^{nH}(2,D+K_X,0)+\langle e^{nH}v,v' \rangle v'$.
Since 
\begin{equation}
\langle e^{nH}v,v' \rangle=(D,D')-2n^2(H^2)+2n(H,D'-D),
\end{equation}
we get
\begin{equation}
\begin{split}
\frac{\rk F}{2} \equiv & 1+(D,D')  \mod 2,\\
c_1(F) \equiv & D+K_X+(D,D')(D'+K_X) \mod 2\\
\equiv &
D+(D,D')D'+\frac{\rk F}{2}K_X \mod 2.
\end{split}
\end{equation}
Hence $D+(D,D')D' \equiv D'' \mod 2$, where $D''$ is a nodal cycle.
\end{proof}






By Lemma \ref{lem:rank2}, it is sufficient to 
describe the irreducible components of $M_\omega(v,L)$
for $v=(2,\xi,0)$ with $(\xi^2)=2$ and $L$ such that 
$L \equiv D +K_X \mod 2$ and $(L,C)=2$
for a general fiber $C$ of an elliptic fibration
$\pi:X \to \P^1$.
For this purpose, we first describe a 2-dimensional component
of the singular locus of $M_\omega(v,L)$.
   
\begin{Lem}\label{ext^2-2}
For a Mukai vector $v=(2,\xi,0)$ with $(\xi^2)=2$ and a divisor $L$
with $L \equiv D+K_X \mod 2$, 
the singular locus of $M_\omega(v,L)$ is
\begin{equation}
\{ \varpi_*(I_W(\widetilde{L})) 
\mid I_W \in \Hilb^n(\widetilde{X}),\; 
\iota(\widetilde{L})=\varpi^*(L)-\widetilde{L},\;
n=(\widetilde{L}^2)/2+1\}
\end{equation}
where $(\widetilde{L}^2)=0,-2$.
In particular the 2-dimensional component of the singular locus is
irreducible.
\end{Lem}


\begin{proof} 
Since $M_\omega(v,L)$ is of expected dimension,
the singular locus is
$$
\{ E \in M_\omega(v,L) \mid \Ext^2(E,E) \ne 0\}.
$$ 
If $E(K_X) \cong E$, then
we see that $E \cong \varpi_*(I_W(\widetilde{L}))$, 
where $W=\emptyset$ or $W=\{w \}$.
If $W=\{w \}$, then 
$E^{\vee \vee}=\varpi_*(\OO_{\widetilde{X}}(\widetilde{L}))$
is a spherical vector bundle with $v(E^{\vee \vee})=(2,L,1)$.
Therefore the claim holds.
\end{proof}

\begin{Rem}
For $x \in X$, we set $\varpi^{-1}(x)=\{z,\iota(z)\}$.
Then $\varpi_*(I_z(\widetilde{L}))$ and 
$\varpi_*(I_{\iota(z)}(\widetilde{L}))$
are not locally free at $x$.
Hence the 2-dimensional component of the singular locus
is a double covering of $X$. 
\end{Rem}





\begin{Lem}\label{lem:irred-comp-2}
There are at most two irreducible components of
$M_\omega(v,L)$.
\end{Lem}

\begin{proof}
If there are two irreducible components of $M_\omega(v,L)$, then
the connectedness of $M_\omega(v,L)$ implies they intersect along 
the 2-dimensional component of the singular locus. 
By \cite{Yamada},
the analytic germ of $M_\omega(v,L)$ is described as a hypersurface
$F(x_1,x_2,x_3)=0$ in $(\C^3,0)$ with a non-trivial quadratic term. 
If $M_\omega(v,L)$ is reducible, then each irreducible component is defined
by a factor of $F(x_1,x_2,x_3)$.
Therefore $M_\omega(v,L)$ has at most two irreducible components.
\end{proof}

Let $f$ be the reduced part of a multiple fiber of $\pi$ and
$E_0 \in M_\omega(v',L-2f)$ be a spherical vector bundle,
where $v'=(2,\xi-2f,0)$.
We shall prove Proposition \ref{prop:irred-comp:v^2=2}
by constructing two irreducible components $M_0$ and $M_1$
of $M_\omega(v,L)$. 

Let $M_0$ be an irreducible component
parametrizing $E$ fitting in an exact sequence
\begin{equation}
0 \to E \to E_0(f) \to \C_x \to 0, \; x \in X.
\end{equation}
Thus it is a $\P^1$-bundle over $X$.




\begin{Lem}
Let $C$ be a general fiber of the elliptic fibration.
Then $E_{0|C} \cong \OO_C(p) \oplus \OO_C(q)$ $(p \ne q)$.
\end{Lem}

\begin{proof}
For a primitive and isotropic Mukai vector $u:=(0,C,1)$,  
$M_\omega(u,C)$ is a fine moduli space
which is isomorphic to $X$, where $\omega$ is general.
We have an elliptic fibration $M_\omega(u,C) \to \P^1$ and
$M_\omega(u,C)$ is regardes as a smooth compactification of a relative 
Picard scheme $\Pic^1(X/\P^1) \to \P^1$ of degree 1.
We shall identify $M_\omega(u,C) \to \P^1$ with
the elliptic fibration $X \to \P^1$.

For a universal family $\EE$, we have a relative Fourier-Mukai transform
$\Phi_{X \to X}^{\EE^{\vee}}: {\bf D}(X) \to {\bf D}(X)$.
Then $F:=\Phi_{X \to X}^{\EE^{\vee}}(E_0)[1]$ is a purely 1-dimensional sheaf
whose support is a double cover of $\P^1$.
Then for a general fiber $C$ with $F_{|C}=\C_p \oplus \C_q$,
$E_{0|C} \cong \OO_C(p) \oplus \OO_C(q)$.
\end{proof}

Let $L$ be a line bundle on a smooth fiber $C \in |2f|$ with $v(L)=(0,2f,0)$.
We set $L^*:=\EE xt^1_{\OO_X}(L,\OO_X)$.
Then $L^*$ is a line bundle of degree 0 on $C$.
\begin{Lem}\label{Lem:M_1}
For a non-zero homomorphism
$\psi:E_0^{\vee} \to L^*$, 
we set $E:=\R \HH om_{\OO_X}(\Cone \psi,\OO_X)[1]$.
Then $E$ is a $\mu$-stable torsion free sheaf with respect to
$H_0+nf$ $n \gg 0$.
If $\psi$ is surjective, then $E$ is a locally free sheaf.
\end{Lem}

\begin{proof}
We have an exact sequence
\begin{equation}
0 \to E_0 \to E \to L \to 0.
\end{equation} 
Since $\psi$ is surjective in codimensin 1 and
$\Cone(\psi)$ is represented by a 2-term complex of locally free sheaves,
$E$ is torsion free.
Let $F$ be a subsheaf of $E$ with $\rk F=1$.
Then $E_0 \cap F$ is a  rank 1 subsheaf of $E_0$.
Since $E_0$ is $\mu$-stable for any ample divisor,
$$
2(c_1(E_0 \cap F),H_0+nf)<(c_1(E_0),H_0+nf),\;n \geq 0.
$$
Hence $2(c_1(E_0 \cap F),f) \leq (c_1(E_0),f)=1$ and
$2(c_1(E_0 \cap F),H_0)<(c_1(E_0),H_0)$.
In particular
$(c_1(E_0 \cap F),f) \leq 0$.
Then $(c_1(F),f) \leq 0$.
If $n>(2f-c_1(E_0),H_0)$, then $2(c(F),H_0+nf)<(c_1(E),H_0+nf)$.
Therefore $E$ is $\mu$-stable.
\end{proof}

\begin{Cor}
For $\omega=H_0+n f$ ($n \gg 0$),
there is an irreducible component $M_1$ of $M_\omega(v,L)$ 
which contains a $\mu$-stable locally free sheaves.
\end{Cor}



By Lemma \ref{lem:irred-comp-2}, $M_0$ and $M_1$ are the irreducible components
of $M_\omega(v,L)$, which shows
Proposition \ref{prop:irred-comp:v^2=2} by Lemma \ref{lem:rank2}.

















\bibliographystyle{plain}
\bibliography{NSF_Research_Proposal}

\end{document}


\subsection{Isotropic walls with no spherical or exceptional classes}

We begin our investigation with (what will likely be) the least involved case: when the isotropic lattice $\HH$ contains no spherical or exceptional classes.  In the notation of Lemma \ref{Lem:isotropic lattice}, let us observe that any object in $M_{\sigma_+}(w_i)$, $i=1,2$, remains $\sigma_0$-stable as the assumption on $\HH$ forces the $w_i$ to generate the extremal rays of $C_{\WW}=P_{\HH}$.  We collect another useful observation in the following simple result:
\begin{Lem}\label{Lem:product bigger than 1}
Suppose that $\HH$ is isotropic but contains no spherical or exceptional classes.  Then $\langle w_1,w_2\rangle\geq 2$ and $\langle v,w_i\rangle\geq 2$ for any positive class $v$.
\end{Lem}
\begin{proof}
If $\langle w_1,w_2\rangle=1$, then $w_1-w_2$ is a spherical class.  Similarly, if $\langle v,w_i\rangle=1$ for some $i$, then $\frac{v^2+1}{2}w_i-v$ is an exceptional class if $v^2$ is odd while $\frac{v^2+2}{2}w_i-v$ is a spherical class if $v^2$ is even.  
\end{proof}

Now we come to the description of the wall-crossing behavior for isotropic lattices $\HH$ with no spherical or exceptional classes.

\begin{Lem}\label{Lem:no spherical or exceptional divisorial contraction}
Suppose that $v^2>4$ and $\langle v,w\rangle=2$ for a primitive, effective, isotropic $w\in\HH$ satisfying $l(w)=2$ and $M_{\sigma_0}(w)=M^s_{\sigma_0}(w)$.  Then $\WW$ induces a divisorial contraction.
\end{Lem}
\begin{proof}

\end{proof}



\begin{Prop}
Suppose that the isotropic lattice $\HH$ contains no spherical or exceptional classes.  Let $v$ be a positive class.  Then $\WW$ can only be totally semistable if $v=w_1+w_2$, $l(w_1)=l(w_2)=2$, and $\langle w_1,w_2\rangle=2$, in which case $\WW$ induces a $\P^1$ fibration.  Furthermore, $\WW$ induces a divisorial contraction if and only if 1) $v^2\geq 4$, for some $i$ $\langle w_i,v\rangle=2$ and $l(w_i)=2$, or 2) $v=w_1+w_2$, $l(w_1)=l(w_2)=2$, and $\langle w_1,w_2\rangle=3$.  Otherwise, $\WW$ is a flopping wall, a fake wall, or not a wall at all.
\end{Prop}
\begin{proof}
First we note that any substack $\FF(a_1,..,a_n)^0$ of objects with Harder-Narasimhan filtration factors of classes $a_1,...,a_n$ such that $a_i^2>0$ for all $i$ satisfies $\codim \FF(a_1,...,a_n)^0\geq 2$ by Proposition \ref{Prop:HN filtration all positive classes}, so we may assume some of the $a_i$ are isotropic, necessarily $a_1$ or $a_n$ as the $a_i$ are ordered according to $\phi_-$.

Now let us break the proof into cases.

\textbf{Case 1:} Suppose that $a_1=b_1w_1,a_n=b_2w_2$, with $b_i\geq 1$, and $l(w_1)=l(w_2)=1$.  Then let $l=n-2$, so we have \begin{align}\label{eq:case 1}
\begin{split}
\codim\FF(a_1,...,a_n)^0&=\sum_{i=1}^n(a_i^2-\dim\MM_{\sigma_-}(a_i))+\sum_{i<j}\langle a_i,a_j\rangle\\
&\geq-\left\lfloor\frac{b_1}{2}\right\rfloor-\left\lfloor\frac{b_2}{2}\right\rfloor+b_1 b_2\langle w_1,w_2\rangle+\sum_{1<i<n}b_1\langle w_1,a_i\rangle+\sum_{1<i<n}b_2\langle w_2,a_i\rangle\\
&+\sum_{1<i<j<n}\langle a_i,a_j\rangle\\
&\geq\frac{b_1(2b_2-1)+b_2(2b_1-1)}{2}+2b_1 l+2b_2 l+l^2-l\geq 2,
\end{split}
\end{align}
if $l\geq 1$, where the second inequality follows from $\langle w_1,w_2\rangle,\langle w_i,a_i\rangle,\langle a_i,a_j\rangle\geq 2$ as in Lemma \ref{Lem:product bigger than 1}.  If $l=0$, then $$\codim\FF(a_1,...,a_n)^0\geq\frac{b_1(2b_2-1)+b_2(2b_1-1)}{2}\geq 1,$$ with equality in the second inequality achieved only if $b_1=b_2=1$.  But then the first inequality of \eqref{eq:case 1} becomes $$\codim\FF(a_1,...,a_n)^0\geq \langle w_1,w_2\rangle\geq 2.$$

\textbf{Case 2:} Suppose that $a_1=b_1w_1,a_n=b_2w_2$, with $b_i\geq 1$, and $l(w_1)=l(w_2)=2$.  Again letting $l=n-2$, we have \begin{align}\label{eq:case 2}
\begin{split}
\codim\FF(a_1,...,a_n)^0&=\sum_{i=1}^n(a_i^2-\dim\MM_{\sigma_-}(a_i))+\sum_{i<j}\langle a_i,a_j\rangle\\
&\geq-b_1-b_2+b_1 b_2\langle w_1,w_2\rangle+\sum_{1<i<n}b_1\langle w_1,a_i\rangle+\sum_{1<i<n}b_2\langle w_2,a_i\rangle\\
&+\sum_{1<i<j<n}\langle a_i,a_j\rangle\\
&\geq b_1(b_2-1)+b_2(b_1-1)+2b_1 l+2b_2 l+l^2-l\geq 2,
\end{split}
\end{align}
if $l\geq 1$.  If $l=0$, then $$\codim\FF(a_1,...,a_n)^0\geq b_1(b_2-1)+b_2(b_1-1)\geq 2,$$ unless $v=w_1+w_2,2w_1+w_2$, or $w_1+2w_2$.  In the latter two cases, we get $\codim\FF(a_1,...,a_n)^0>1$ if $\langle w_1,w_2\rangle>2$, while $\codim\FF(a_1,...,a_n)^0=1$ if $\langle w_1,w_2\rangle=2$.  If $v=2w_1+w_2$, then $\langle v,w_1\rangle=2$, and similarly, if $v=w_1+2w_2$, then $\langle v,w_2\rangle=2$.  Then by Lemma \ref{Lem:no spherical or exceptional divisorial contraction} $\WW$ induces a divisorial contraction.  

If $v=w_1+w_2$, then clearly $\codim\FF(a_1,...,a_n)^0\geq 2$ if $\langle w_1,w_2\rangle>3$, while  $\codim\FF(a_1,...,a_n)=1$ if $\langle w_1,w_2\rangle=3$, and $\codim\FF(a_1,...,a_n)=0$ if $\langle w_1,w_2\rangle=2$.  Indeed, for these latter two claims, notice that $M_{\sigma_0}(w_i)=M^s_{\sigma_0}(w_i)$ has dimension 2 for each $i$ because $l(w_i)=2$, so stability guarantees that $\ext^1(F_2,F_1)=\langle w_1,w_2\rangle$ for any $F_i\in M_{\sigma_0}(w_i)$.  Thus those non-trivial extensions $E$ fitting into a short exact sequence $$0\to F_1\to E\to F_2\to 0$$ are $\sigma_+$-stable by \cite[Lemma 6.9]{BM14b} and sweep out a locus of dimension $$\dim M_{\sigma_0}(w_1)+\dim M_{\sigma_0}(w_2)+\dim\P\Ext^1(F_2,F_1)=2+2+(\langle w_1,w_2\rangle-1)=3+\langle w_1,w_2\rangle,$$ which is $\dim M_{\sigma_+}(v)$ if $\langle w_1,w_2\rangle=2$ and $\dim M_{\sigma_+}(v)-1$ if $\langle w_1,w_2\rangle=3$, as claimed.  Furthermore, as $\ext^1(F_2,F_1)>1$ in each case, $\WW$ contracts curves of non-isomorphic $S$-equivalent objects ($\P^1$'s in the former case and $\P^2$'s in the latter).

\textbf{Case 3:} Suppose that $a_1=b_1w_1,a_n=b_2w_2$, with $b_i\geq 1$, and $l(w_1)\neq l(w_2)$.  Then without loss of generality, we may assume $l(w_1)=1,l(w_2)=2$.  Again letting $l=n-2$, we have \begin{align}\label{eq:case 3}
\begin{split}
\codim\FF(a_1,...,a_n)^0&=\sum_{i=1}^n(a_i^2-\dim\MM_{\sigma_-}(a_i))+\sum_{i<j}\langle a_i,a_j\rangle\\
&\geq-\left\lfloor\frac{b_1}{2}\right\rfloor-b_2+b_1 b_2\langle w_1,w_2\rangle+\sum_{1<i<n}b_1\langle w_1,a_i\rangle+\sum_{1<i<n}b_2\langle w_2,a_i\rangle\\
&+\sum_{1<i<j<n}\langle a_i,a_j\rangle\\
&\geq \frac{2b_2(b_1-1)+b_1(2b_2-1)}{2}+2b_1 l+2b_2 l+l^2-l\geq 2,
\end{split}
\end{align}
if $l\geq 1$. If $l=0$, then as $\frac{2b_2(b_1-1)+b_1(2b_2-1)}{2}\geq\frac{1}{2}$, we still have $\codim\FF(a_1,...,a_n)^0\geq 1$.  Moreover, $\frac{2b_2(b_1-1)+b_1(2b_2-1)}{2}\geq\frac{3}{2}$ unless $b_1=b_2=1$.  Thus $\codim\FF(a_1,...,a_n)^0=1$ can only happen if $v=w_1+w_2$ with $\langle w_1,w_2\rangle=2$, in which case $\langle v,w_2\rangle=2$ and $l(w_2)=2$ so that $\WW$ induces a divisorial contraction by Lemma \ref{Lem:no spherical or exceptional divisorial contraction}.

\textbf{Case 4:} Now suppose that $a_i$ is isotropic for only one $i$, say $a_1=b_1w_1$ (the case $a_n=b_2w_2$ being entirely analogous).  Then letting $l=n-1$ and noting that $l>0$, we get \begin{align}\label{eq:case 4}
\begin{split}
\codim\FF(a_1,...,a_n)^0&=\sum_{i=1}^n(a_i^2-\dim\MM_{\sigma_-}(a_i))+\sum_{i<j}\langle a_i,a_j\rangle\\
&\geq-b_1+\sum_{1<i}b_1\langle w_1,a_i\rangle+\sum_{1<i<j}\langle a_i,a_j\rangle\\
&\geq-b_1+2b_1 l+l^2-l=(2l-1)b_1+l^2-l\geq 2,
\end{split}
\end{align}
unless $l=b_1=1$.  Moreover, if $l(w_1)=1$ then $\dim\MM_{\sigma_-}(w_1)=0$, so $$\codim\FF(a_1,...,a_n)^0\geq 2l+l^2-l\geq 2.$$  Thus we can only get $\codim\FF(a_1,...,a_n)^0=1$ in this case if $l=b_1=1$,$l(w_1)=2$, and $\langle w_1,v-w_1\rangle=2$.  But this gives $\langle v,w_1\rangle=2$, so again by Lemma \ref{Lem:no spherical or exceptional divisorial contraction} we get a divisorial contraction.  

Finally, we observe that the first case for a divisorial contraction stated in the proposition must give one of the situations described in Cases 2-4 above.  Indeed, if for example $l(w_1)=2$ and $\langle v,w_1\rangle=2$, then $(v-w_1)^2\geq 0$.  If $v^2>4$ then we must be in Cases 2 or 4.  If $v^2=4$, then $v-w_1=bw_2$ for some $b>0$.  As $$2=\langle v-w_1,w_1\rangle=\langle bw_2,w_1\rangle=b\langle w_1,w_2\rangle\geq 2b\geq 2,$$ we must have $b=1$ and we are in Cases 2 or 3.
\end{proof}


\subsubsection{The case where $\ell(u_1)=2$}
Assume that $\ell(u_1)=2$, and hence
 $\ell(u_2)=2$.
\begin{Lem}\label{Lem:FM:rank}
Let $v$ be a Mukai vector such that $v^2$ is odd.
Then  
$\langle v, u_1 \rangle$ is odd.
\end{Lem}

\begin{proof}
Since $\ell(u_1)=2$, there is an isometry $\Phi$ of the Mukai lattice 
such that $\Phi(u_1)=(0,0,1)$.
Since $\Phi(v)^2=v^2$ is odd, $\rk \Phi(v)$ is odd.
Hence $\langle v,u_1 \rangle=\langle \Phi(v),\Phi(u_1) \rangle=-\rk \Phi(v)$
is odd. 
\end{proof}



\begin{Prop}
Assume that there is $w \in \HH$ with $w^2=-1$.
If $\langle v,w \rangle \geq 0$,
then $\MM_{\sigma_\pm}(v) \setminus \MM_{\sigma_0}(v)^s$
is at least of codimension 2 unless
$\langle v, u_1 \rangle=1,2$. 
\end{Prop}

\begin{proof}
Since $\ell(u_1)=\ell(u_2)=2$,
by using a Fourier-Mukai transform,
we may assume that $u_1=(0,0,1)$, 
$\MM_{\sigma_-}(v)$ is the moduli of Gieseker semi-stable
sheaves and $\MM_{\sigma_+}(v)$ parametrizes the dual
of Gieseker semi-stable sheaves.
In this case, $\MM_{\sigma_0}(v)^s$ consists of $\mu$-stable locally free sheaves.
Then it is easy to see that the claim holds (see \cite[Thm. 2.1 Case B]{Yos16a}). 
We shall explain a similar argument as in the previous subsection. 

(I) Assume that $a_i^2<0$ for some $i$.
We may assume $a_1=b_1 w$.
(I-1)
Assume that $\#\{i \mid a_i^2=0 \}=2$.
Then we may set $a_2:=b_2 u_1, a_3:=b_3 u_2$.
We note that $\langle u_1,u_2 \rangle \geq 2$.
\begin{equation}
\begin{split}
& \sum_i(a_i^2-\dim \MM_{\sigma_-}(a_i))+\sum_{i<j}\langle a_i,a_j \rangle\\
 \geq & -[\tfrac{b_1^2}{2}]+\sum_{i \geq 2} b_1 \langle w,a_i \rangle
-b_2-b_3+b_2 b_3 \langle u_1,u_2 \rangle\\
\geq &  b_1 \langle w,v \rangle+[\tfrac{b_1^2+1}{2}] 
-b_2-b_3+b_2 b_3 \langle u_1,u_2 \rangle\\
\geq & b_2(b_3-1)+b_3(b_2-1)+[\tfrac{b_1^2+1}{2}]+b_1 \langle w,v \rangle
\geq [\tfrac{b_1^2+1}{2}].
\end{split}
\end{equation}
If $(b_2,b_3) \ne (1,1)$, then 
$b_2(b_3-1)+b_3(b_2-1) \geq 1$, and hence 
$\sum_i(a_i^2-\dim \MM_{\sigma_-}(a_i))+\sum_{i<j}\langle a_i,a_j \rangle
\geq 2$.
Assume that $b_2=b_3=1$. 
If 
$\sum_i(a_i^2-\dim \MM_{\sigma_-}(a_i))+\sum_{i<j}\langle a_i,a_j \rangle=1$,
then 
$n=3$,
$b_1=1$ and $\langle w,v \rangle=0$.
Then $\langle w,v \rangle=-b_1+\langle u_1+u_2,w \rangle=-1$,
which is a contradiction.

(I-2)
Assume that $\#\{i \mid a_i^2=0 \}=1$ and $n \geq 3$.
We may sssume that $a_2=b_2 u_p$ $(p \in \{1,2\})$ and $a_i^2>0$ 
for $i \geq 3$.
Then
\begin{equation}
\begin{split}
& \sum_i(a_i^2-\dim \MM_{\sigma_-}(a_i))+\sum_{i<j}\langle a_i,a_j \rangle\\
 \geq & -[\tfrac{b_1^2}{2}]+\sum_{i \geq 2} b_1 \langle w,a_i \rangle
-b_2+b_2 \sum_{i \geq 3}\langle u_p,a_i \rangle\\
\geq &  b_1 \langle w,v \rangle+[\tfrac{b_1^2+1}{2}] 
-b_2+b_2 \sum_{i \geq 3} \langle u_p,a_i \rangle\\
\geq & b_2(\langle u_p,v-w \rangle-1)
+[\tfrac{b_1^2+1}{2}]+b_1 \langle w,v \rangle
\geq [\tfrac{b_1^2+1}{2}].
\end{split}
\end{equation}

If $\langle w,u_p \rangle<0$, then
$\langle u_p,v-w \rangle=\langle u_p,v \rangle-\langle u_p,w \rangle
\geq \langle u_p,v \rangle+1 \geq 2$.
Hence
$\sum_i(a_i^2-\dim \MM_{\sigma_-}(a_i))+\sum_{i<j}\langle a_i,a_j \rangle 
\geq 2$.
Assume that $\langle w,u_p \rangle >0$.
%Then $\MM_{\sigma_0}(u_p)$ consists of $\sigma_0$-stable
%objects and defines a Fourier-Mukai transform.
Then $\sum_i(a_i^2-\dim \MM_{\sigma_-}(a_i))+\sum_{i<j}\langle a_i,a_j \rangle
\geq 2$ unless $\langle v,w \rangle=0$,
$b_1=1$, 
$\langle u_p,a_3 \rangle=1$ and $n=3$.
For the remaining case, $a_3=w+m u_p$ and
$M_{\sigma_-}(a_3)$ is isomorphic to the Hilbert scheme of points.
Then $v=2w+m u_1$, which implies 
$0=\langle v,w \rangle=-2+m \langle u_p,w \rangle$.
By Lemma \ref{Lem:FM:rank}, $\langle u_p,w \rangle$ is odd, and hence
$m=2$ and $\langle u_p,w \rangle=1$.
Hence $v=2(w+u_p)$ with $\langle u_p,w \rangle=1$.
In particular, $v^2=4$.

(I-3)
Assume that $\#\{i \mid a_i^2=0 \}=1$ and $n =2$.
We may sssume that $v=b_1 w+b_2 u_p$ $(p \in \{1,2\})$.
Then $0 \leq \langle v,w \rangle=-b_1+b_2 \langle u_p,w \rangle$
implies $\langle u_p,w \rangle \geq 1$. 
Hence $\MM_{\sigma_0}(u_p)$ consists of $\sigma_0$-stable
objects.
\begin{equation}
\begin{split}
&\sum_i(a_i^2-\dim \MM_{\sigma_-}(a_i))+\sum_{i<j}\langle a_i,a_j \rangle \\
=&-[\tfrac{b_1^2}{2}]-b_2+b_1 b_2 \langle w,u_p \rangle\\
\geq & b_2(b_1 \langle w,u_p \rangle/2-1).
\end{split}
\end{equation}
If $b_1 \geq 3$, then $b_2 \langle u_p, w \rangle \geq 3$.
$b_2(b_1 \langle w,u_p \rangle/2-1) \geq 3/2$.
Hence 
$\sum_i(a_i^2-\dim \MM_{\sigma_-}(a_i))+\sum_{i<j}\langle a_i,a_j \rangle
\geq 2$.

Assume that $b_1=2$.
If $\langle w,u_p \rangle \geq 3$, then
obviously
 $\sum_i(a_i^2-\dim \MM_{\sigma_-}(a_i))+\sum_{i<j}\langle a_i,a_j \rangle
\geq 2$.
If $\langle w,u_p \rangle <3$, Lemma \ref{Lem:FM:rank} implies
$\langle w,u_p \rangle=1$.
then
$b_2 \geq b_1 \geq 2$.
Hence 
 $\sum_i(a_i^2-\dim \MM_{\sigma_-}(a_i))+\sum_{i<j}\langle a_i,a_j \rangle
=b_2-2 \geq 2$ if $b_2 \geq 4$.
If $b_2=2,3$, then
$v=2(w+u_p), 2w+3 u_p$.
In particular, $v^2=4,8$ and $\langle v,u_p \rangle=2$.
 
If $b_1=1$, 
$\sum_i(a_i^2-\dim \MM_{\sigma_-}(a_i))+\sum_{i<j}\langle a_i,a_j \rangle
\geq 2$ unless $\langle w,u_p \rangle=1$.










(II)
Assume that $a_i^2 \geq 0$ for all $i$.
(II-1)
Assume that $a_1=b_1 u_1, a_2=b_2 u_2$ and
$\ell(u_1)=2$.
Then $\langle u_1,u_2 \rangle>0$ is even.
 
$\sum_i(a_i^2-\dim \MM_{\sigma_-}(a_i))+\sum_{i<j}\langle a_i,a_j \rangle
\geq -b_1-b_2+b_1 b_2 \langle u_1,u_2 \rangle+
\sum_{i \geq 3}\langle a_i,a_1+a_2 \rangle$.
If $\langle u_1,u_2 \rangle \geq 4$, then
$-b_1-b_2+b_1 b_2 \langle u_1,u_2 \rangle \geq 2$.
Assume that $\langle u_1,u_2 \rangle =2$.
Then $-b_1-b_2+2b_1 b_2 =
b_1(b_2-1)+b_2(b_1-1) \geq 0$.
If $n \geq 3$, then $\langle a_i,a_1+a_2 \rangle \geq 2$.
Hence we may assume that $n=2$.
Then $-b_1-b_2+2b_1 b_2 \geq 2$ unless
$(b_1,b_2)=(1,1),(1,2),(2,1)$.
For these cases, $v=u_1+u_2, 2u_1+u_2,u_1+2u_2$.
Assume that $\langle w,u_1 \rangle>0$. Then
$\langle u_1,w \rangle=1$ and $u_2=u_1+2w$.
Hence $v=2(w+u_1),2w+3u_1$ and 
$v \ne u_1+2u_2$.

(II-2)
Assume that $a_1=b_1 u_1$ and $a_i^2>0$ for $i \geq 2$.
If $n \geq3$, then 
$-b_1+\sum_{i \geq 2} \langle b_1 u_1,a_i \rangle +\langle a_2,a_3 \rangle
\geq 2$.
Assume that $n=2$. 
If $\langle u_1,a_2 \rangle \geq 3$, then
we also have $-b_1+\langle b_1 u_1,a_2 \rangle \geq 3$.
If $\langle u_1,a_2 \rangle=1,2$, then
we have a divisorial contraction.

\end{proof}



\subsubsection{}

Assume that $w^2=-2$ and $c_1(w)=D+(\rk w/2)K_X \mod 2$
for a nodal cycle $D$.
Then there is a $\sigma_0$-semi-stable object
$E_0$ with $v(E_0)=\pm w$.
In this case, we also see that $E_0$ is $\sigma_0$-stable.
 Let $E$ be a $\sigma_0$-semi-stable object such that
$v(E)^2=0$, $v(E)$ is primitive and
$\langle v(E),v(E_0) \rangle>0$.
Then $E$ is $\sigma_0$-stable.

It follows that we need only consider three cases: $l(w_1)=l(w_2)=2$, $l(w_1)=l(w_2)=1$, and $l(w_2)=2,l(w_1)=1$.

\subsubsection{The case $l(w_1)=l(w_2)=2$} 

Using Proposition \ref{Prop:Uhlenbeck morphism}, we can easily identify another sufficient condition for $\WW$ to be a totally semistable wall:

\begin{Lem}
Let $\WW$ be a potential isotropic wall with $l(w_i)=2$ for each $i$.  If there exists an isotropic wall $w\in\HH_{\WW}$ with $\langle v,w\rangle=1$, then $\WW$ is a totally semistable wall.  If, in addition, $\langle v,s\rangle\geq 0$, then $\WW$ induces a divisorial contraction.
\end{Lem}
\begin{proof}
By Lemma \ref{Lem:isotropic lattice}, $w$ must be effective and thus equal to either $w_1$ or $w_2$.  If $w=w_2$ then by Proposition \ref{Prop:Uhlenbeck morphism} and our assumption on $w_2$, we may assume $w=(0,0,1)$ so that $1=\langle v,w\rangle$ implies that $-v$ has rank 1 and $M_{\sigma_+}(v)$ is isomorphic to the Hilbert scheme of $\frac{v^2+1}{2}$ points.  It follows that $\WW$ is the Hilbert-Chow wall inducing the Hilbert-Chow morphism as in \cite[Proposition 13.1]{Nue14b}, so it is totally semistable and induces a divisorial contraction.

Otherwise $w=w_1$, and, as $w_2=w_1+\langle w_1,s\rangle s$ with $\langle w_1,s\rangle<0$, we get $$1\leq\langle v,w_2\rangle=\langle v,w_1\rangle+\langle w_1,s\rangle\langle v,s\rangle=1+\langle w_1,s\rangle\langle v,s\rangle.$$ Thus $\langle v,s\rangle\leq 0$.  If $\langle v,s\rangle<0$, then $\WW$ is totally semistable by Lemma \ref{Lem:semistable positive}, while if $\langle v,s\rangle=0$, then we must have $\langle v,w_2\rangle=1$ as well, so we may reduce to the case above.
\end{proof}


Thus we may assume $\langle v,s\rangle\geq 0$, $\langle v,w_1\rangle\geq 2$, and $\langle v,w_2\rangle\geq 3$.
\begin{Prop}
Assume that $\langle v,s\rangle\geq 0$, $\langle v,w_1\rangle\geq 2$, and $\langle v,w_2\rangle\geq 3$.  Then $M_{\sigma_0}^s(v)\neq\varnothing$.  Moreover, if $\WW$ induces a divisorial contraction, then $\langle v,w_1\rangle=2$, $\langle v,w_2\rangle=3$, or $\langle v,s\rangle=0$.  Otherwise, $\WW$ is either a flopping wall or not a wall at all.
\end{Prop}
\begin{proof}  As usual we let the Harder-Narasimhan filtration of $E$ with respect to $\sigma_-$ correspond to a decomposition $v=\sum_i a_i$.  We shall estimate the codimension of the sublocus of destabilized objects which is equal to
 \begin{equation}
\sum_i (a_i^2-\dim \MM_{\sigma_-}(a_i))+\sum_{i<j}\langle a_i,a_j \rangle.
\end{equation}


(I) We first assume that one of $a_i$ satisfies $a_i^2<0$.
For simpicity, we assume that $a_0=b_0 s$ with $s^2=-2$.

Assume that $a_1$ and $a_2$ are isotropic.
We may set $a_1=b_1 w_1$ and $a_2=b_2 w_2$.
Then 
 \begin{equation}\label{eq: spherical 2,2 case I}
\begin{split}
& \sum_i (a_i^2-\dim \MM_{\sigma_-}(a_i))+\sum_{i<j}\langle a_i,a_j \rangle\\
\geq & -b_0^2+\sum_{i \geq 1}b_0 \langle s,a_i \rangle
-b_1-b_2+b_1 b_2 \langle w_1,w_2 \rangle\\
\geq & b_0^2+b_0 \langle s,v \rangle-b_1-b_2+b_1 b_2 \langle w_1,w_2 \rangle \\
\geq & 1-b_1-b_2+b_1 b_2 \geq 0,
\end{split}
\end{equation} 
 and equality holds in the last inequality only if $\langle v, s\rangle=0,\langle w_1,w_2\rangle=1$, $v=a_0+a_1+a_2$ with $b_0=1$ and either $b_1=1$ or $b_2=1$.  But then $v=s+w_1+b_2 w_2$ or $s+b_1 w_1+w_2$.  But $\langle w_1,w_2\rangle=1$ implies $\langle s,w_1\rangle =-1$ and $\langle s,w_2\rangle=1$.  So in the latter case, $0\leq\langle s,v\rangle=-2-b_1+1$ and thus $b_1\leq -1$,which is impossible.  In the first case, $\langle v,w_2\rangle=2$, contrary to assumption.  Thus we must have strict inequality in \eqref{eq: spherical 2,2 case I}.  If $\codim\FF(a_0,a_1,...,a_n)^0=1$, then we must have $b_1 b_2=b_1+b_2$ whose only solution is $b_1=b_2=2$.  Thus $v=s+2w_1+2w_2$ and $\langle w_1,w_2\rangle=1$.  But then $\langle v,s\rangle=-2<0$, contrary to assumption.  So in fact $\codim\FF(a_0,a_1,...,a_n)^0\geq 2$ in this case.


Now assume that $a_1=b_1 w_j$ and $a_i^2>0$ for $i>1$.  Then 
 \begin{equation}
\begin{split}
 \sum_i (a_i^2-\dim \MM_{\sigma_-}(a_i))+\sum_{i<j}\langle a_i,a_j \rangle
&\geq  -b_0^2+\sum_{i \geq 1}b_0 \langle s,a_i \rangle
-b_1+\sum_{i \geq 2} b_1 \langle w_j,a_i \rangle\\
\geq & b_0 \langle s,v \rangle+b_0^2
-b_1+b_1 \langle w_j,a_2 \rangle 
\geq  b_0^2 \geq 1,
\end{split}
\end{equation}
with equality only if $v=s+b_1w_j+a_2$, $\langle w_j,a_2\rangle=1$ and $\langle s,v\rangle=0$.  But then $$\langle v,w_j\rangle=\langle s,w_j\rangle+\langle a_2,w_j\rangle=\langle s,w_j\rangle+1,$$ so $\langle s,w_1\rangle<0$ implies that $j=2$ and $\langle v,w_2\rangle\geq 3$ gives $\langle s,w_2\rangle\geq 2$.  Moreover, as $w_2=w_1+\langle w_1,s\rangle s$, $$1=\langle a_2,w_2\rangle=\langle a_2,w_1\rangle+\langle w_1,s\rangle\langle a_2,s\rangle\geq 1+\langle w_1,s\rangle\langle a_2,s\rangle,$$ so $\langle a_2,s\rangle\geq 0$ follows from $\langle w_1,s\rangle<0$.  But then $$0=\langle s,v\rangle=-2+b_1\langle s,w_2\rangle+\langle s,a_2\rangle\geq 2b_1-2\geq 0,$$ so we must have $b_1=1$, $\langle s,w_2\rangle=2$ and $\langle s,a_2\rangle=0$.  Moreover, by \eqref{eq: spherical isotropic constraint} we get $\langle w_1,w_2\rangle=(-\langle w_2,s\rangle)^2=4$.  Otherwise, $\codim\FF(a_0,a_1,...,a_n)^0\geq 2$.

We can now assume that there are no positive classes in the Harder-Narasimhan factors, i.e. $v=b_0 s+b_1 w_j$.  But $v^2>0$ forces $j=2$, so we may assume this outright.  Then $0 \leq \langle v,s \rangle=-2b_0+b_1 \langle w_2,s \rangle$, so our estimate becomes 

 \begin{equation}\label{eq: spherical 2,2 case I, c}
\begin{split}
\codim\FF(a_0,a_1)^0=&\sum_i (a_i^2-\dim \MM_{\sigma_-}(a_i))+\sum_{i<j}\langle a_i,a_j \rangle\\
= & -b_0^2-b_1+b_0b_1\langle s,w_2\rangle=-b_0^2+\frac{b_0b_1\langle s,w_2\rangle}{2}+b_1\left(\frac{b_0\langle s,w_2\rangle}{2}-1\right)\\
= & \frac{b_0}{2}\langle v,s\rangle+b_1\left(\frac{\langle v,w_2\rangle}{2}-1\right)\geq \frac{b_0}{2}\langle v,s\rangle+\frac{b_1}{2}\geq \frac{b_1}{2}\geq\frac{1}{2},
\end{split}
\end{equation}
as $\langle v,w_2\rangle\geq 3$.  If $b_1\geq 3$, then $\codim\FF(a_0,a_1)^0\geq 2$.  If $\langle v,s\rangle\geq 1$, then $$\codim\FF(a_0,a_1)^0\geq \frac{b_0}{2}+\frac{b_1}{2},$$ which shows that $\codim\FF(a_0,a_1)^0\geq 2$ unless $v=s+w_2$ with $\langle v,w_2\rangle=3$.  If instead $\langle v,s\rangle=0$, then as $3\leq\langle v,w_2\rangle=b_0\langle s,w_2\rangle$, the second line of \eqref{eq: spherical 2,2 case I, c} becomes $$\codim\FF(a_0,a_1)^0=b_0^2-b_1\geq\frac{b_0^2}{3}\geq\frac{4}{3}.$$    So $\codim\FF(a_0,a_1)^0\geq 2$ in this case. 

Finally, assume that other than $a_0=b_0 s$, $a_i^2>0$ for all $i>0$.  Then the estimate becomes 

\begin{equation}\label{eq: spherical 2,2 case I, d}
\begin{split}
\codim\FF(a_0,...,a_n)^0=&\sum_i (a_i^2-\dim \MM_{\sigma_-}(a_i))+\sum_{i<j}\langle a_i,a_j \rangle\\
= & b_0^2+b_0\langle s,v\rangle+\sum_{0<i<j}\langle a_i,a_j\rangle\geq b_0^2\geq 1,
\end{split}
\end{equation}

with equality only if $\langle s,v\rangle=0$ and $v=s+a_1$.  



(II) We next assume that $a_i^2 \geq 0$ for all $i$.
 

We assume $a_1=b_1 w_1$ and $a_2=b_2 w_2$.
Then 
 \begin{equation}\label{eq: spherical 2,2 case II,a}
\begin{split}
\codim\FF(a_1,...,a_n)^0=& \sum_i (a_i^2-\dim \MM_{\sigma_-}(a_i))+\sum_{i<j}\langle a_i,a_j \rangle\\
\geq &
-b_1-b_2+b_1 b_2 \langle w_1,w_2 \rangle
+\sum_{i \geq 3} \langle a_i,a_1\rangle+\sum_{i\geq 3}\langle a_i,a_2 \rangle\\
= & -b_1-b_2+\sum_{i\geq 3}\langle b_1w_1,a_i\rangle+b_2\langle w_2,v\rangle\\
\geq &
-b_1+2b_2+\sum_{i\geq 3}\langle b_1w_1,a_i\rangle.
\end{split}
\end{equation}
If $a_3\neq 0$, then $$-b_1+\sum_{i\geq 3}\langle b_1 w_1,a_i\rangle\geq 0,$$ so $\codim\FF(a_1,...,a_n)^0\geq 2$.  Otherwise, $v=b_1w_1+b_2w_2$ and \eqref{eq: spherical 2,2 case II,a} becomes 
\begin{equation}\label{eq: spherical 2,2 case II,aa}
\begin{split}
\codim\FF(a_1,a_2)^0\geq&-b_1-b_2+b_1b_2\langle w_1,w_2\rangle\\
=& -b_1-b_2+\frac{b_1}{2}\langle v,w_1\rangle+\frac{b_2}{2}\langle v,w_2\rangle\geq\frac{b_2}{2}\geq\frac{1}{2}.
\end{split}
\end{equation}

Thus $\codim\FF(a_1,a_2)^0\geq 2$ if $b_2\geq 3$.  As $0\leq\langle v,s\rangle=\langle w_2,s\rangle(b_2-b_1)$ we must have $b_2\geq b_1$.  So if $b_2=1$, then the only solution is $b_1=1$ and $\langle w_1,w_2\rangle=3$.  If $b_2=2$, then $$2\langle w_1,w_2\rangle=b_2\langle w_1,w_2\rangle\geq b_1\langle w_1,w_2\rangle=\langle v,w_2\rangle\geq 3,$$ which gives $\langle w_1,w_2\rangle\geq 2$, so \eqref{eq: spherical 2,2 case II,aa} becomes $$\codim\FF(a_1,a_2)^0\geq -b_1-2+2b_1\langle w_1,w_2\rangle\geq 4b_1-2\geq 2.$$


Now we assume that $a_1=b_1 w_j$ and $a_i^2>0$ for $i \geq 2$.
In this case, we also see that
\begin{equation}
\begin{split}
\codim\FF(a_1,...,a_n)^0=&\sum_i (a_i^2-\dim \MM_{\sigma_-}(a_i))+\sum_{i<j}\langle a_i,a_j \rangle\\
\geq&-b_1+\sum_{i>1}b_1\langle w_j,a_i\rangle+\sum_{1<i<k}\langle a_i,a_k\rangle\\
=&b_1(\langle v,w_j\rangle-1)+\sum_{1<i<k}\langle a_i,a_k\rangle\\
\geq&b_1(\langle v,w_j\rangle-1).
\end{split}
\end{equation}

So if $j=2$, then $\langle v,w_j\rangle\geq 3$ and $\codim\FF(a_1,...,a_n)^0\geq 2$.  If $j=1$, then we just have $\langle v,w_j\rangle\geq 2$, so $\codim\FF(a_1,...,a_n)^0\geq 2$ unless $v=w_1+a_2$ with $\langle v,w_1\rangle=2$.

Finally, if $a_i^2>0$ for all $i$, then $\codim\FF(a_1,...,a_n)^0\geq 2$ by Proposition \ref{Prop:HN filtration all positive classes}.
\end{proof}



\subsection{Isotropic walls with an exceptional class}



Assume that $w^2=-1$.
We may assume that there is a $\sigma_0$-semi-stable object $E_0$
with $v(E_0)=w$.

If there is a properly $\sigma_0$-semi-stable object $E$ with $v(E)^2 \leq 0$,
then there is a decomposition $v(E)=\sum_i n_i a_i$ such that  
$a_i^2 \geq -1$ and $a_i \ne a_j$ $(i \ne j)$.
If $a_i^2 \geq 0$ for all $i$, then $\langle a_i,a_j \rangle>0$, which implies
$v(E)^2>0$. Hence one of $a_i$ is $\pm w$.
If $E=E_0$, then $w=w+\sum_{j \ne i} n_j a_j$ implies $E_0$ is 
$\sigma_0$-stable.



Let $E$ be a $\sigma_0$-semi-stable object such that
$v(E)^2=0$, $v(E)$ is primitive and
$\langle v(E),v(E_0) \rangle>0$.
Then $E$ is $\sigma_0$-stable.
Indeed if $E$ is properly $\sigma_0$-semi-stable, then
for the decomposition $v(E)=\sum_j n_j a_j$,
$\langle v(E),a_j \rangle >0$ if $a_j^2 \geq 0$.
We may assume that $a_0=w$.
Since $v(E) \ne w$ and
$0=v(E)^2=\sum_j \langle v(E),a_j \rangle$,
$0 > \langle v(E),n_0 a_0 \rangle$, which implies
$\langle v(E),w \rangle<0$.
Conversely if there is a $\sigma_0$-stable object $E$ with 
$v(E)^2=0$, then 
$\Hom(E_0,E)=\Hom(E,E_0(K_X))=0$ imply
$\langle w,v(E) \rangle \geq 0$.
If the equality holds, then 
$\HH$ is not hyperboloc.
Hence $\langle w, v(E) \rangle>0$.

Let $L^+$ be the half pane of $\HH$ such  that
$v \in L^+$ if and only if $Z(v) \in \R_{\leq 0}$. 
Then the effective cone belongs to $L^+$.
Hence
\begin{Lem}
The effective cone is $\R_{\geq 0} u_1+\R_{\geq 0} w$ with 
$\langle w,u_1 \rangle>0$
and $P^+=\R_{\geq 0} u_1+\R_{\geq 0}u_2$,
where $u_2=u_1+2\langle u_1,w \rangle w$. 
\end{Lem}

\begin{proof}
Let $\pm u_1,\pm u_2$ be primitive
and isotropic Mukai vectors.
We may assume that $w,u_1,u_2 \in L^+$.
Replacing $u_1$ by $u_2$ if necessary, we may assume that
$u_1$ is one of the boundary of the effective cone.
Since $u_1$ is indecomposable in the effective cone,
there is a $\sigma_0$-stable object $E_1$ with
$v(E_1)=u_1$. Then $\langle u_1,w \rangle>0$, which implies
$u_1+2\langle u_1,w \rangle w$ is effective.
Hence $u_2=u_1+2\langle u_1,w \rangle w$.
Therefore the first claim holds.

The second claim is obvious.
\end{proof}

We have 
\begin{equation}
\phi_{\sigma_\pm}(u_1) < \phi_{\sigma_\pm}(u_2)<
\phi_{\sigma_\pm}(w)
\end{equation}
or
\begin{equation}
\phi_{\sigma_\pm}(u_1) > \phi_{\sigma_\pm}(u_2)>
\phi_{\sigma_\pm}(w).
\end{equation}







\subsubsection{Estimate for the case where $\ell(u_1)=1$. }  

Assume that $\ell(u_1)=1$.

Assuming $\langle v,v(E_0) \rangle \geq 0$
for any spherical object  $E_0$ and also any exceptional object 
$E_0$,
we shall estimate 
 \begin{equation}\label{eq:iso:codim}
\sum_i (a_i^2-\dim \MM_{\sigma_-}(a_i))+\sum_{i<j}\langle a_i,a_j \rangle.
\end{equation}

(I)
We first assume that one of $a_i$ satisfies $a_i^2<0$.
Since \eqref{eq:iso:codim} is symmetric
with respect to $a_i$,
we may assume that $a_1=b_1 w$ with $w^2=-1$
(see also Remark \ref{Rem:order}).
Using Lemma \ref{Lem:negative stable classes}, we get $$(a_1^2-\dim\MM_{\sigma_-}(a_1))=\left\lceil\frac{1-b_1^2}{2}\right\rceil\geq-\frac{b_1^2}{2}$$ in this case.
Moreover,  $\ell(u_2)=\ell(u_1)=1$.
We also have$\langle v,w \rangle \geq 0$.

Assume that $a_2$ and $a_3$ are isotropic.
We may set $a_2=b_2 u_1$ and $a_3=b_3 u_2$.
Then 
 


Assume that $a_2=b_2 u_1$ and $a_i^2>0$ for $i>2$. 
 \begin{equation}
\begin{split}
& \sum_i (a_i^2-\dim \MM_{\sigma_-}(a_i))+\sum_{i<j}\langle a_i,a_j \rangle\\
\geq & -\frac{b_1^2}{2}+\sum_{i \geq 2}b_1 \langle w,a_i \rangle
-[\tfrac{b_2}{2}]+\sum_{i \geq 3} b_2 \langle u_1,a_i \rangle\\
\geq & b_1 \langle w,v \rangle+\frac{b_1^2}{2}
-[\tfrac{b_2}{2}]+b_2 \langle u_1,a_3 \rangle \\
\geq & \frac{b_1^2}{2} -[\tfrac{b_2}{2}]+b_2 \geq \frac{3}{2}.
\end{split}
\end{equation}

Assume that 
$v=b_1 w+b_2 u_1$.
Then $0 \leq \langle v,w \rangle=-b_1+b_2 \langle u_1,w \rangle$.

 \begin{equation}
\begin{split}
& \sum_i (a_i^2-\dim \MM_{\sigma_-}(a_i))+\sum_{i<j}\langle a_i,a_j \rangle\\
\geq & -[\tfrac{b_1^2}{2}]-[\tfrac{b_2}{2}]+b_1 b_2 \langle w,u_1 \rangle\\
\geq & \frac{b_1^2}{2} -[\tfrac{b_1^2}{2}]+
\frac{b_1}{2}\langle v,w \rangle+\frac{b_2}{2}
(\langle v,u_1 \rangle-1).
\end{split}
\end{equation}
Assume that $\langle v,u_1 \rangle \geq 2$.
If $\langle u_1,w \rangle \geq 2$, 
then we see that
$\sum_i (a_i^2-\dim \MM_{\sigma_-}(a_i))+
\sum_{i<j}\langle a_i,a_j \rangle \geq 2$.
If $\langle u_1,w \rangle =1$, then
$b_1 \geq 2$.
Then we see that
$\sum_i (a_i^2-\dim \MM_{\sigma_-}(a_i))+
\sum_{i<j}\langle a_i,a_j \rangle \geq 2$
unless $v=2(w+u_1)$.

If $\langle v,u_1 \rangle =1$, that is,
$v=w+u_1,w+2u_1$,  then
$b_1=\langle w,u_1 \rangle=1$.
If $b_2 \geq 3$, then
$\sum_i (a_i^2-\dim \MM_{\sigma_-}(a_i))+
\sum_{i<j}\langle a_i,a_j \rangle \geq 2$.
If $b_2=1,2$, then
$\sum_i (a_i^2-\dim \MM_{\sigma_-}(a_i))+
\sum_{i<j}\langle a_i,a_j \rangle =1$.

\begin{Rem}
Assume that $\langle w,u_1 \rangle=1$.
In this case, $\MM_{\sigma_0}(u_1)$
consists of stable objects.
If $v=w+u_1$, then $\langle v^2 \rangle=1$.
If $v=w+2 u_1$, then $\langle v^2 \rangle=3$ and
the codimension 1 locus is contractible.
\end{Rem}


Assume that $a_1=b_1 w$ and $a_i^2>0$ for $i>1$.
Then
 \begin{equation}
\begin{split}
& \sum_i (a_i^2-\dim \MM_{\sigma_-}(a_i))+\sum_{i<j}\langle a_i,a_j \rangle\\
\geq & -[\tfrac{b_1^2}{2}]+\sum_{i \geq 2}b_1 \langle w,a_i \rangle 
+\sum_{1<i<j}\langle a_i,a_j \rangle\\
\geq & b_1 \langle w,v \rangle+[\tfrac{b_1^2+1}{2}] 
+\sum_{1<i<j}\langle a_i,a_j \rangle\geq 2,
\end{split}
\end{equation}
unless $v=a_1+a_2$, 
$b_1$ and $\langle w,v \rangle=0$.
If $v=w+a_2$ with $\langle w,v \rangle=0$,
then $a_2^2=(v-w)^2=v^2-1>0$. Thus $v^2>1$.
Conversely under these conditions,
we have a codimension 1 locus in $\MM_{\sigma_+}(v)$
parameterizing
\begin{equation}
0 \to E_1 \to E \to E_2 \to 0
\end{equation}
where we assume that $\phi_{\sigma_-}(w)<\phi_{\sigma_-}(v)$ and
$E_2 \in \{E_0,E_0(K_X)\}$.
This divisor is not contracted.
If $v^2 \geq 3$, then
the codimension 2 locus
in 
$\MM_{\sigma_+}(v)$
parameterizing $E$ fitting in an exact sequence 
\begin{equation}
0 \to E_1 \to E \to E_0 \oplus E_0(K_X) \to 0
\end{equation}
is contacted and a general fiber is $\P^1 \times \P^1$.



\begin{Rem}\label{Rem:order}
If $\phi_{\sigma_-}(w)<\phi_{\sigma_-}(u_2)<\phi_{\sigma_-}(v)
<\phi_{\sigma_-}(u_1)$, then $a_n \in \Z_{>0}w$, $a_{n-1} \in \Z_{>0}u_2$
and $a_1 \in \Z_{>0} u_1$.
\end{Rem}

(II) We next assume that $a_i^2 \geq 0$ for all $i$.
 

We assume $a_1=b_1 u_1$ and $a_2=b_2 u_2$.
Then 
 \begin{equation}
\begin{split}
& \sum_i (a_i^2-\dim \MM_{\sigma_-}(a_i))+\sum_{i<j}\langle a_i,a_j \rangle\\
\geq &
-[\tfrac{b_1}{2}]-[\tfrac{b_2}{2}]+b_1 b_2 \langle u_1,u_2 \rangle
+\sum_{i \geq 3} \langle a_i,a_1+a_2 \rangle\\
\geq & 
-[\tfrac{b_1}{2}]-[\tfrac{b_2}{2}]+b_1 b_2 \langle u_1,u_2 \rangle\\
\geq & \frac{b_1(b_2-1)+b_2(b_1-1)}{2}.
\end{split}
\end{equation}
If one of $b_i$ is 1, then 
$-[\tfrac{b_1}{2}]-[\tfrac{b_2}{2}]+b_1 b_2 \langle u_1,u_2 \rangle>0$.
Assume that 
$\sum_i (a_i^2-\dim \MM_{\sigma_-}(a_i))+
\sum_{i<j}\langle a_i,a_j \rangle=1$.
Then $v=a_1+a_2$.
If $b_1,b_2 \geq 2$, then
$\sum_i (a_i^2-\dim \MM_{\sigma_-}(a_i))+
\sum_{i<j}\langle a_i,a_j \rangle \geq 2$
unless $v=2(u_1+u_2)$.
If $\sum_i (a_i^2-\dim \MM_{\sigma_-}(a_i))+
\sum_{i<j}\langle a_i,a_j \rangle=1$,
then $v=2 u_1+u_2, u_1+u_2$ with $\langle u_1,u_2 \rangle=1$.


We assume that $a_1=b_1 u_1$ and $a_i^2>0$ for $i \geq 2$.
In this case, we also see that
$\sum_i (a_i^2-\dim \MM_{\sigma_-}(a_i))+\sum_{i<j}\langle a_i,a_j \rangle>1$
unless $s=2$, 
$\langle v,u_1 \rangle=1$ and $b_1=1,2$, i.e.,
 $v=u_1+a_2,2u_1+a_2$ with $\langle v,u_1 \rangle=1$.


\begin{Prop}\label{Prop:non-empty}
\begin{enumerate}
\item[(1)]
Let $v \in P^+$ be a Mukai vector.
Assume that $v^2 \geq 0$ and $\langle v,w \rangle \geq 0$.
If $v$ is primitive or $v^2>0$, then
$\MM_{\sigma_0}(v)^s \ne \emptyset$.
\item[(2)]
If $v^2 \geq -1$ and $\langle v,u_1 \rangle=1$, then
we also have $\MM_{\sigma_0}(v)^s \ne \emptyset$.
If $v^2 \geq 0$ and $\langle v,w \rangle=0$, then
we also have $\MM_{\sigma_0}(v)^s \ne \emptyset$. 
\end{enumerate}
\end{Prop}

\begin{proof}
(1)
We note that $\MM_{\sigma_0}(v)^s=
\MM_{\sigma_+}(v)^s \cap \MM_{\sigma_-}(v)^s$.
If $v^2>0$ or $v$ is primitive, 
then $\MM_{\sigma_\pm}(v)^s$ are open dense substacks
of $\MM_{\sigma_\pm}(v)$. Hence the claim holds.

(2)
In this case, $\HH=\Z w+\Z u_1$ and $u_2=u_1+2n w$, where
$n=\langle u_1,w \rangle$.
We set $v=x u_1+y w$, $(x,y \in \Z)$.
Then $\langle v,u_1 \rangle=1$ implies
$n=y=1$. Since $v^2=-1+2x$,
$x \geq 0$. Therefore $v=w$ or $v=x u_1+w$, $(x \geq 1)$.
For the second case,
since $\langle v,w \rangle=x-1$,
(1) implies
 $\MM_{\sigma_0}(v)^s \ne \emptyset$.
If $\langle v,w \rangle=0$, then $\langle v-w,w \rangle=1$ and
$(v-w)^2=v^2-1 \geq -1$. 
Moreover $(v-w)^2=-1$ implies $v-w =\pm w$. 
Therefore $(v-w)^2 \geq 0$. 
Hence (1) implies the claim.
\end{proof}

\begin{Cor}
Assume that $\langle v,u_1 \rangle=1$.
If $(v-k u_1)^2 \geq -1$ $(k>0)$, then
$\MM_{\sigma_0}(v-k u_1)^s \ne \emptyset$.
\end{Cor}

\begin{proof}
The claim follows from $\langle v-k u_1,u_1 \rangle=1$.
\end{proof}


\begin{Prop}
Assume that $\langle v,w \rangle \geq 0$.
Then $\dim(\MM_{\sigma_-}(v) \setminus \MM_{\sigma_0}(v)^s) \geq 1$
and the equality holds if and only if 
$\langle v,u_1 \rangle=1$ and $v^2 \geq 1$ or
$\langle v,w \rangle=0$ and $v^2 \geq 0$.
Moreover if $\langle v,u_1 \rangle=1$ and $v^2 \geq 3$, then
$D:=\MM_{\sigma_-}(v) \setminus \MM_{\sigma_0}(v)^s$
can be contracted.
\end{Prop}

\begin{proof}
If $\langle v,u_2 \rangle=1$, then
$\langle v,u_1 \rangle+2\langle w,u_1 \rangle \langle v,w \rangle=1$.
Since $\langle v,u_1 \rangle > 0$ and $\langle v,w \rangle \geq 0$,
$\langle v,u_1 \rangle=1$ and $\langle v,w \rangle=0$.

Assume that $v^2 \geq 3$. Then $(v-2u_1)^2 \geq -1$.
We set $D:=\MM_{\sigma_-}(v) \setminus \MM_{\sigma_0}(v)^s$.
Then a general member $E \in D$ fits in an exact sequence
\begin{equation}
0 \to E_1 \to E \to E_2 \to 0
\end{equation}
where $E_1 \in \MM_{\sigma_0}(v-2u_1)^s$ and 
$E_2 \in \MM_{\sigma_0}(2u_1)^s$.
Assume that $\langle v,w \rangle=0$ and $v^2 \geq 1$.
We set $D:=\MM_{\sigma_-}(v) \setminus \MM_{\sigma_0}(v)^s$.
Then a general member $E \in D$ fits in an exact sequence
\begin{equation}
0 \to E_1 \to E \to E_2 \to 0
\end{equation}
where $E_1 \in \{ E_0, E_0(K_X) \}$ and 
$E_2 \in \MM_{\sigma_0}(v-w)^s$.
\end{proof}

\begin{Rem}
If $\langle v,w \rangle=0$ and $\langle v,u_1 \rangle=1$, then
$v=w+u_1$ with $\langle w,u_1 \rangle=1$. In particular $v^2=1$.
\end{Rem}





If there is no $(-1)$ vector in $\HH$ and $\ell(u_1)=\ell(u_2)=1$, then
a similar claim also holds.

\begin{Prop}
Assume that $\langle v,w \rangle \geq 0$.
Then $\dim(\MM_{\sigma_-}(v) \setminus \MM_{\sigma_0}(v)^s) \geq 1$
and the equality holds if and only if 
$\langle v,u_1 \rangle=1$ and $v^2 \geq 1$.
Moreover if $\langle v,u_1 \rangle=1$ and $v^2 \geq 3$, then
$D:=\MM_{\sigma_-}(v) \setminus \MM_{\sigma_0}(v)^s$
can be contracted.
\end{Prop}



\begin{Lem}
Assume that $\HH$ is non-isotropic.  If $v$ is the minimal class in its $G_{\HH}$-orbit and there is no spherical or exceptional class orthogonal to $v$, then the set of $\sigma_0$-stable objects in $M_{\sigma_+}(v)$ has complement of codimension at least two.
\end{Lem}
\begin{proof}
Consider the complement of the set of $\sigma_0$-stable objects in $M_{\sigma_+}(v)$ and take the relative Harder-Narasimhan filtration on any irreducible component.  The proof is similar to that of \ref{Lem:non-isotropic no totally semistable wall}, and we use the same notation.  

Under the current assumptions, however, we can improve the inequality $a^2\leq v^2$ if $I\neq\{1,\ldots,n\}$.  Indeed, if we write $b:=\sum_{i\in I^c}a_i$, then the assumptions imply that $\langle v,b\rangle>0$.  If $b^2>0$, then we automatically have $\langle a,b\rangle\geq 2$, so $$v^2=a^2+2\langle a,b\rangle+b^2\geq a^2+5.$$  If, instead, $b^2<0$, then $\langle v,b\rangle>0$ gives $$a^2=v^2-2\langle v,b\rangle+b^2\leq v^2-3,$$ so that $a^2\leq v^2-3$ in any case.  The analogue of \eqref{non-isotropic estimate} is now 
\begin{equation}
\sum_{i<j,(i,j)\in(I^2)^c}\langle a_i,a_j\rangle\geq\frac{3}{2}-\frac{1}{2}\sum_{i\in I^c}a_i^2.
\end{equation}
Using this estimate in \eqref{codim estimate} gives 
\begin{align}
\begin{split}
\codim\FF^0(a_1,\ldots,a_n)&\geq\frac{3}{2}+\sum_{i\in I^c}(\frac{a_i^2}{2}-\dim\MM_H(a_i)^{ss})+\sum_{i<j,(i,j)\in (I^2)^c}\langle a_i,a_j\rangle\\
&=\frac{3}{2}+\sum_{a_i\mbox{ exceptional}}(-\frac{1}{2}l_i^2+2\lfloor\frac{l_i}{2}\rfloor^2+2\lfloor\frac{l_i}{2}\rfloor+1)+\sum_{i<j,(i,j)\in I^2}\langle a_i,a_j\rangle\\
&\geq\frac{3}{2}+|I|(|I|-1)\geq\frac{3}{2}.
\end{split}
\end{align}
Thus if $|I|\neq n$, then $\codim\FF^0(a_1,\ldots,a_n)\geq 2$.  

If instead $|I|=n$, then the estimate is simpler and we get 
\begin{equation}
\codim\FF^0(a_1,\ldots,a_n)=\sum_{i<j}\langle a_i,a_j\rangle\geq n(n-1)\geq 2,
\end{equation}
as $n\geq 2$.  
\end{proof}



\begin{Cor}
Assume that $\HH$ is non-isotropic and that there does not exist a spherical or exceptional class orthogonal to $v$.  Then a potential wall associated to $\HH$ cannot induce a divisorial contraction except possibly when $X$ is nodal and $v^2=2$.  
\end{Cor}
\begin{proof}
The proof is identical to that of \cite[Corollary 7.3]{BM14b} upon recognizing that since $M_{\sigma_+}(v)$ and $M_{\sigma_+}(v_0)$ have torsion canonical divisor, any birational map between them is an isomorphism in codimension one.  It is for this purpose that we exclude $v^2=2$ and $X$ nodal.  
\end{proof}

Before we can prove the proposition, we must modify some of the lemmas used in \cite{BM14b}, namely Lemmas 9.2 and 9.3, to include the non-primitive case.
\begin{Lem} \label{lem:findparallelogram}
Let $M$ be a lattice of rank two, and $C \subset M \otimes \R^2$ be a convex cone not containing a
line. If $v \in M \cap C$ can be written as the sum $v = a + b$ of two classes in
$a, b \in M \cap
C$, neither of which is proportional to $v$, then it can be written as a sum $v = a' + b'$ of two classes $a', b' \in M \cap C$, neither of which is proportional to $v$, in such a way that 
the parallelogram with vertices $0, a', v, b'$ does not contain any other lattice point besides
its vertices and those on the ray $\R_{>0}v$.
\end{Lem}
\begin{proof}
The proof proceeds as in \cite[Lemma 9.2]{BM14b}.  If the parallelogram with vertices $0,a,v,b$ contains an additional lattice point $a'$ which is not on the ray $\R_{>0}v$, then we may replace $a$ by $a'$ and $b$ by $v-a'$, which also cannot lie on the ray $\R_{>0}v$.  This process terminates.
\end{proof}

\begin{Lem} \label{lem:parallelogram}
Let $a, b, v \in \HH \cap C_\WW$ be effective classes with $v = a + b$.
Assume that the following conditions are satisfied:
\begin{itemize}
\item The phases of $a, b$ satisfy $\phi^+(a) < \phi^+(b)$.
\item The objects $A, B$ are $\sigma_+$-stable with $v(A) = a, v(B) = b$.
\item The parallelogram in $\HH \otimes \R$ with vertices $0, a, v, b$ has positive area and does not contain any other
lattice point except possibly those on the ray $\R_{>0}v$.
\item The extension $A \into E \onto B$ satisfies $\Hom(B, E) = 0$.
\end{itemize}
Then $E$ is $\sigma_{+}$-semistable.
\end{Lem}
\begin{proof}
A careful reading of the proof of \cite[Lemma 9.3]{BM14b} shows that it remains valid under these conditions to give the above claim.
\end{proof}


\todo{What happens in the second case?}


For a stable sheaf $E$ fitting in the exact sequence

$$
0 \to E \to O_X \oplus K_X \to k_p \oplus k_q \to 0,
$$

the S-equivalence class is
$(k_p \oplus k_q)[-1] \oplus (O_X \oplus K_X)$.

A description of the fiber:

We set $F:=O_X \oplus K_X$.
Then the quotient $F \to k_p \oplus k_q$ is parametrized by
$\P(\Hom(F,k_p)) \times \P(\Hom(F,k_q))/Aut(F)$.
Since $Aut(F)/\C^*=(\C^* \times \C^*)/\C^* \cong \C^*$


$\P(\Hom(F,k_p))=\C^*  \cup \{x_1,x_2\}$, where $x_1=(1:0)$ and $x_2=(0:1)$,
and

$\P(\Hom(F,k_q))=\C^*  \cup \{y_1,y_2\}$, where $y_1=(1:0)$ and $y_2=(0:1)$.

For $(x,y) \in \C^* \times \C^*$, we have exact sequences

$
0 \to I_{p,q} \to E \to K_X \to 0,
0 \to I_{p,q}(K_X) \to E \to O_X \to 0
$

Hence $E$ is stable.
If $(x,y) =(x_1,y_1), (x_2,y_2)$, then
$E$ is not semi-stable, since $E$ contaions $O_X$ or $K_X$.
If $(x,y) \ne (x_1,y_1),(x_2,y_2)$, then $E$ is semi-stable.


Hence set-theoretically, we have a parameter space

$(\P^1 \times \P^1 \setminus \{(x_1,y_1),(x_2,y_2)\})/\C^*$
which seems to be bijective to $\P^1$.


But then $z=-w$ is effective.  Now let $\mathbf{Q}$ be the parallelogram with vertices $0,z,v,v-z$, and consider the function $f(a)=a^2$ on $\mathbf{Q}$.  By homogeneity, the minimum of $f$ occurs on one of the boundary segments; thus $$(z+t(v-z))^2=z^2+2t\langle z,v-z\rangle+t^2(v-z)^2$$

Consider the $(-1)$-reflection $$\rho_z(v):=v+2\langle v,z\rangle z$$ induced by the FM transform associated to an exceptional object $S$ with $v(S)=z$.  Set $v'=\rho_z(v)-z=v-(2\langle v,w\rangle+1)z$, and consider the parallelogram $\mathbf{P}$ with vertices $0,(2\langle w,v\rangle+1)z,v,v'$.  Clearly, $v'^2\geq -1$ and $\langle z,v'\rangle=\langle v,w\rangle+1\geq 2$.  The lattice points of $\mathbf{P}$ are given by $kz$ and $v'+kz$ for integral $k\in[0,2\langle v,w\rangle+1]$.  Indeed, otherwise these would be an additional lattice point in the parallelogram with vertices $0,z,v,v-z$.  It follows that the parallelogram with vertices $0,z,\rho_z(v),v'$ has no lattices points other than its vertices.  

Let $Z$ and $F$ be $\sigma_-$-stable objects of class $z$ and $v'$, respectively, and assume, without loss of generality, that $\phi^-(z)<\phi^-(v)$.  Then any non-trivial extension $$0\to Z\to E\to F\to 0$$ has class $\rho_z(v)$ and is $\sigma_-$-stable by \cite[Lemma 9.3]{BM14b}.  As $\ext^1(F,Z)\geq2$ we get a positive dimensional projective space worth of $\sigma_-$-stable extensions contracted by $\pi^-$.  By \eqref{eq:iso:Phi}, the corresponding FM transform induces an isomorphism $M_{\sigma_+}(v)\isomor M_{\sigma_-}(\rho_z(v))$ which preserves the $S$-equivalence relation with respect to $\sigma_0$.  Thus we get a positive dimensional contracted fiber of $\pi^+$.




